\begin{cvblock}{Research}
  \end{cvblock}
I've been involved in fusion plasma science since my M.Sci. thesis in
Physics in 1999. During these \FPtrunc\mydegree{\mydegree}{0}\mydegree\ years I've tried to expand as much as
possible my personal research skills focusing in particular on collection, analysis, interpretation and modeling of experimental data
in fusion oriented experiments (Reversed Field Pinches,
Tokamaks and Stellarators), with particular emphasis on the comparison
with theoretical and numerical results. Main research subjects
together with some of the relevant publications may be summarized as follow: 
\begin{description}[labelindent=0.5pt, labelsep*=0.4em, leftmargin=!, itemsep=0.05ex]
\item[(a) Electromagnetic turbulence induced transport:]with
emphasis on anomalous transport studies induced by different source of
turbulence: electrostatic as Drift-induced or interchange induced
transport, or electromagnetic including the role of magnetic flutter fluxes
in the mechanism of particle and energy losses \cite{Vianello:2015ek, Vianello:2016bm}
\item[(b) Statistical analysis of plasma turbulence:] with the use of advanced statistical tool (as Wavelet Transforms, Local Intermittency
  Measurements, Waiting Time distribution) and dynamical system
  model as Self-Organized Criticality (SOC) systems, shell-models \cite{Antoni:2001p3221, Sattin:2011p4955}
\item[(c) Blobs and ELM filaments:]non linear coherent structures have been experimentally investigated. The research
includes studies on the generation and evolution of these
structures including
their parallel dynamics with emphasis on turbulent \emph{blobs} and
ELM \emph{filaments} \cite{Spolaore:2009p4115, PhysRevLett.106.125002}
\item[(d) Sheared flow generation:] Non linear interaction
between turbulence and sheared flows with emphasis on the role of Maxwell and Reynolds stress in the
momentum generation of edge flow \cite{Vianello:2005p1976, Vianello:2006p1149}
\item[(e) Magnetic topology and its relation with plasma flow:] with
emphasis on the effect of non-axysimmetric magnetic field perturbation on kinetic
properties of the plasma, as plasma flow, ambipolar electric field and
Plasma Wall Interaction \cite{Vianello:2013jt, Rea:2015he}
\item[(f) Relationship between divertor condition and upstream SOL and
  pedestal properties:] with emphasis on the so-called \emph{shoulder
    formation} and enhanced SOL transport observed in high density
  regimes both in L and H-Mode \cite{Vianello:2017ku, vianello:nf2019}
\end{description}

% Among the results the following should be highlighted:
% \begin{enumerate}[itemsep=0.05ex, label=\textbf{\roman*}]
% \item First experimental proof of non applicability of \emph{Self
%     Organized Criticality} paradigm to edge plasma
%   turbulence \parencite{Spada:2001p3574,Antoni:2001p3221}
% \item First experimental evidence of non-linear generation of edge
%   flow in Reversed Field Pinches through Reynolds stress
%   mechanism \parencite{Vianello:2005p1976,Vianello:2005p2671}
% \item First experimental measurements of parallel current associated
%    to coherent structures in a fusion relevant
%    plasma \parencite{Spolaore:2009p4115} 
% \item First experimental evidence of the existence of a particular 
%   class of coherent structure, named \emph{Drift-Kinetic Alfv\'en
%     vortices}, arising because of the non linear coupling of Drift and
%   Kinetic Alfv\'en waves in a laboratory plasma \parencite{Vianello:2010p4670}. This type of structure has been
%   previously detected in the magnetosphere
% \item First experimental estimate of parallel current associated to
%    Edge Localized Modes filament \parencite{PhysRevLett.106.125002}
% \item First experimental measurements of 2D current distribution
% associated to plasma blobs \parencite{Furno:2011cs}
% \item Experimental evidence of transition towards helical states in
% high current Reversed Field Pinch
% operation \parencite{Lorenzini:2009p4248} and its consequence on edge
% ambipolarity \parencite{Spizzo:2014jn}
% \item Experimental investigation on the role of \emph{blobs} in the
%   formation of the so-called \emph{shoulder} in density gradient in
%   high density regime \parencite{Carralero:prl2015, Vianello:2017ku}
% \end{enumerate}
I've always tried to conjugate a strong experimental
insight on the data collection and a rigorous theoretical approach in
the data analysis and interpretation, using theories and numerical
tools as a framework to understand real plasma signals. This
approach helped me to build a bridge between theories and experiments,
a necessary effort in order to understand complex plasma
dynamics. I've always advocated the strong relationship existing
between the pedestal and edge region and the dynamics of the SOL. I
strongly believe that a real comprehension of the former can't neglect
the influence of the latter. Given that I believe
I can bring a valuable contribution to the ITPA Edge and Pedestal
group activity. 

I've authored a total number of \textbf{133 Articles} in peer reviewed journal,
\textbf{92 Conference proceedings} and personally presented \textbf{18 oral
contributions}. The complete list of publications is available on
request.  \\
h-index factor: \textcolor{red}{27} according to ISI Web of Knowledge
(last update \today). 

\begin{multicols}{2}
\setlength\bibitemsep{0pt}
\printbibliography[type=article,  title = {Cited publications}, heading=subbibliography, prefixnumbers={A}, resetnumbers=true]
\end{multicols}

