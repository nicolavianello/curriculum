\documentclass[12pt,stdletter,a4paper,dateno,sigleft]{newlfm}
% \usepackage{kpfonts}
\usepackage{url}
%\usepackage{charter}
\usepackage{fontspec,lipsum}
% ----------
% font used
\usepackage{fontspec,lipsum, xltxtra, xunicode}
\defaultfontfeatures{Mapping=tex-text}
\newfontfamily\bodyfont[]{Helvetica Neue}
\newfontfamily\thinfont[]{Helvetica Neue UltraLight}
\newfontfamily\headingfont[]{Helvetica Neue Condensed Bold}
\setmainfont[Mapping=tex-text, Color=textcolor]{Helvetica Neue Light} 

%\widowpenalty=1000
%\clubpenalty=1000

%for the logo
% \newsavebox{\RFXlogo}
% \sbox{\RFXlogo}{%
% 	\parbox[b]{1.75in}{%
% 		\vspace{0.5in}%
% 		\includegraphics[scale=.2,ext=.pdf]
% 		{RFX-logo}%
% 	}%
% }%
% \makeletterhead{Uiuc}{\Lheader{\usebox{\RFXlogo}}}

% for the signature
\newsavebox{\Sigx} 
\sbox{\Sigx}{%
          \includegraphics[height=7\baselineskip]{signature.pdf}}
\signature{\usebox{\Sigx}}
\makesignature{NV}{\newsavebox{\Signature}} 

\newlfmP{headermarginskip=10pt}
\newlfmP{sigsize=50pt}
\newlfmP{dateskipafter=20pt}
%\newlfmP{paperheight=700pt}
\newlfmP{addrfromphone}
\newlfmP{addrfromemail}
\newlfmP{MinFoot=13pt}
%\newlfmP{MinHead=50pt}
\PhrPhone{Phone}
\PhrEmail{Email}
%\encllist{Reference Letter from Prof. Francesco Gnesotto}
% \lthUiuc
\sigNV

\namefrom{Nicola Vianello}
\addrfrom{%
    \today\\[10pt]
    Swiss Plasma Center \\
    Ecole Polytechnique F{\'e}d{\'e}rale de Lausanne\\
    CH-1015 Lausanne\\
    Switzerland
}
\phonefrom{+41 21 69 34308 }
\emailfrom{nicola.vianello@epfl.ch}

%\addrto{%
%\textbf{To:} Mr. Hans Jahreiss, Fusion for Energy, European Domestic Agency \\
%\textbf{Job Title:} Plasma Physics,  Scientific Coordinator \\
%\textbf{Ref:}  Senior Scientific Officer POP-009 }

\greetto{To Whom It May Concern,}
\closeline{Sincerely,}
\begin{document}
\begin{newlfm}
my name is Nicola Vianello, I am a 40 years old Phd Physic Scientist,
currently working at the Swiss Plasma Center on the TCV experiment. I would like to apply for
the faculty position in Plasma Physics at tenured assistant professor
level. 

I've been involved in Fusion plasma Science since my M.Sci. Thesis in
Physics in 1999. My primary research interests is transport phenomena in fusion
oriented plasmas with strong emphasis on non-linear dynamics. I have
addressed the problem both experimentally, through the collection, 
analysis, interpretation and modeling of experimental data, and
numerically, through the use of massive parallel fluid codes. 

I've a strong attitude in
data analysis and evaluation with a particular emphasis on the
comparison with theories and codes which provide the suitable framework for the
correct interpretation of real data. I've been involved in
the interpretation of a variety of phenomena, ranging from electrostatic
turbulence particle transport, sheared flow turbulence generation,
turbulent non-linearly generated structures, interplay between 3D
magnetic perturbation and plasma transport phenomena. Each of these
topics has required the development of a solid theoretical
background, and the adaptation of theories to the studied framework. I strongly believe that this
attitude, which combines experimental expertise  with clear and solid
theoretical background provide an additional important element for the
establishment of a comprehensive integrated comprehension of
tokamak and more generally plasma physics. My future research plans
are aimed to extend the studies of transport and fluctuations also to
superthermal particles focusing in particular to the interplay between
turbulence and energetic particles. To address this topic an
aggressive experimental program has to be set up with the installation
of suitable diagnostic for fast-ion studies, but need to be
complemented by modeling and interpretative approach to disentangle
both the role of turbulence on fast particle population and the action
of energetic particle in generating and modulating turbulent
fluctuations.  

I've been actively involved in the coordinations of experimental
activities in various international experiment (RFX-mod, JET, TCV and
Asdex-Upgrade), both as task force leader and as scientific
coordinator. I believe these experiences
provide me a good management and coordination capabilities also in an
international environment. My international experience is indeed very good, with vital and active
collaborations with different european and international laboratories. 

I think that during my research carrier I have proved strong autonomy
accompanied by good capability to work in small and large groups. 

All these qualities and competences fits well with the
requirements for the position. 

\end{newlfm}

\end{document}
