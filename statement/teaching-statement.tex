\documentclass[12pt,a4paper]{article}
\usepackage[english]{babel}
\usepackage{geometry}
\geometry{a4paper,margin=2.5cm}
\usepackage{amsmath}									% Math
% ----------
% font used
\usepackage{fontspec,lipsum, xltxtra, xunicode}
\defaultfontfeatures{Mapping=tex-text}
\newfontfamily\bodyfont[]{Helvetica Neue}
\newfontfamily\thinfont[]{Helvetica Neue UltraLight}
\newfontfamily\headingfont[]{Helvetica Neue Condensed Bold}
\setmainfont[Mapping=tex-text, Color=textcolor]{Helvetica Neue Light} % same font used in
% ----------
% color code
\usepackage[usenames,dvipsnames,svgnames,table]{xcolor}
\definecolor{lg}{gray}{0.85} % used for hrules
\definecolor{darkBlue}{RGB}{39, 64, 139} % used in headings
\definecolor{bl}{rgb}{0.0,0.2,0.6} % alternative blue
% ----------
%   Biblio
\usepackage[normalem]{ulem}
\usepackage[sorting=ynt,style=numeric,defernumbers=true,maxnames=300,firstinits=true,uniquename=init,backend=biber,arxiv=abs, isbn=true]{biblatex}
\addbibresource{../utils/biblio.bib}
\addbibresource{../utils/others.bib}
\renewcommand{\bibfont}{\normalfont\footnotesize}
\DeclareFieldFormat[article]{pages}{#1}
\DeclareFieldFormat[inproceedings]{pages}{\lowercase{pp.}#1}
\DeclareFieldFormat[article]{volume}{\textbf{#1}}
\DeclareNameAlias{sortname}{first-last}
\DeclareFieldFormat{date}{(#1)}
\renewcommand*{\newunitpunct}{\addcomma\space}
% Don't use "In:" in bibliography. Omit urls from journal articles.
% macro for article
\DeclareBibliographyDriver{article}{%
  \usebibmacro{bibindex}%
  \usebibmacro{begentry}%
  \usebibmacro{author/editor}%
  \setunit{\labelnamepunct}\nopunct
{ \addfontfeature{Color=lightgray}\itshape%
  \usebibmacro{title}
}
  \newunit\newblock
  \usebibmacro{byauthor}%
  \newunit\newblock
  \usebibmacro{byeditor+others}%
  \newunit\newblock
  \usebibmacro{journal+issuetitle}%
  \setunit{\bibpagespunct}%
  \printfield{pages}
  \newunit\newblock
  \usebibmacro{date}
  \newunit\newblock
  \usebibmacro{pageref}%
  \usebibmacro{finentry}}
% Remove dot between volume and number in journal articles.
\renewbibmacro*{journal+issuetitle}{%
  \usebibmacro{journal}%
  \setunit*{\addspace}%
  \iffieldundef{series}
    {}
    {\newunit
     \printfield{series}%
     \setunit{\addspace}}%
  \printfield{volume}%
  \setunit{\addcomma\space}%
  \printfield{eid}%
  \newunit\newblock
  \usebibmacro{issue}%
  \newunit}

% added a trick to underline my surname
\renewbibmacro*{name:first-last}[4]{%
  \usebibmacro{name:delim}{#2#3#1}%
  \usebibmacro{name:hook}{#2#3#1}%
  \ifthenelse{\equal{#1}{Vianello}}% matches last name against YourLastName
    {
      \uline{% wrapped with \uline
      \ifblank{#2}{}{\mkbibnamefirst{#2}\isdot\bibnamedelimd}%
      \ifblank{#3}{}{%
        \mkbibnameprefix{#3}\isdot%
        \ifpunctmark{'}%
          {}%
          {\ifuseprefix{\bibnamedelimc}{\bibnamedelimd}}}%
      \mkbibnamelast{#1}\isdot%
      \ifblank{#4}{}{\bibnamedelimd\mkbibnameaffix{#4}\isdot}}}%
    {% original
      \ifblank{#2}{}{\mkbibnamefirst{#2}\isdot\bibnamedelimd}%
      \ifblank{#3}{}{%
        \mkbibnameprefix{#3}\isdot%
        \ifpunctmark{'}%
          {}%
          {\ifuseprefix{\bibnamedelimc}{\bibnamedelimd}}}%
      \mkbibnamelast{#1}\isdot%
      \ifblank{#4}{}{\bibnamedelimd\mkbibnameaffix{#4}\isdot}}}

\makeatletter

% ----------
% Sections
\usepackage{sectsty}
\usepackage[compact]{titlesec} 
\titleformat{\section}[block]{\normalfont\Large\color{darkBlue}}{\thesection}{1em}{}[\vspace{-6pt}\textcolor{lg}{\titlerule}]
\titleformat{\subsection}[block]{\normalfont\color{darkBlue}}{\thesection}{1em}{}


% ----------
%  header
\usepackage{fancyhdr}
	\pagestyle{fancy}					% Enabling the custom headers/footers
\usepackage{lastpage}	
	% Header (empty)
	\lhead{}
	\chead{}
	\rhead{}
	% Footer (you may change this to your own needs)
	\lfoot{\footnotesize \textcolor{Gray}{\textit{Dr. Vianello Teaching Statement}}}
	\cfoot{}
	\rfoot{\footnotesize \textcolor{Gray}{\textit{page \thepage\ of \pageref{LastPage}}}}	% "Page 1 of 2"
	\renewcommand{\headrulewidth}{0.0pt}
	\renewcommand{\footrulewidth}{0.4pt}

% ----------
% Change the abstract environment
\usepackage[runin]{abstract}			% runin option for a run-in title
\setlength\absleftindent{30pt}		% left margin
\setlength\absrightindent{30pt}		% right margin
\abslabeldelim{\quad}						% 
\setlength{\abstitleskip}{-10pt}
\renewcommand{\abstractname}{}
\renewcommand{\abstracttextfont}{\color{bl} \small \slshape}	% slanted text
% ----------
%   Lettrine
\usepackage{lettrine}
% ----------
% enumitem
\usepackage{enumitem}

%%% Start of the document
\begin{document}
\begin{flushright}
\textcolor{darkBlue}{ \Huge
\lettrine[lines=3,findent=-1pt, loversize = -0.42,
lraise=0.6]{N}{icola Vianello}}\\[2pt]
\textcolor{Gray}{\resizebox{0.288\linewidth}{!}{Swiss Plasma Center EPFL}}\\
\textcolor{Gray}{\resizebox{0.288\linewidth}{!}{Batiment 13, CH-1025, Lausanne }}\\
\textcolor{Gray}{\resizebox{0.288\linewidth}{!}{+41 216934308 $\cdot$  nicola.vianello@epfl.ch }}
\end{flushright}
% now the title
% ----------
%   Title 
\begin{flushleft} 
\textcolor{darkBlue}{ \Huge
\lettrine[lines=2,findent=-1pt, loversize = -0.42,
lraise=0.6]{T}{eaching statement}} 
\end{flushleft}
I strongly believe that teaching is a foundamental part for an
academic career. As a physicist, I really enjoy explaining physics
concept and I receive a great intellectual pleasure whenever I observe
students understand them. I also strongly believe that teaching must
go beyond imparting facts which is why I encourage the development of
critical thinking analysis and synthesis skills. This can be reached
actually through an adaptive process according to the students
reaction to the course. 

A challenging task as an educator is to keep lively the attention of
the students. This can be reached through well-organized engaging
lecture which possess a core topic clearly stated since the beginning
of the class. This topic has to be motivated at various levels: 
for the course itself, for physics more broadly intended, and, crucially, for current research.
We need to make the students aware
 that even basic physics offer method which is currently in use in
 actual research and more generally to actual life. I also believe that understanding of a particular
issue can be easier by teaching also a bit of 
of the history and steps which have lead to the present understanding
of the discussed topic. \emph{Learning from the past} indeed can provide the
students with a real practical example of a successful scientific
and critical method. 

Interaction with the students is vital for a teacher and provide the
indispensable energy to keep motivated the teacher
himself. Interaction can occur at various level. 
With a
single student, or small group,  which comes with question and
clarification I try to push him/them to analyze the problem again, offering
the solution step by step, or even better posing bite-sized questions
they can answer so that at the end the logically come to the right
conclusion. Whenever you deal with large classes instead understanding
the real comprehension of the topic, apart from resolving textbook
exercise is more complex. Furthermore often students can be
intimidated from the professor and interaction become
complicated. Different methods have been proposed to overcome this
issue. Among the other I would like to try the so-called \emph{Muddy
  cards} method. It consists at the end of each lectures (or of a
series of lectures cycle) to distribute anonymous cards where students
must provide feedback on which topic they consider as the more
complicated and obscure. Once reviewed the
teacher could decide to provide answers in different ways: by posting
questions and answers (Q\& A) in the course web site (which could
serve also in the following years), by preparing handouts to be
distributed the following lecture or addressing the problem directly
in the next class meeting. An
appropriate choice of the moment where to perform this exercise (for
example after a cycle of lectures which conclude a sub-program, or
after challenging lecture) can provide immediate feedback of possible
misconceptions and assists the teacher in tailoring following lectures. 

Another method which I think I would try, in particular in big
classes is the so called \emph{Peer Instruction Method},  originally
introduced by Eric Mazur \cite{mazur2013peer}. The method engages
students through activities where they both have to apply core concepts
presented, and at the same time they need to explain those concepts to
their fellow students. The process of \emph{Peer Instruction} processes
as follow: 1) the question is asked. This question, dubbed by Mazur 
ConceptTest, is a primarily multiple-choice, small conceptual question
2) The students have then the time to think and 3) the answer are
recorded. Different method could be implemented, as color cards, or
even electronic classroom response. 4) The students discuss their
answers with neighboring class mates and 5) a new record of revised
answer is recorded. Finally the teacher explain the correct answer. In
principle the method can be applied multiple times in within a single
lecture and can be used as an active method to accompany traditional
lecture. This method differ from the common practice of asking
informal questions which typically engages only  a few highly
motivated students, and provide the teacher with immediate feedback on
the rate of comprehension in the class. Furthermore discussion within
class mates help the students to present their knowledge
in a clear a structured way in order to convince the others of the
correctness of the proper choice. 

Finally, a good method to introduce the students to the practice of
doing scientific research is to stimulate them to present
their weekly homework to their peers in the form of an oral
presentation. This help also to student to gain the capability to
explain science which is, according to me, a part of the activity
of a scientist. 

So far I have always taught in small class-room where some of these
methods are actually less needed, because you can create a personal
contact with the single student. I always tried to encourage students
to think critically,  analyze and synthesize and I'm eager to apply
some of the aforementioned methods to future classrooms.

As a mentor I had followed students for Bachelor, Master
and PhD thesis. Being a mentor is actually a big
responsibility, because you need to teach the young person not only
concepts but also independence, critical spirit, method,
strictness. 
You need also to provide the enthusiasm to overcome the
possible difficulties you can easily run into when doing research. But
having the possibility to mentor a student, seeing him to accomplish
his duty and finalizing his effort can also be very rewarding and I
believe is one of the more important task for a faculty member. 


\printbibliography[title=References]

\end{document}
