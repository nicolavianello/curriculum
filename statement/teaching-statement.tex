\documentclass[12pt,a4paper]{article}
\usepackage[english]{babel}
\usepackage{geometry}
\geometry{a4paper,margin=2.5cm}
\usepackage{amsmath}									% Math
% ----------
% font used
\usepackage{fontspec,lipsum, xltxtra, xunicode}
\defaultfontfeatures{Mapping=tex-text}
\newfontfamily\bodyfont[]{Helvetica Neue}
\newfontfamily\thinfont[]{Helvetica Neue UltraLight}
\newfontfamily\headingfont[]{Helvetica Neue Condensed Bold}
\setmainfont[Mapping=tex-text, Color=textcolor]{Helvetica Neue Light} % same font used in
% ----------
% color code
\usepackage[usenames,dvipsnames,svgnames,table]{xcolor}
\definecolor{lg}{gray}{0.85} % used for hrules
\definecolor{darkBlue}{RGB}{39, 64, 139} % used in headings
\definecolor{bl}{rgb}{0.0,0.2,0.6} % alternative blue
% ----------
%   Biblio
\usepackage[normalem]{ulem}
\usepackage[sorting=ynt,style=numeric,defernumbers=true,maxnames=300,firstinits=true,uniquename=init,backend=biber,arxiv=abs, isbn=true]{biblatex}
\addbibresource{../utils/biblio.bib}
\addbibresource{../utils/others.bib}
\renewcommand{\bibfont}{\normalfont\footnotesize}
\DeclareFieldFormat[article]{pages}{#1}
\DeclareFieldFormat[inproceedings]{pages}{\lowercase{pp.}#1}
\DeclareFieldFormat[article]{volume}{\textbf{#1}}
\DeclareNameAlias{sortname}{first-last}
\DeclareFieldFormat{date}{(#1)}
\renewcommand*{\newunitpunct}{\addcomma\space}
% Don't use "In:" in bibliography. Omit urls from journal articles.
% macro for article
\DeclareBibliographyDriver{article}{%
  \usebibmacro{bibindex}%
  \usebibmacro{begentry}%
  \usebibmacro{author/editor}%
  \setunit{\labelnamepunct}\nopunct
{ \addfontfeature{Color=lightgray}\itshape%
  \usebibmacro{title}
}
  \newunit\newblock
  \usebibmacro{byauthor}%
  \newunit\newblock
  \usebibmacro{byeditor+others}%
  \newunit\newblock
  \usebibmacro{journal+issuetitle}%
  \setunit{\bibpagespunct}%
  \printfield{pages}
  \newunit\newblock
  \usebibmacro{date}
  \newunit\newblock
  \usebibmacro{pageref}%
  \usebibmacro{finentry}}
% Remove dot between volume and number in journal articles.
\renewbibmacro*{journal+issuetitle}{%
  \usebibmacro{journal}%
  \setunit*{\addspace}%
  \iffieldundef{series}
    {}
    {\newunit
     \printfield{series}%
     \setunit{\addspace}}%
  \printfield{volume}%
  \setunit{\addcomma\space}%
  \printfield{eid}%
  \newunit\newblock
  \usebibmacro{issue}%
  \newunit}

% added a trick to underline my surname
\renewbibmacro*{name:first-last}[4]{%
  \usebibmacro{name:delim}{#2#3#1}%
  \usebibmacro{name:hook}{#2#3#1}%
  \ifthenelse{\equal{#1}{Vianello}}% matches last name against YourLastName
    {
      \uline{% wrapped with \uline
      \ifblank{#2}{}{\mkbibnamefirst{#2}\isdot\bibnamedelimd}%
      \ifblank{#3}{}{%
        \mkbibnameprefix{#3}\isdot%
        \ifpunctmark{'}%
          {}%
          {\ifuseprefix{\bibnamedelimc}{\bibnamedelimd}}}%
      \mkbibnamelast{#1}\isdot%
      \ifblank{#4}{}{\bibnamedelimd\mkbibnameaffix{#4}\isdot}}}%
    {% original
      \ifblank{#2}{}{\mkbibnamefirst{#2}\isdot\bibnamedelimd}%
      \ifblank{#3}{}{%
        \mkbibnameprefix{#3}\isdot%
        \ifpunctmark{'}%
          {}%
          {\ifuseprefix{\bibnamedelimc}{\bibnamedelimd}}}%
      \mkbibnamelast{#1}\isdot%
      \ifblank{#4}{}{\bibnamedelimd\mkbibnameaffix{#4}\isdot}}}

\makeatletter

% ----------
% Sections
\usepackage{sectsty}
\usepackage[compact]{titlesec} 
\titleformat{\section}[block]{\normalfont\Large\color{darkBlue}}{\thesection}{1em}{}[\vspace{-6pt}\textcolor{lg}{\titlerule}]
\titleformat{\subsection}[block]{\normalfont\color{darkBlue}}{\thesection}{1em}{}


% ----------
%  header
\usepackage{fancyhdr}
	\pagestyle{fancy}					% Enabling the custom headers/footers
\usepackage{lastpage}	
	% Header (empty)
	\lhead{}
	\chead{}
	\rhead{}
	% Footer (you may change this to your own needs)
	\lfoot{\footnotesize \textcolor{Gray}{\textit{Dr. Vianello Teaching Statement}}}
	\cfoot{}
	\rfoot{\footnotesize \textcolor{Gray}{\textit{page \thepage\ of \pageref{LastPage}}}}	% "Page 1 of 2"
	\renewcommand{\headrulewidth}{0.0pt}
	\renewcommand{\footrulewidth}{0.4pt}

% ----------
% Change the abstract environment
\usepackage[runin]{abstract}			% runin option for a run-in title
\setlength\absleftindent{30pt}		% left margin
\setlength\absrightindent{30pt}		% right margin
\abslabeldelim{\quad}						% 
\setlength{\abstitleskip}{-10pt}
\renewcommand{\abstractname}{}
\renewcommand{\abstracttextfont}{\color{bl} \small \slshape}	% slanted text
% ----------
%   Lettrine
\usepackage{lettrine}
% ----------
% enumitem
\usepackage{enumitem}

%%% Start of the document
\begin{document}
\begin{flushright}
\textcolor{darkBlue}{ \Huge
\lettrine[lines=3,findent=-1pt, loversize = -0.42,
lraise=0.6]{N}{icola Vianello}}\\[2pt]
\textcolor{Gray}{\resizebox{0.288\linewidth}{!}{Swiss Plasma Center EPFL}}\\
\textcolor{Gray}{\resizebox{0.288\linewidth}{!}{Batiment 13, CH-1025, Lausanne }}\\
\textcolor{Gray}{\resizebox{0.288\linewidth}{!}{+41 216934308 $\cdot$  nicola.vianello@epfl.ch }}
\end{flushright}
% now the title
% ----------
%   Title 
\begin{flushleft} 
\textcolor{darkBlue}{ \Huge
\lettrine[lines=2,findent=-1pt, loversize = -0.42,
lraise=0.6]{T}{eaching statement}} 
\end{flushleft}
I strongly believe that teaching is a foundamental part for an
academic career. As a physicist, I really enjoy explaining physics
concept and I receive a great intellectual pleasure whenever I observe
students understand them. I also strongly believe that teaching must
go beyond imparting facts which is why I encourage the development of
critical thinking analysis and synthesis skills. This can be reached
actually through an adaptive process according to the students
reaction to the course. 

A challenging task as an educator is to keep lively the attention of
the students. This can be reached through well-organized engaging
lecture which possess a core topic clearly stated since the beginning
of the class. This topic has to be motivated at various levels: 
for the course itself, for physics more broadly intended, and, crucially, for current research.
We need to make the students aware
 that even basic physics offer method which is currently in use in
 actual research. I also believe that understanding of a particular
issue can be easier by teaching also a bit of 
of the history and steps which have lead to the present understanding
of the discussed topic. \emph{Learning from the past} indeed can provide the
students with a real practical example of a successful scientific
and critical method. 

Interaction with the students is vital for a teacher and provide the
indispensable energy to keep motivated the teacher
himself. Interaction can occur at various level. 
With a
single student, or small group,  which comes with question and
clarification I try to push him/them to analyze the problem again, offering
the solution step by step, or even better posing bite-sized questions
they can answer so that at the end the logically come to the right
conclusion. Whenever you deal with large class instead understanding
the real comprehension of the topic, apart from resolving textbook
exercise is more complex. Up to now I always taught to relative small
groups up to 30-40 students  
\end{document}
