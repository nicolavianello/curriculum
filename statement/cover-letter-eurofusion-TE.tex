%% start of file `template.tex'.
%% Copyright 2006-2013 Xavier Danaux (xdanaux@gmail.com).
%
% This work may be distributed and/or modified under the
% conditions of the LaTeX Project Public License version 1.3c,
% available at http://www.latex-project.org/lppl/.


\documentclass[12pt,a4paper,sans]{moderncv}        % possible options include font size ('10pt', '11pt' and '12pt'), paper size ('a4paper', 'letterpaper', 'a5paper', 'legalpaper', 'executivepaper' and 'landscape') and font family ('sans' and 'roman')

% moderncv themes
\moderncvstyle{classic}                            % style options are 'casual' (default), 'classic', 'oldstyle' and 'banking'
\moderncvcolor{green}                              % color options 'blue' (default), 'orange', 'green', 'red', 'purple', 'grey' and 'black'
%\renewcommand{\familydefault}{\sfdefault}         % to set the default font; use '\sfdefault' for the default sans serif font, '\rmdefault' for the default roman one, or any tex font name
%\nopagenumbers{}                                  % uncomment to suppress automatic page numbering for CVs longer than one page

% character encoding
\usepackage[utf8]{inputenc}                       % if you are not using xelatex ou lualatex, replace by the encoding you are using
%\usepackage{CJKutf8}                              % if you need to use CJK to typeset your resume in Chinese, Japanese or Korean

% adjust the page margins
\usepackage[scale=0.75, centering]{geometry}
%\setlength{\hintscolumnwidth}{3cm}                % if you want to change the width of the column with the dates
%\setlength{\makecvtitlenamewidth}{10cm}           % for the 'classic' style, if you want to force the width allocated to your name and avoid line breaks. be careful though, the length is normally calculated to avoid any overlap with your personal info; use this at your own typographical risks...
\usepackage{ragged2e}
% personal data
\name{Nicola}{Vianello}
%\title{Resumé title}                               % optional, remove / comment the line if not wanted
\address{Corso Stati Uniti 4}{I-35127 Padova}{Italy}% optional, remove / comment the line if not wanted; the "postcode city" and and "country" arguments can be omitted or provided empty
%\phone[mobile]{+1~(234)~567~890}                   % optional, remove / comment the line if not wanted
\phone[fixed]{+39~(049)~829~5991}                    % optional, remove / comment the line if not wanted
%\phone[fax]{+3~(456)~789~012}                      % optional, remove / comment the line if not wanted
\email{nicola.vianello@igi.cnr.it}                               % optional, remove / comment the line if not wanted
% \homepage{www.johndoe.com}                         % optional, remove / comment the line if not wanted
% \extrainfo{additional information}                 % optional, remove / comment the line if not wanted
% \photo[64pt][0.4pt]{picture}                       % optional, remove / comment the line if not wanted; '64pt' is the height the pictu
% re must be resized to, 0.4pt is the thickness of the frame around it (put it to 0pt for no frame) and 'picture' is the name of the picture file
% \quote{Some quote}                                 % optional, remove / comment the line if not wanted

% to show numerical labels in the bibliography (default is to show no labels); only useful if you make citations in your resume
%\makeatletter
%\renewcommand*{\bibliographyitemlabel}{\@biblabel{\arabic{enumiv}}}
%\makeatother
%\renewcommand*{\bibliographyitemlabel}{[\arabic{enumiv}]}% CONSIDER REPLACING THE ABOVE BY THIS

% bibliography with mutiple entries
%\usepackage{multibib}
%\newcites{book,misc}{{Books},{Others}}
%----------------------------------------------------------------------------------
%            content
%----------------------------------------------------------------------------------
\begin{document}
%-----       letter       ---------------------------------------------------------
% recipient data
\recipient{EUROfusion Program Management
  Unit}{EUROfusion\\Boltmanstra{\"s}{\"s}e 2 Garching,  Germany}
\date{June 12, 2020}
\opening{Dear Sir or Madam,}
\closing{Yours faithfully, \\
  \vspace{2pt}
  \includegraphics[height=4\baselineskip]{signature.pdf}}
%\enclosure[Attached]{curriculum vit\ae{}}          % use an optional argument to use a string other than "Enclosure", or redefine \enclname
\makelettertitle
\justify
my name is Nicola Vianello, I am a 44 years old Phd Physic Scientist,
currently working at the Consorzio RFX, Padova. I would like to apply for
the Task for Leader for Tokamak Exploitation within the Fusion Science
Department. 

I've been involved in Fusion plasma Science since my M.Sci. Thesis in
Physics in 1999. My primary research interests is transport phenomena in fusion
oriented plasmas with strong emphasis on non-linear dynamics. I have
addressed the problem both experimentally, through the collection, 
analysis, interpretation and modeling of experimental data, and
numerically, through the use of massive parallel fluid codes. 

I've a strong attitude in
data analysis and evaluation with a particular emphasis on the
comparison with theories and codes which provide the suitable framework for the
correct interpretation of real data. I've been involved in
the interpretation of a variety of phenomena, ranging from electrostatic
turbulence particle transport, sheared flow turbulence generation,
turbulent non-linearly generated structures, interplay between 3D
magnetic perturbation and plasma transport phenomena, relationship
between divertor conditions and upstream SOL and pedestal properties. Each of these
topics has required the development of a solid theoretical
background, and the adaptation of theories to the studied framework. I strongly believe that this
attitude provides an additional important element for the
establishment of a comprehensive integrated comprehension of
tokamak and more generally plasma physics. 

I've been acting as Scientific Coordinator of WP-MST1 experiments since
its establishment (AUG14-2.2-3, TCV15-2.2-3, TCV15-1.5-1, Topic 21 and
Topic 16) as well as in WP-JET1 Workpackage as Scientific Coordinator
of Task T18-02 and Experiment M18-41. I've deeply enjoyed the
possibility offered me by the EUROfusion framework to work on
different devices since I strongly believe that the level of
comprehension mandatory to extrapolate the behavior of future
machines overcome specific issues of singular experiment and can be
easily achieved  by the appropriate comparison of different
tokamaks. Furthermore EUROfusion offered me the possibility to enlarge
my scientific contact, and in general I believe that gathering
european expertise to work together in specific issues in a
goal-oriented approach helped Europe as a whole community to
consolidate the leadership on fusion science.

Acting as Scientific Coordinator pushed me to increase my  management and coordination capabilities in an
international environment. I've always act in order in an inclusive
manner, by proper spreading knowledge and information while keeping
the leadership in pushing forward the scientific experiment. I'm a
strong supporter of Open Science and I've acted in order to share
codes and results within the scientific team in the hope that this
could speed up the research activity.



I think that during my research carrier I have proved strong autonomy
accompanied by good capability to work in small and large groups. 

All these qualities and competences fits well with the
requirements for the position. 


\makeletterclosing

\end{document}


%% end of file `template.tex'.



% \documentclass[12pt,stdletter,a4paper,dateno,sigleft]{newlfm}
% % \usepackage{kpfonts}
% \usepackage{url}
% %\usepackage{charter}
% \usepackage{fontspec,lipsum}
% % ----------
% % font used
% \usepackage{fontspec,lipsum, xltxtra, xunicode}
% \defaultfontfeatures{Mapping=tex-text}
% \newfontfamily\bodyfont[]{Helvetica Neue}
% \newfontfamily\thinfont[]{Helvetica Neue UltraLight}
% \newfontfamily\headingfont[]{Helvetica Neue Condensed Bold}
% \setmainfont[Mapping=tex-text, Color=textcolor]{Helvetica Neue Light} 

% %\widowpenalty=1000
% %\clubpenalty=1000

% %for the logo
% \newsavebox{\RFXlogo}
% \sbox{\RFXlogo}{%
% 	\parbox[b]{1.75in}{%
% 		\vspace{0.5in}%
% 		\includegraphics[scale=.2,ext=.pdf]
% 		{RFX-logo}%
% 	}%
% }%
% \makeletterhead{Uiuc}{\Lheader{\usebox{\RFXlogo}}}

% % for the signature
% \newsavebox{\Sigx} 
% \sbox{\Sigx}{%
%           \includegraphics[height=7\baselineskip]{signature.pdf}}
% \signature{\usebox{\Sigx}}
% \makesignature{NV}{\newsavebox{\Signature}} 

% \newlfmP{headermarginskip=10pt}
% \newlfmP{sigsize=50pt}
% \newlfmP{dateskipafter=20pt}
% %\newlfmP{paperheight=700pt}
% \newlfmP{addrfromphone}
% \newlfmP{addrfromemail}
% \newlfmP{MinFoot=13pt}
% %\newlfmP{MinHead=50pt}
% \PhrPhone{Phone}
% \PhrEmail{Email}
% %\encllist{Reference Letter from Prof. Francesco Gnesotto}
% % \lthUiuc
% \sigNV

% \namefrom{Nicola Vianello}
% \addrfrom{%
%     \today\\[10pt]
%     Consorzio RFX \\
%     Corso Stati Uniti 4\\
%     I-35127 Padova\\
%     Italy
% }
% \phonefrom{+39 0498295991}
% \emailfrom{nicola.vianello@igi.cnr.it}

% %\addrto{%
% %\textbf{To:} Mr. Hans Jahreiss, Fusion for Energy, European Domestic Agency \\
% %\textbf{Job Title:} Plasma Physics,  Scientific Coordinator \\
% %\textbf{Ref:}  Senior Scientific Officer POP-009 }

% \greetto{To Whom It May Concern,}
% \closeline{Sincerely,}
% \begin{document}
% \begin{newlfm}
% my name is Nicola Vianello, I am a 40 years old Phd Physic Scientist,
% currently working at the Consorzio RFX, Padova. I would like to apply for
% the ITPA position in the Pedestal and Edge physics group. 

% I've been involved in Fusion plasma Science since my M.Sci. Thesis in
% Physics in 1999. My primary research interests is transport phenomena in fusion
% oriented plasmas with strong emphasis on non-linear dynamics. I have
% addressed the problem both experimentally, through the collection, 
% analysis, interpretation and modeling of experimental data, and
% numerically, through the use of massive parallel fluid codes. 

% I've a strong attitude in
% data analysis and evaluation with a particular emphasis on the
% comparison with theories and codes which provide the suitable framework for the
% correct interpretation of real data. I've been involved in
% the interpretation of a variety of phenomena, ranging from electrostatic
% turbulence particle transport, sheared flow turbulence generation,
% turbulent non-linearly generated structures, interplay between 3D
% magnetic perturbation and plasma transport phenomena, relationship
% between divertor conditions and upstream SOL and pedestal properties. Each of these
% topics has required the development of a solid theoretical
% background, and the adaptation of theories to the studied framework. I strongly believe that this
% attitude, which combines experimental expertise  with clear and solid
% theoretical background provide an additional important element for the
% establishment of a comprehensive integrated comprehension of
% tokamak and more generally plasma physics. I strongly believe that a
% physical properties of the pedestal and edge confined region are
% tightly linked to the SOL and boundary condition: moving from this
% assumption I'm trying to focus my experimental activities to bridge
% the gap between these two region and I believe that this expertise may
% be extremely relevant for the Pedestal and Edge physics group. 
% I've been actively involved in the coordinations of experimental
% activities in various international experiment (RFX-mod, JET, TCV and
% Asdex-Upgrade), both as task force leader and as scientific
% coordinator. I believe these experiences
% provide me a good management and coordination capabilities also in an
% international environment. My international experience is indeed very good, with vital and active
% collaborations with different European and international laboratories. 

% I think that during my research carrier I have proved strong autonomy
% accompanied by good capability to work in small and large groups. 

% All these qualities and competences fits well with the
% requirements for the position. 

% \end{newlfm}

% \end{document}
