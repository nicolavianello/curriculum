\documentclass[12pt,a4paper]{article}
\usepackage[english]{babel}
\usepackage{geometry}
\geometry{a4paper,margin=2.5cm}
\usepackage{amsmath}									% Math
% ----------
% font used
\usepackage{fontspec,lipsum, xltxtra, xunicode}
\defaultfontfeatures{Mapping=tex-text}
\newfontfamily\bodyfont[]{Helvetica Neue}
\newfontfamily\thinfont[]{Helvetica Neue UltraLight}
\newfontfamily\headingfont[]{Helvetica Neue Condensed Bold}
\setmainfont[Mapping=tex-text, Color=textcolor]{Helvetica Neue Light} % same font used in
% ----------
% color code
\usepackage[usenames,dvipsnames,svgnames,table]{xcolor}
\definecolor{lg}{gray}{0.85} % used for hrules
\definecolor{darkBlue}{RGB}{39, 64, 139} % used in headings
\definecolor{bl}{rgb}{0.0,0.2,0.6} % alternative blue
% ----------
%   Biblio
\usepackage[normalem]{ulem}
\usepackage[sorting=ynt,style=numeric,defernumbers=true,maxnames=300,firstinits=true,uniquename=init,backend=biber,arxiv=abs, isbn=true]{biblatex}
\addbibresource{../utils/biblio.bib}
\addbibresource{../utils/others.bib}
\renewcommand{\bibfont}{\normalfont\footnotesize}
\DeclareFieldFormat[article]{pages}{#1}
\DeclareFieldFormat[inproceedings]{pages}{\lowercase{pp.}#1}
\DeclareFieldFormat[article]{volume}{\textbf{#1}}
\DeclareNameAlias{sortname}{first-last}
\DeclareFieldFormat{date}{(#1)}
\renewcommand*{\newunitpunct}{\addcomma\space}
% Don't use "In:" in bibliography. Omit urls from journal articles.
% macro for article
\DeclareBibliographyDriver{article}{%
  \usebibmacro{bibindex}%
  \usebibmacro{begentry}%
  \usebibmacro{author/editor}%
  \setunit{\labelnamepunct}\nopunct
{ \addfontfeature{Color=lightgray}\itshape%
  \usebibmacro{title}
}
  \newunit\newblock
  \usebibmacro{byauthor}%
  \newunit\newblock
  \usebibmacro{byeditor+others}%
  \newunit\newblock
  \usebibmacro{journal+issuetitle}%
  \setunit{\bibpagespunct}%
  \printfield{pages}
  \newunit\newblock
  \usebibmacro{date}
  \newunit\newblock
  \usebibmacro{pageref}%
  \usebibmacro{finentry}}
% Remove dot between volume and number in journal articles.
\renewbibmacro*{journal+issuetitle}{%
  \usebibmacro{journal}%
  \setunit*{\addspace}%
  \iffieldundef{series}
    {}
    {\newunit
     \printfield{series}%
     \setunit{\addspace}}%
  \printfield{volume}%
  \setunit{\addcomma\space}%
  \printfield{eid}%
  \newunit\newblock
  \usebibmacro{issue}%
  \newunit}

% added a trick to underline my surname
\renewbibmacro*{name:first-last}[4]{%
  \usebibmacro{name:delim}{#2#3#1}%
  \usebibmacro{name:hook}{#2#3#1}%
  \ifthenelse{\equal{#1}{Vianello}}% matches last name against YourLastName
    {
      \uline{% wrapped with \uline
      \ifblank{#2}{}{\mkbibnamefirst{#2}\isdot\bibnamedelimd}%
      \ifblank{#3}{}{%
        \mkbibnameprefix{#3}\isdot%
        \ifpunctmark{'}%
          {}%
          {\ifuseprefix{\bibnamedelimc}{\bibnamedelimd}}}%
      \mkbibnamelast{#1}\isdot%
      \ifblank{#4}{}{\bibnamedelimd\mkbibnameaffix{#4}\isdot}}}%
    {% original
      \ifblank{#2}{}{\mkbibnamefirst{#2}\isdot\bibnamedelimd}%
      \ifblank{#3}{}{%
        \mkbibnameprefix{#3}\isdot%
        \ifpunctmark{'}%
          {}%
          {\ifuseprefix{\bibnamedelimc}{\bibnamedelimd}}}%
      \mkbibnamelast{#1}\isdot%
      \ifblank{#4}{}{\bibnamedelimd\mkbibnameaffix{#4}\isdot}}}

\makeatletter

% ----------
% Sections
\usepackage{sectsty}
\usepackage[compact]{titlesec} 
\titleformat{\section}[block]{\normalfont\Large\color{darkBlue}}{\thesection}{1em}{}[\vspace{-6pt}\textcolor{lg}{\titlerule}]
\titleformat{\subsection}[block]{\normalfont\color{darkBlue}}{\thesection}{1em}{}


% ----------
%  header
\usepackage{fancyhdr}
	\pagestyle{fancy}					% Enabling the custom headers/footers
\usepackage{lastpage}	
	% Header (empty)
	\lhead{}
	\chead{}
	\rhead{}
	% Footer (you may change this to your own needs)
	\lfoot{\footnotesize \textcolor{Gray}{\textit{Dr. Vianello Research Statement}}}
	\cfoot{}
	\rfoot{\footnotesize \textcolor{Gray}{\textit{page \thepage\ of \pageref{LastPage}}}}	% "Page 1 of 2"
	\renewcommand{\headrulewidth}{0.0pt}
	\renewcommand{\footrulewidth}{0.4pt}

% ----------
% Change the abstract environment
\usepackage[runin]{abstract}			% runin option for a run-in title
\setlength\absleftindent{30pt}		% left margin
\setlength\absrightindent{30pt}		% right margin
\abslabeldelim{\quad}						% 
\setlength{\abstitleskip}{-10pt}
\renewcommand{\abstractname}{}
\renewcommand{\abstracttextfont}{\color{bl} \small \slshape}	% slanted text
% ----------
%   Lettrine
\usepackage{lettrine}
% ----------
% enumitem
\usepackage{enumitem}

%%% Start of the document
\begin{document}
\begin{flushright}
\textcolor{darkBlue}{ \Huge
\lettrine[lines=3,findent=-1pt, loversize = -0.42,
lraise=0.6]{N}{icola Vianello}}\\[2pt]
\textcolor{Gray}{\resizebox{0.288\linewidth}{!}{Swiss Plasma Center EPFL}}\\
\textcolor{Gray}{\resizebox{0.288\linewidth}{!}{Batiment 13, CH-1025, Lausanne }}\\
\textcolor{Gray}{\resizebox{0.288\linewidth}{!}{+41 216934308 $\cdot$  nicola.vianello@epfl.ch }}
\end{flushright}
% now the title
% ----------
%   Title 
\begin{flushleft} 
\textcolor{darkBlue}{ \Huge
\lettrine[lines=2,findent=-1pt, loversize = -0.42,
lraise=0.6]{R}{esearch statement}} 
\end{flushleft}


\section*{Fluctuations in the preparation to burning plasma}
Plasmas, and in particular Fusion plasmas represent a complex system
where severa interacting degrees of freedom coexist determining a
variety of non-linear behaviors spreading over a broad range of
spatio-temporal scales
\cite{Kadomtsev:1992us,Sornette:2006dt}. It is long known that 
transport of energy, particles and momentum cannot be correctly
described in terms of simplified diffusion models, so that different
paradigms are to be invoked. Proper description of plasma
dynamics requires consequently to disentangle the role played by
fluctuations, which are found to emerge at all spatial and temporal
scales. My personal research activity has been devoted to the collection, analysis,
interpretation and modeling of experimental and numerical results
obtained in magnetized plasmas, with emphasis
on magnetically confined ones. More in details I've focused my effort
on electromagnetic fluctuations induced transport of energy, particle
and momentum, interpreting the experimental findings within the wider framework of turbulence
theory. During my research activity I have gathered a thorough 
experience on electromagnetic transport analysis, working on different
magnetic configurations, from Reversed Field Pinches (
RFX-mod operating in Padova, Extrap-T2R operating in Stockholm and
TPE-1RM20 which was in operation in Japan), stellarators (with
experimental activity on TJ-II heliac type operating in Spain) and Tokamaks
(ASDEX-Upgrade and JET), and low temperature plasmas like the Simple
Magnetized Torus experiment TORPEX at EPFL. 

I've started working on fluctuation during my M.Sci. thesis, by
studying the effect of externally modified $\mathbf{E}\times\mathbf{B}$
flow on turbulence and transport in the RFX-mod Reversed Field Pinch
\cite{Antoni:2000p3587}, observing transport quenching caused by
phase decoupling between fluctuations inhomogeneous in the
$k_{\perp}$ spectra. The necessity of a multi-scale approach lead me to deepen my
interest in turbulence and dynamical systems. It is known since
pioneering work of Kolmogorov \cite{Frisch:1996ue} that a proper
description of a non-deterministic process is based on a statistical
approach.  The Kolmogorov hypothesis is strongly based on the
assumption of self-similarity, but plasma turbulence, as well as turbulence observations from hydrodynamic to
astrophysics till econophysics \cite{Sornette:2006dt}, has been proved
to exhibit an high degree of \emph{intermittency}. I have contributed
to proving this in a variety of experiments
\cite{Antoni:2001p662,Vianello:2002p3579}, with a successful comparison 
with solar wind and atmospheric turbulence data
\cite{Carbone:2002p2809}, attaining a wide comprehension of the
mechanism and mastering advanced investigation tools used also in other research
fields. Stimulated from the observation of this strong intermittent
character I focused my effort on the characterization and
comprehension of those fluctuations (localized both in time and
$k_{\perp}$ spectra) responsible for the multifractal nature of plasma turbulence
\cite{Bohr:1998fn}. I've worked on the experimental characterization
of these eddies in a variety
of different devices and magnetic configurations
\cite{Spolaore:2009p4115,Vianello:2010p4670,Furno:2011cs,Spolaore:2015ij}. The
eddies are localized pressure perturbation with a vortex-like pattern in
the plane perpendicular to the guiding field \cite{Antoni:2006p3585},
extended along the direction of the guiding magnetic field (they are also dubbed \emph{filaments})
and with an associated parallel current
\cite{Spolaore:2009p4115,Vianello:2010p4670}. These eddies form as a
result of non-linear evolution of plasma instabilities, and among
various possible explanation,  one of the possible mechanism has been identified as a result
of the coupling between Drift Waves (DW) and Kinetic Alfv{\'e}n Waves
(KAW) \cite{Vianello:2010p4670}, in analogy with what observed in the
magnetosphere \cite{Martines:2009p4483,Sundkvist:2005is}. Extending
the analogy to the astrophysical plasmas, a striking similarity
between the enhancement of convective transport caused by these
filaments (depending on the parallel closure along the field line) and
the modification of the density gradient of the equatorial iogenic
plasma torus in the Jovian magnetosphere \cite{Frank:2002eu} was 
experimentally proved \cite{carralero:prl2015}. This testifies that investigation on
plasma fluctuations as observed in fusion plasmas have a
larger impact and exhibits deep analogies with other disciplines. 
Expertise in
current filaments studies has been extended to the studies of Edge
Localized Modes (ELM)
filaments providing for the first time an experimental direct estimation of the
current density associated to a type-I filament
\cite{PhysRevLett.106.125002,Naulin:2011im,Muller:2011kj,Spolaore:2015tk}.

Multi-scale dynamics is involved also in the process of
turbulent-generated flow. It is indeed well known that Turbulence, and
in particular Drift Wave Turbulence (DWT) can spontaneously generate patterns [zonal flows and
currents/fields, generally zonal structures (ZS)] 
on scales that are typically larger than the perpendicular (w.r.t. the
equilibrium magnetic field $B_0$) 
wavelength of the underlying fluctuations, $\lambda_{\perp}$, but
still shorter than the equilibrium nonuniformity scale length
\cite{Hasegawa:1979cp}. 
The emergence of this sheared ZF arises due to a spatial anisotropy of
fluctuation coupling, through a mechanism known in the fluid
turbulence literature as the turbulent Reynolds stress
\cite{Tynan:2009p4426}. I deeply investigated this process \cite{Vianello:2005p1976,Vianello:2006p1149},
which can be interpreted also as a
non-linear energy transfer process: it is worth noting that it is one
of the invoked mechanisms 
to explain the bifurcation to enanched confinement regimes
(H-modes) in tokamaks \cite{Schmitz:2012hwa}. 

Apart from the aforementioned multi-scale process regarding turbulence
and flow, which has been the primary subject of my research so far,
other complex dynamical processes connected with the presence of non-thermal
energetic particles (EP) are presently under consideration by the
scientific community. This is motivated
by the fact that EPs are expected to dominate the power balance in
future burning fusion plasmas \cite{Zonca:2015hs} and by the necessity
for example 
to correctly predict the behavior of natural and injected fast
particle in future devices like ITER. Studying the mutual
interaction between fast particle population and thermal plasmas, in
the framework of complex dynamics, 
represents for me the natural prosecution of my research topic and it
is at the basis of my application for EPFL faculty position. 

The interaction between fast ion and background turbulence has to be
considered from a dual perspective. 
On one side the role of background fluctuations on the
redistribution of fast ions has to be addressed. From the theoretical
point of view as a first approximation fast ions should not be
affected by small scale fluctuations due to their large gyroradius
through a process known as \emph{orbit averaging}. Recent numerical and
theoretical observations \cite{Albergante:2011bj} actually reveal
that the influence of electrostatic turbulence on fast ions depends on the ratio
$E/T_e$: in some cases it has been indicated as one of the causes of
the observed experimental observations of fast broadening of current profile after
off-axis injection of Neutral Beam
\cite{Heidbrink:2009wv,Baranov:2009p4033}, although the question is
still controversial.

On the other hand Energetic
particles can provide the free energy source for Shear Alfv{\'e}n
Waves (SAW) and drift-Alfv{\'e}n waves (DAW) excitation on the micro and
meso-scales \cite{Zonca:2015hs} with the inclusion of even shorter
wavelengths through a process of mode conversion of SAW/DAW to kinetic
Alfv{\'e}n waves. All of these instabilities could generate Zonal-like
structures at intermediate scales which interact with the original
fluctuations in a complex cross-scale self-regulated process. 
Energetic Particle induced Zonal Structures, as those for
example of the Energetic Particle Geodesic Acoustic Modes (EGAMS)
\cite{Zarzoso:2013kw,Nazikian:2008hk,Ido:2011gy} have a complicated
interaction with turbulent eddies induced by thermal ion
instabilities: this interaction could provide enhanced plasma transport
and it is thus obvious that in view of burning plasmas with a
substantial fraction of energetic particles these mechanisms should be
addressed. This topic is a hot issue in the fusion community but
there is a variety of open issues which need to be investigated,
particularly from the experimental point of view:
\begin{description}[labelindent=0pt, labelsep*=0.3em, leftmargin=!, itemsep=0.02ex]
\item[(i)] How is fast particle density modified in the presence of
  different types of fluctuations? For example when switching from electron to
  ion dominated turbulence is anomalous fast particle transport different?
\item[(ii)] Which is the role of magnetic fluctuations in determining
  the fast-particle current redistribution?
\item[(iii)] How do shaping and \emph{exotic} configuration like negative
  triangularity affect the fast-particle distribution?
\item[(iv)] Is it possible to experimentally investigate the
  cross-scale coupling between EP-driven ZF and DWT ZF?
\item[(v)] How do different populations of fast particles, generated for
  example during reconnecting processes interact with each other and
  with the background plasma?
\end{description}

The TCV tokamak in operation at the Swiss Plasma Center at EPFL has 
recently received a significant upgrade with the installation of a new Neutral Beam
Injector with energy up to 30 kEV and delivered power up to 1 MW. A
second NBI at higher energy (50 keV) is foreseen, and its procurement
is a part of the recent financial obtained from the Swiss Government. This
new system, coupled with the forthcoming upgrade of the Electron
Cyclotron Resonant Heating system will allow to reach unexplored
scenarios for this machine, with high values of normalized pressure $\beta$ and a wide range of
$T_e/T_i$ including $T_e\sim T_i\i$ and with a significant population
of fast ion. It appears clearly that the TCV machine could be the
ideal test-bed for the proposed investigation of multi-scale and
energetic particle physics, but this will require an aggressive
program on the 
experimental side. This represents the main subject of the present proposal. 
So far indeed the tokamak is equipped with a
Compact Neutral Particle Analyzer which can be used for the
investigation of fast particles. Actually to experimentally address
this topic new diagnostics should be installed. Among them I can list the
following, together with possible international collaboration to be activated or reinforced:
\begin{description}[labelindent=0pt, labelsep*=0.4em, leftmargin=!, itemsep=0.02ex]
\item[FIDA:] \emph{Fast Ion D$_{\alpha}$ diagnostic} is based on the
  same principle as Charge eXchange Recombination Spectroscopy,
  considering the emission from $n=3$ to $n=2$ Balmer series, combined
  with a proper geometrical arrangement in order to disentangle this
  emission from other sources of radiation. Presently this diagnostic
  is under consideration for a fast-track implementation of
  TCV. Possible collaborations to be activated University of
  California, Irvine, Princeton Plasma Physics Laboratory, Max Planck Instit{\"u}t f{\"u}r Plasmaphysik
\item[FILD:] \emph{Fast Ion Loss Diagnostic} can be described as a
  mass-spectrometer for Fast Ions providing discrimination of Energy
  (through gyro-orbit evaluation) and pitch angle of the collected
  fast ions. It can also be combined with photodiodes or
  photomultipliers which can give information on fast-ion fluxes at
  higher temporal resolution. Collaboration to be activated, Max
  Planck Instit{\"u}t f{\"u}r Plasmaphysik
\item[Ion Energy Analyzer probe:] This type of probe is presently
  under development in the framework of a European collaboration and
  it will combined with fast magnetic measurements giving the
  possibility for local investigation of the relation between magnetic
  fluctuations and fast ion fluxes
\item[DLP:] \emph{directional langmuir probe} can be considered as an
  extension of the Mach probe for flow measurement and can be used in
  the 
  presence of a population of fast ion induced by tangential
  beam. The direct experience I gained in the design of probes for different
  machines, could help me in a rapid development of this diagnostic. 
\item[Neutron Camera:] Collimated Neutron Flux Camera for the
  measurement of the 2.45 MeV neutron emission from the D-D fusion
  reaction can be installed. It can provide spatial and time
  resolved volume integrated neutron emissivity in the presence of NBI
  heated plasmas \cite{Cecconello:2014hw}. Possible collaboration to be activated
  University of Uppsala
\end{description}
To complement fast ion investigation a diagnostic for the
determination of the current profile is mandatory, and in this
framework the implementation of a Motional Stark Effect is suggested.
The development, installation and exploitation of each of these 
diagnostic systems can be performed within the cycle of a PhD thesis, offering the
student the possibility to gain deep insight on the experiment, and
the possibility to contribute with the data obtained in a fascinating
and cutting edge research. All the information gained by these
diagnostics can be coupled with the already existing fluctuations
diagnostic, like the correlation ECE, Doppler Reflectometry and Tangential
PCI which would give the fundamental information on the typical
spatio-temporal scale of underlying turbulence. Such an ambitious
experimental program must be tightly linked to the theory department
of plasma physics of SPC. Indeed an effort is already in progress (see
for example \cite{Albergante:2011bj,Pfefferle:2014bk}) for
the investigation of the fast ion dynamics but the proposed
experimental program could provide the indispensable results to test
and validate theoretical models. On this,  development of synthetic
diagnostics for proper interpretation of experimental data is foreseen
and envisaged as well as the application of a transport paradigm different
from the usual diffusion/convective one (non-diffusive, Levy Statistics
etc. see for example \cite{Perrone:2013hp,Greco:2003fx}). This could strengthen the collaboration between
experiment and theory department. 

As a final remark, it must be noted that fast ion studies on basic
plasma devices is already ongoing in the basic plasma
department. Apart from that a program for the establishment of an
astrophysical plasma experiment is foreseen in the following year for
the SPC. 
Within this program the study of energetic ions woulde represent an 
important ramification considering that ion acceleration is almost ubiquitous in
astrophysical phenomena,  from reconnection in the planet
magnetosphere \cite{space-science-review} to shock waves driven
outward by Coronal Mass Ejections (CMEs) or from solar flares \cite{Reames:1999bu} 
Thus
technology and expertise developed for the Tokamak experiment could be
transferred in the view of a cross-fertilization between different
branches of the plasma physics.

Finally addressing the role of fast ions and its relation with
turbulence media represents a frontier research field in the
plasma science. It requires a deep knowledge of both plasma
physics and non-linear dynamics, as multi-scale processes are involved
and paradigms beyond simple diffusive processes must be considered. I'm
convinced that my previous research on non-linear dynamics and
fluctuations gives me the proper background to tackle this
subject even despite the novelty of the research. I'm convinced that
the proposed plan will put the Swiss Plasma Center at the forefront in
the plasma physics research, and I'm excited at the idea of possibly
contributing to this.
\clearpage
\printbibliography[title=Personal publications cited,notkeyword=others, prefixnumbers={A}, resetnumbers=true]
\printbibliography[title=Other Sources, keyword=others, prefixnumbers={B}, resetnumbers=true]
\end{document}
