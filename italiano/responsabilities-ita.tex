\begin{cvblock}{Ruoli e responsabilit{\`a}}
    \cvitem{2007--2015}{Responsabile della diagnostica \emph{manipolatori remoti e 
      sonde di bordo} per l'esperimento RFX-mod}
\\[-10pt]
\cvitem{2015-2016}{ Responsabile per il sistema tomografico a raggi X
  soffici per l'esperimento TCV operante a Losanna. Vice responsabile
  per il sistema di riscaldamento addizionale basato sull'iniezione di
  fasci di particelle 
  neutre energetiche}

\\[-10pt]

\cvitem{2009}{Task force leader per l'esperimento RFX-mod della task force
  \emph{Particle, Momentum and energy transport}. La task aveva
  l'incarico di implementare il programma scientifico e coordinare
  l'attivit{\`a} di analisi atte alla comprensione dei meccanismi
  fisici alla base del trasporto di particelle energia e momento.}

\\[-10pt]

\cvitem{2010}{Task force leader per l'esperimento RFX-mod della task force
\emph{Physics integration for high performance RFP}. La task aveva
  l'incarico di implementare il programma scientifico e coordinare
  l'attivit{\`a} di analisi atte alla comprensione dei meccanismi
  fisici alla base della transizione ai regimi di confinamento
  migliorati osservati ad alta corrente.}

\\[-10pt]

\cvitem{2011}{Coordinatore del working group europeo EFDA  \emph{3D field effects in
  edge and SOL and diagnostic development} implementato nell'EFDA Transport
Topical Group. Questo gruppo di lavoro aveva il compito di coordinare
gli sforzi di numerosi laboratori europei dedicati ai seguenti
argomenti di ricerca:
\begin{enumerate}[noitemsep,leftmargin=*,topsep=0pt,partopsep=0pt]
\item Analisi e comprensione degli effetti di campi magnetici
  non-assialsimmetrici sulle strutture turbolente
\item Analisi e comprensione delle modifiche ai meccanismi di
  transporto nella regione esterna indotti da campi magnetici 3D non assialsimmetrici
\item Caratterizzazione della turbolenza nella regione esterna in
  sistemi 3D, inclusi gli effetti su ioni ed elettroni sovratermici
\item Sviluppo di codici per lo studio della turbolenza e del
  trasporto che includano effetti di campi magnetici non assial-simmetrici.
\item Confronto fra le diverse configurazioni magnetiche, tokamak,
  stellarators e reversed field pinches
\end{enumerate}
I coordinatori promuovo piattaforme di discussione fra le diverse
associazioni europee e indicano possibili linee di ricerca da perseguire.}

\\[-2pt]

\cvitem{2012}{Membro del Program committee del workshop \emph{17th Joint EU-US Transport Task Force
Meeting in combination with the 4th EFDA Transport Topical Group
meeting}, 3-6 September 2012, Padova, Italy}

\\[-10pt]

\cvitem{2013}{Coordinatore scientifico dell'esperimento \emph{B13-19
    Investigation of M-Mode} presso l'esperimento europeo JET. Il
  ruolo implicava il coordinamento dell'attivit{\`a} preparatoria per
  la campagna sperimentale, dell'attivit{\`a} di analisi e della
  divulgazione tramite pubblicazione scientifica e presentazione a
  conferenze dei risultati raggiunti.}

\\[-10pt]

\cvitem{2014}{Coordinatore scientifico dell'esperimento \emph{AUG14-2.2-3, SOL
filamentary transport at high density}, nell'ambito del Work Programme
Medium-Sized Tokamaks (MST1) implementato dal consorzio europeo
EUROfusion. L'attivit{\`a} ha comportato la stesura del programma 
sperimentale, il coordinamento dell'attivit{\`a} svolta dal team
scientifico che coinvolgeva ricercatori da diversi laboratori europei
nonch{\`e} l'attivit{\`a} di divulgazione e monitoring dei risultati scientifici.}

\\[-10pt]

\cvitem{2015-2016}{Coordinatore scientifico dell'esperimento 
  \emph{TCV15-2.2-3: Filamentary Transport in the SOL} nell'ambito del Work Programme
Medium-Sized Tokamaks (MST1) implementato dal consorzio europeo
EUROfusion. L'attivit{\`a} ha comportato la stesura del programma 
sperimentale, il coordinamento dell'attivit{\`a} svolta dal team
scientifico che coinvolgeva ricercatori da diversi laboratori europei
nonch{\`e} l'attivit{\`a} di divulgazione e monitoring dei risultati scientifici.}

\\[-10pt]

\cvitem{2015-2016}{Coordinatore scientifico dell'esperimento 
  \emph{TCV15-1.5-1, Mitigation of high Z impurity accumulation
    through combined central ECRH and tailoring of MHD activity in
    high performance H-modes}
nell'ambito del Work Programme
Medium-Sized Tokamaks (MST1) implementato dal consorzio europeo
EUROfusion. L'attivit{\`a} ha comportato la stesura del programma 
sperimentale, il coordinamento dell'attivit{\`a} svolta dal team
scientifico che coinvolgeva ricercatori da diversi laboratori europei
nonch{\`e} l'attivit{\`a} di divulgazione e monitoring dei risultati scientifici.}

\\[-10pt]

\cvitem{2017-2018}{Coordinatore scientifico del Topic 21
  \emph{Filamentary transport in high-power H-mode conditions and in
    no/small-ELM regimes to predict heat and particle loads on PFCs
    for future devices }
nell'ambito del Work Programme
Medium-Sized Tokamaks (MST1) implementato dal consorzio europeo
EUROfusion. L'attivit{\`a} ha comportato la stesura del programma 
sperimentale, il coordinamento dell'attivit{\`a} svolta dal team
scientifico che coinvolgeva ricercatori da diversi laboratori europei
nonch{\`e} l'attivit{\`a} di divulgazione e monitoring dei risultati scientifici.}

\\[-10pt]

\cvitem{2018-2019}{Coordinatore scientifico del task T18-02 
  \emph{Scrape-off layer and SOL- pedestal interaction }
  nell'ambito del Work Programme
JET1  implementato dal consorzio europeo
EUROfusion. L'attivit{\`a} ha comportato  il coordinamento dell'attivit{\`a} svolta dal team
scientifico che coinvolgeva ricercatori da diversi laboratori europei
nonch{\`e} l'attivit{\`a} di divulgazione e monitoring dei risultati
scientifici. L'ammontare del tempo uomo coordinato risulta pari a 374
ppd (person per day) nel 2018 e 503 ppdf nel 2019 }

\\[-10pt]

\cvitem{2019-2020}{Coordinatore scientifico dell'esperiment M18-41 
  \emph{Divertor geometry effect on detachment and SOL }
  nell'ambito del Work Programme
JET1  implementato dal consorzio europeo
EUROfusion. L'attivit{\`a} ha comportato la stesura del programma 
sperimentale, il coordinamento dell'attivit{\`a} svolta dal team
scientifico che coinvolgeva ricercatori da diversi laboratori europei
nonch{\`e} l'attivit{\`a} di divulgazione e monitoring dei risultati scientifici.
 L'ammontare del tempo uomo coordinato risulta pari a 480 ppd
nel 2019 mentre il dato non {\'e} ancora disponibile per il 2020. }

\\[-10pt]

\cvitem{2019-2020}{Coordinatore scientifico del Topic 16 
  \emph{Effect of filamentary transport on heat and particle loads }
  nell'ambito del Work Programme
MST1 Medium Size Tokamak  implementato dal consorzio europeo
EUROfusion. L'attivit{\`a} ha comportato la stesura del programma 
sperimentale, il coordinamento dell'attivit{\`a} svolta dal team
scientifico che coinvolgeva ricercatori da diversi laboratori europei
nonch{\`e} l'attivit{\`a} di divulgazione e monitoring dei risultati scientifici.
Importo finanziato dal Consorzio EUROfusion per l'esperimento
coordinato risulta pari a circa 800kEuro, stima basata sul tempo
macchina occupato dal progetto coordinato sulla base del costo medio
per impulso nel tokamak TCV operante a Losanna e del tokamak
ASDEX-Upgrade operante in Germania.
Include inoltre il costo del tempo uomo (ppy)
del team scientifico associato all’esperimento
che il candidato ha coordinato. }

\\[-10pt]

\cvitem{2020-}{Rappresentante Europeo nominato  dell'\emph{International Tokamak
    Physics Agreement (ITPA)}  per il gruppo \emph{Pedestal \& Edge Physics Topical
  Group}. Responsabilit\`a del gruppo \'e quello di coordinare
\emph{joint activity research } sui principali esperimenti di fusione
termonucleare nel mondo.}

\\[-10pt]

\cvitem{2020-}{Responsabile del task  D34 \emph{Far Scrape Off Layer
    transport}
  nell'ambito del gruppo ITPA Div-SOL}

\\[-10pt]

\cvitem{2020-2024}{Deputy Task Force Leader del Work Package Tokamak
  Exploitation (WPTE) implementato dal consorzio europeo EUROfusion.  Il
  workpackage WPTE \`e stato istituito al fine di coordinare
  l'attivit\`a di ricerca europea sulla fusione in supporto al
  progetto ITER e alla finalizzazione del design dell'esperimento
  DEMO. L'attivit\'a sperimentale sar\`a svolta su 4 diversi
  esperimenti europei, ASDEX-Upgrade (Germania),  TCV (Svizzera),
  MAST-U (UK),  WEST (Francia). Compito della task force
  \`e implementare e coordinare il programma sperimentale,  monitorare
il raggiungimento degli obbiettivi scientifici prefissati e stabilire
il programma plurieannale per un efficiente utilizzo degli esperimenti
di fusione termonucleare in Europa. Per il solo anno 2021 l'ammontare
del tempo uomo coordinato risulta pari a 54 ppy (person per year) che
si aggiungono al budget per il tempo macchina utilizzato nei 4 esperimenti.}
\end{cvblock}


\begin{cvblock}{Partecipazione a progetti di ricerca}
\cvitem{2000}{Partecipazione presso il gruppo di ricerca in fisica dei plasmi e fusione termonucleare presso il Consorzio RFX, sotto il contratto di Associazione Euratom-ENEA per ricerche sulla Fusione termonucleare controllata, Contratto
  EUR 343-88-1-FUAI A.A. 7, ERB5000CT88-0031-007}

\\[-10pt]

\cvitem{2001}{Partecipazione al gruppo di ricerca in fisica dei plasmi e fusione termonucleare presso il Consorzio RFX, sotto il contratto di Associazione Euratom-ENEA per ricerche sulla Fusione termonucleare controllata, Contratto
  EUR 343-88-1-FUAI A.A. 8 ERB5000CT88-0031-008}

\\[-10pt]

\cvitem{2002}{Partecipazione al gruppo di ricerca in fisica dei plasmi e fusione termonucleare presso il Consorzio RFX, sotto il contratto di Associazione Euratom-ENEA per ricerche sulla Fusione termonucleare controllata, Contratto EUR 343-88-I FUAI A.A. 9
  FU05-CT-2000-00024}

\\[-10pt]

\cvitem{2003}{Partecipazione al gruppo di ricerca in fisica dei plasmi e fusione termonucleare presso il Consorzio RFX, sotto il contratto di Associazione Euratom-ENEA per ricerche sulla Fusione termonucleare controllata, EUR 343-88-I FUAI A.A. 10
  FU05-CT-2002-00044}

\\[-10pt]

\cvitem{2004-2005}{Partecipazione al gruppo di ricerca in fisica dei plasmi e fusione termonucleare presso il Consorzio RFX, sotto il contratto di Associazione Euratom-ENEA per ricerche sulla Fusione termonucleare controllata, EUR 343-88-I FUAI A.A. 11
  FU06-CT-2003-00074}

\\[-10pt]

\cvitem{2006}{Partecipazione al gruppo di ricerca in fisica dei plasmi e fusione termonucleare presso il Consorzio RFX, sotto il contratto di Associazione Euratom-ENEA per ricerche sulla Fusione termonucleare controllata, EUR 343-88-I FUAI A.A. 12
  FU06-CT-2004-00069}

\\[-10pt]

\cvitem{2007}{Partecipazione al gruppo di ricerca in fisica dei plasmi e fusione termonucleare presso il Consorzio RFX, sotto il contratto di Associazione Euratom-ENEA per ricerche sulla Fusione termonucleare controllata, EUR 343-88-I FUAI A.A. 13
  FU06-CT-2006-00422}

\\[-10pt]

\cvitem{2008-2013}{Partecipazione al gruppo di ricerca in fisica dei
  plasmi e fusione termonucleare presso il Consorzio RFX, sotto il
  contratto di Associazione Euratom-ENEA per ricerche sulla Fusione
  termonucleare controllata, EUR FU07-CT-2007-00053}

\\[-10pt]

\cvitem{2009}{Partecipazione alla task EFDA WP09-TGS-02b \emph{Physics
    of Rotation in plasmas} sotto task WP09- TGS-02b-01/ENEA-RFX/BS}

\\[-10pt]

\cvitem{2010}{Partecipazione alla task EFDA WP10-TRA-05 \emph{Statistical properties of edge turbulent transport}, sotto task WP10-TRA-05-01-xx-01/ENEA-RFX/BS, WP10-TRA-05-01-xx-02/ENEA-RFX/BS, WP10-TRA-05-01-xx-03/ENEA-RFX/BS}

\cvitem{2011}{Partecipazione alla task EFDA WP11-TRA-05
  \emph{Statistical properties of edge turbulent transport: role of
    3-D physics}, sotto task WP11-TRA-05-01-01/ENEA-RFX,
  WP11-TRA-05-01-02/ENEA-RFX/, WP11-TRA-05-01-03/ENEA-RFX}

\cvitem{2012}{Partecipazione alla task EFDA WP12-IPH-A06
  \emph{Pedestal Instabilities (ELMs) Mitigation and Heat Loads} sotto
  task WP12-IPH-A06-1-1-05/PS-01/ENEA-RFX e WP12-IPH-A06-2-01}

\cvitem{2012}{Partecipazione alla task EFDA WP12-IPH-A08 \emph{Physics of the Pedestal and H-Mode} sotto task WP12-IPH-A08-2-08}
\end{cvblock}
