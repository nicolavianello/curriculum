% \iftoggle{nicolacv}{
% \section{Pedagogical activity}
% \subsection*{Teaching}
% \begin{entrylist}
% \entry
% {2008-2009}
% {Assistant for the course \emph{Fluid and Plasma Physics}}
% {University of Padova} {tenured Prof. T. Bolzonella. Tot. 4 h.  Seminar on MHD and Fluid turbulence. 
% A summary is presented on the theory and experimental results on turbulence, both in
% ordinary fluid and in plasmas. A description of the most
% recent results regarding turbulence and eddy’s characterization in
% thermonuclear relevant plasmas is given. Exercises on fluid
% turbulence}

% \entry
% {2010}{Assistant for the course \emph{Fluid and Plasma Physics}}
% {University of Padova} {tenured Prof. T. Bolzonella. Tot. 6 h.
%   Tangential stresses in ordinary fluid. Seminar on MHD and fluid
%   turbulence (see previous years)}

% \entry
% {2011-2012}{Assistant to the course \emph{Fundamental of Plasma
%     Physics}}{University of Padova}{Tenured Prof. G. Serianni. Tot
%   8h. Lectures on plasma waves including Plasma Oscillations,
%   Langmuir waves,  Ion acoustic waves,  Upper and Lower Hybrid waves,
% Whistler waves and MHD waves (magneto-acoustic and Alfv\'en waves)}

% \entry
% {2012-2013}{Assistant to the course \emph{Fundamental of Plasma
%     Physics}}{University of Padova}{Tenured Prof. G. Serianni. Tot
%   10h. Lectures on plasma waves including Plasma Oscillations,
%   Langmuir waves,  Ion acoustic waves,  Upper and Lower Hybrid waves,
% Whistler waves and MHD waves (magneto-acoustic and Alfv\'en waves),
% plasma stability}

% \entry
% {2013-2014}{Assistant to the course \emph{Fundamental of Plasma
%     Physics}}{University of Padova}{Tenured Prof. G. Serianni. Tot
%   10h. Lectures on plasma waves including Plasma Oscillations,
%   Langmuir waves,  Ion acoustic waves,  Upper and Lower Hybrid waves,
% Whistler waves and MHD waves (magneto-acoustic and Alfv\'en waves),
% plasma stability}

% \entry
% {2013-2014}{Joint Research Doctorate
% and European Interuniversity Doctoral Network on Fusion Science and Engineering}{University of Padova}
% {Lecturer for basic Physics course. Tot 4h. Lecture on Transport and Turbulence}

% \entry
% {2014-2015}{Assistant to the course \emph{Fundamental of Plasma
%     Physics}}{University of Padova}{Tenured Prof. G. Serianni. Tot
%   10h. Lectures on plasma waves including Plasma Oscillations,
%   Langmuir waves,  Ion acoustic waves,  Upper and Lower Hybrid waves,
% Whistler waves and MHD waves (magneto-acoustic and Alfv\'en waves),
% plasma stability}

% \end{entrylist}

% \subsection*{Supervising}
% \begin{entrylist}
% \entry
% {2007}{Supervisor for Bachelor Thesis in Physics}{University of
%   Padova}
% { \textbf{Candidate:} Alessandro Scaggion \\
% \textbf{Thesis title:} \emph{Electrostatic fluctuations characterization in
%    RFX-mod experiment in different experimental condition} \\
% \textbf{Thesis subject:} Characterization of floating potential
%  measurements as obtained from an internal array of sensors in
%  different discharge conditions highlighting dependence on
%  equilibrium and density. 
% }

% \entry
% {2009}{Supervisor for M.Sci Thesis in Physics}{University of
%   Padova}
% { \textbf{Candidate:} Alessandro Scaggion \\
% \textbf{Thesis title:} \emph{Filamentary structures in the edge
%   turbulence of fusion device} \\
% \textbf{Thesis subject:} Characterization of turbulence
% electromagnetic structures in two different devices: RFX-mod Reversed
% Field Pinch experiment, characterized by the presence of
% Drift-Alfv\'en filaments, and ASDEX-Upgrade tokamak, with emphasis on
% type I ELM's filaments.
% }

% \entry
% {2011}{Supervisor for Bachelor Thesis in Physics}{University of
%   Padova}
% { \textbf{Candidate:} Alberto Mazzi \\
% \textbf{Thesis title:} \emph{Experimental evaluation of toroidal
%    velocity distribution in the edge region of RFX-mod and its impact
%    on high density regimes} \\
% \textbf{Thesis subject:} Experimental determination of the
%  spatio-temporal distribution of the toroidal
%  velocity  in RFX-mod and its relationship with edge magnetic
%  topology. The strong link between magnetic islands and
%  plasma flow distribution is addressed.
% }

% \entry
% {2013}{Supervisor for M.Sci. Thesis in Physics}{University of
%   Padova}
% { \textbf{Candidate:} Alberto Mazzi \\
% \textbf{Thesis title:} \emph{Analysis of vorticity coherent structures
% in magnetically confined plasmas} \\
% \textbf{Thesis subject:} Experimental characterization of vorticity
% and parallel current associated to the existence of plasma structure
% at large scales (associated to MHD activity) and small scales
% (associated to fully developed MHD turbulence)
% }

% \entry
% {2015}{Supervisor for PhD Thesis in Physics}{University of
%   Padova}
% { \textbf{Candidate:} Cristina Rea \\
% \textbf{Thesis title:} \emph{} \\
% \textbf{Thesis subject:} Experimental characterization of the effect
% of a 3D magnetic field on electrostatic fluctuations responsible for
% turbulence induced transport. Comparative analysis between different
% magnetic configuration)
% }

% \end{entrylist}
% }
% {
\begin{cvblock}{Insegnamento}

\cvitem{2008--2009}{Lezioni tenute per il corso \emph{Fisica dei Fluidi e
    dei Plasmi}, Dipartimento di Fisica,  Universit{\`a} degli Studi
  di Padova, titolare: Dr. Tommaso Bolzonella}

\\[-4pt]

\cvitem{2010}{Lezioni tenute per il corso \emph{Fisica dei Fluidi e
    dei Plasmi}, Dipartimento di Fisica,  Universit{\`a} degli Studi
  di Padova, titolare: Dr. Tommaso Bolzonella}

\\[-4pt]

\cvitem{2011--2012}{Lezioni tenute per il corso \emph{Fondamenti di Fisica
  del Plasma}, Dipartimento di Fisica,  Universit{\`a} degli Studi
  di Padova, titolare: Dr. Gianluigi Serianni.}

\\[-4pt]

\cvitem{2012-2013}{Lezioni tenute per il corso \emph{Fondamenti di Fisica
  del Plasma}, Dipartimento di Fisica,  Universit{\`a} degli Studi
  di Padova, titolare: Dr. Gianluigi Serianni.}

\\[-4pt]

\cvitem{2013-2014}{Lezioni tenute per il corso \emph{Fondamenti di Fisica
  del Plasma}, Dipartimento di Fisica,  Universit{\`a} degli Studi
  di Padova, titolare: Dr. Gianluigi Serianni.}

\\[-4pt]

\cvitem{2013-2014}{Lezioni per il corso \emph{Basic Physics} per il Joint Research Doctorate
and European Interuniversity Doctoral Network on Fusion Science and Engineering}

\\[-4pt]

\cvitem{2014-2015}{Lezioni tenute per il corso \emph{Fondamenti di Fisica
  del Plasma}, Dipartimento di Fisica,  Universit{\`a} degli Studi
  di Padova, titolare: Dr. Gianluigi Serianni.}
\end{cvblock}

\begin{cvblock}{Supervisione}

\cvitem{2007}{Corelatore, Laurea Triennale in Fisica, Dipartimento di
  Fisica,
  Universit{\`a} of Padova, candidato: A. Scaggion}

\\[-4pt]

\cvitem{2009}{Corelatore, Laurea Specialistica in Fisica, Dipartimento di
  Fisica,
  Universit{\`a} of Padova, candidato: A. Scaggion}

\\[-4pt]

\cvitem{2011}{Corelatore, Laurea Triennale in Fisica, Dipartimento di
  Fisica,
  Universit{\`a} of Padova, candidato: A. Mazzi}

\\[-4pt]

\cvitem{2013}{Corelatore, Laurea Specialistica in Fisica, Dipartimento di
  Fisica,
  Universit{\`a} of Padova, candidato: A. Mazzi}

\\[-4pt]

\cvitem{2015}{Corelatore, Tesi di Dottorato in Fisica, Dipartimento di
  Fisica,
  Universit{\`a} of Padova, candidato: C. Rea}

\\[-4pt]

\cvitem{2009}{Corelatore,  M.Sci. Thesis, Ecole Polytechnique
  Federale de Lausanne, candidato: M. Pedro}
\end{cvblock}

\begin{cvblock}{Comitato di valutazione PhD}

\cvitem{2014}{PhD Commitee presso il Department of Physics, Technical
  University of Denmark. Candidato: N. Yan}

\\[-4pt]
\cvitem{2015}{PhD Commitee presso la Facult{\'e} de Sciences de Base,
  Ecole Polytechnique Federale de Lausanne. Candidato: F. Avino}

\\[-4pt]
\cvitem{2016}{PhD Commitee presso la Facult{\'e} de Sciences de Base,
  Ecole Polytechnique Federale de Lausanne. Candidato: F. Nespoli}

\\[-4pt]
\cvitem{2017}{PhD Commitee presso Department of Physics,
  University of York. Candidato: A. Wynn}

\\[-4pt]
\cvitem{2018}{PhD Commitee presso la Facult{\'e} de Sciences de Base,
  Ecole Polytechnique Federale de Lausanne. Candidato: P. Paruta}

\\[-4pt]
\cvitem{2014}{PhD Commitee presso il Department of Physics, Technical
  University of Denmark. Candidato: J. M. B. Olsen}

\end{cvblock}
