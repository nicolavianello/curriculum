\begin{cvblock}{Concorsi ed Abilitazioni}
\cvitem{May 2009}{Vincitore del Consorso pubblico (Ref.364/12) presso il Consiglio
  Nazionale delle Ricerche, per una posizione a tempo
  indeterminato, ricercatore III livello}

\\[-10pt]

\cvitem{Commissione}{Prof. A. Fasoli, Ecole Polytechnique Federale de Lausanne,
  Svizzera}
\cvitem{}{Dr. V. Antoni, Consiglio Nazionale delle Ricerche,
Istituto Gas Ionizzati,  Padova}
\cvitem{}{Dr. D. Farina,  Consiglio Nazionale delle Ricerche, Istituto di
  Fisica del Plasma, Milano}
%------------------
\\[-6pt]
\cvitem{2012}{Abilitazione Scientifica Nazionale alle funzioni di professore universitario di
  Seconda fascia nel settore concorsuale 02/B1 - Fisica Sperimentale
  della Materia. Bando D.D. 222/2012}

\\[-10pt]

\cvitem{Commissione}{Prof. Mattera Lorenzo, Universit{\'a} degli
  Studi di Genova, Italy}
\cvitem{}{Prof. Rinaldo Cubeddu, Politecnico di Milano, Italy}
\cvitem{}{Prof. Stefano Nannarone, Universit{\'a} degli Studi di
  Modena e Reggio Emilia}
\cvitem{}{Prof. Mobilio Settimio, Universit{\'a} degli Studi di Roma
  Tre}
\cvitem{}{Prof. Andrea Cavalleri, Max Planck Institute for the
  Structure and Dynamics of Matter, Hamburg}
\cvitem{}{Abilitazione valida dal 11/12/2013 al 11/12/2019}
% ------------------
\\[-6pt]
\cvitem{2018}{Abilitazione Scientifica Nazionale alle funzioni di professore universitario di
  Prima fascia, settore concorsuale 02/B1 - Fisica Sperimentale
  della Materia. Bando D.D. 1532/2016, tornata 2016, quinto quadrimestre}

\\[-10pt]

\cvitem{Commissione}{Prof. Federico Boscherini, Universit{\'a} degli
  Studi di Bologna}
\cvitem{}{Prof. Giulio Nicola Cerullo, Politecnico di Milano}
\cvitem{}{Prof.ssa Pasqualino Maria Maddalena, Universit{\'a} degli Studi di
  Napoli}
\cvitem{}{Prof. Francesco Saverio Pavone, Universit{\'a} degli Studi di Firenze}
\cvitem{}{Prof. Sandro Santucci, Universit{\'a} degli Studi
  dell'Aquila}
\cvitem{}{Abilitazione valida dal 26/07/2018 al 26/07/2027}
%------------------
\\[-6pt]
\cvitem{2018}{Abilitazione Scientifica Nazionale alle funzioni di professore universitario di
  Seconda fascia nel settore concorsuale 02/B1 - Fisica Sperimentale
  della Materia. Bando D.D. 1532/2016, tornata 2016, quinto quadrimestre}

\\[-10pt]

\cvitem{Commissione}{Prof. Federico Boscherini, Universit{\'a} degli
  Studi di Bologna}
\cvitem{}{Prof. Giulio Nicola Cerullo, Politecnico di Milano}
\cvitem{}{Prof. Pasqualino Maria Maddalena, Universit{\'a} degli Studi di
  Napoli}
\cvitem{}{Prof. Francesco Saverio Pavone, Universit{\'a} degli Studi di Firenze}
\cvitem{}{Prof. Sandro Santucci, Universit{\'a} degli Studi
  dell'Aquila}
\cvitem{}{Abilitazione valida dal 26/07/2018 al 26/07/2027}
%------------------
\\[-6pt]
\cvitem{2018}{Abilitazione Scientifica Nazionale alle funzioni di professore universitario di
  Seconda fascia nel settore concorsuale 02/B2 - Fisica Teorica
  della Materia. Bando D.D. 1532/2016, tornata 2016, quinto quadrimestre}

\\[-10pt]

\cvitem{Commissione}{Prof. Federico Boscherini, Universit{\'a} degli
  Studi di Bologna}
\cvitem{}{Prof.ssa Vincenza Cupri, Universit{\'a} degli Studi di
  Messina}
\cvitem{}{Prof. Amos Maritan, Universit{\'a} degli Studi di Padova}
\cvitem{}{Prof. Alessandro Tredicucci, Universit{\'a} degli Studi di Pisa}
\cvitem{}{Prof. Pierluigi Veltri, Universit{\'a} della Calabria}
\cvitem{}{Abilitazione valida dal 08/08/2018 al 08/08/2027}
\end{cvblock}

