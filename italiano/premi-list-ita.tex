\begin{statementblock}{Premi}
\end{statementblock}
\begin{enumerate}[label={[M\arabic*]}]
\item \textbf{2005:} Premio Individuale del Consorzio RFX per l'anno 2004 per
  documentati contributi individuali innovativi nel campo della
  modellistica del plasma, teoria ed interpretazione dei dati
  
\item \textbf{2006:} Premio del Consorzio RFX per l'anno 2005 per
  pubblicazioni rilevanti o brevetti depositati nel campo della
  modellistica del plasma, teoria ed interpretazione dati assegnato
  alla pubblicazione: \emph{Shear flows generated by plasma turbulence
    and their influence on transport} Plasma Physics and Controlled
  Fusion \textbf{47} B13 (2005)
  
\item \textbf{2009:} Premio Individuale del Consorzio RFX per l'anno 2008 per
  documentati contributi individuali innovativi nel campo della
  modellistica del plasma, teoria ed interpretazione dei dati

\item \textbf{2010:} Premio del Consorzio RFX per l'anno 2009 per
  pubblicazioni rilevanti o brevetti depositati nel campo della
  modellistica del plasma, teoria ed interpretazione dati assegnato
  alla pubblicazione: \emph{Direct Measurement of current filament
    structures in a magnetic-confinement fusion device} Physical
  Review Letters \textbf{102} 165001 (2009)
  
\item \textbf{2010:} Premio del Consorzio RFX per l'anno 2010 per
  pubblicazioni rilevanti o brevetti depositati nel campo della
  modellistica del plasma, teoria ed interpretazione dati assegnato
  alla pubblicazione: \emph{Drift-Alfv{\'e}n vortex structures in the
    edge region of a fusion relevant plasmas} Nuclear Fusion
  \textbf{50} (2010) 042002

\item \textbf{2014:} Premio 2013 per Pubblicazioni rilevanti o brevetti
  depositati nel campo della modellistica del plasma, teoria e
  interpretazione dei dati da parte del Consorzio RFX, Associazione
  Euratom-ENEA, alla pubblicazione: M. Zuin, S. Spagnolo, I. Predebon,
  F. Sattin, F. Auriemma, R. Cavazzana, A. Fassina, E. Martines,
  R. Paccagnella, M. Spolaore, and N.Vianello,
  \emph{Experimental Observation of Microtearing Modes in a Toroidal
    Fusion Plasma}, Phys. Rev. Lett. \textbf{110}, 055002 (2013)

\item \textbf{2015:} Premio del Consorzio RFX per l'anno 2015 per
  pubblicazioni rilevanti o brevetti depositati nel campo della
  modellistica del plasma, teoria ed interpretazione dati assegnato
  alla pubblicazione: a M. Spolaore, N. Vianello, I. Furno,
  D. Carralero,
  M. Agostini, J.A.Alonso, F. Avino, R. Cavazzana, G. De Masi,
  A. Fasoli,
  C. Hidalgo, E.
  Martines, B. Momo, A. Scaggion, P. Scarin, S. Spagnolo, G. Spizzo, C.
Theiler, M. Zuin, \emph{Electromagnetic turbulent structures: A ubiquitous
feature of the edge region of toroidal plasma configurations}, Physics
of Plasmas, \textbf{22} 012310

\item \textbf{2017:} Premio del Consorzio RFX per l'anno 2016 per
  pubblicazioni
  rilevanti o brevetti depositati nel campo della modellistica del
  plasma,
  teoria ed interpretazione dati assegnato alla pubblicazione:
  \emph{On the statistics features of turbulent structures in RFX-mod}
  Plasma Phys. Contr. Fusion \textbf{58} 044009 (2016)
\end{enumerate}
\begin{statementblock}{Comitati di Valutazione}
\end{statementblock}
\nopagebreak
\begin{enumerate}[label={[N\arabic*]}]
\item \textbf{01/03/2013-22/03/2013:} Responsabile di valutazione per
  il bando “Futuro in Ricerca 2013” della proposta codice RBFR13MXVQ
\item \textbf{14/03/2013-02/04/2013:} Responsabile di valutazione
    per il bando “PRIN 2012” della proposta codice 2012XAS7WZ
\item \textbf{01/05/2020-30/06/2020} Membro esperto per la valutazione
  di proposte presso Office of Fusion Energy Science,  Department of
  Energy USA
\item \textbf{20/4/2020-31/8/2020} Membro esperto valutatore per Czech
  Academy of Science  
\item \textbf{23/6/2022-31/8/2022} Membro esperto valutatore per Swedish Research Council 
\end{enumerate}
\begin{statementblock}{Comitati di Conferenze}
\end{statementblock}
\begin{enumerate}[label={[O\arabic*]}]
\item Membro del Program committee del workshop \emph{17th Joint EU-US Transport Task Force
Meeting in combination with the 4th EFDA Transport Topical Group
meeting}, 3-6 September 2012, Padova, Italy
\item Membro del Program committee della conferenza \emph{48th EPS
    Conference on Plasma Physics}, 27 Giugno-1 Luglio 2022, Maastricht 
\end{enumerate}
\begin{statementblock}{Abilitazione Scientifica Nazionale}
\end{statementblock}
\begin{enumerate}[label={[P\arabic*]}]
\item \textbf{2012: } Abilitazione Scientifica Nazionale alle funzioni di professore universitario di
  Seconda fascia nel settore concorsuale 02/B1 - Fisica Sperimentale
  della Materia. Bando D.D. 222/2012

\item \textbf{2018:} Abilitazione Scientifica Nazionale alle funzioni di professore universitario di
  Prima fascia nel settore concorsuale 02/B1 - Fisica Sperimentale
  della Materia. Bando D.D. 1532/2016, tornata 2016, quinto
  quadrimestre. Abilitazione valida fino al 26/07/2029

\item \textbf{2018:} Abilitazione Scientifica Nazionale alle funzioni di professore universitario di
  Seconda fascia nel settore concorsuale 02/B1 - Fisica Sperimentale
  della Materia. Bando D.D. 1532/2016, tornata 2016, quinto
  quadrimestre. Abilitazione valida fino al 26/07/2029

\item \textbf{2018: } Abilitazione Scientifica Nazionale alle funzioni di professore universitario di
  Seconda fascia nel settore concorsuale 02/B2 - Fisica Teorica
  della Materia. Bando D.D. 1532/2016, tornata 2016, quinto
  quadrimestre. Abilitazione valida fino al 26/07/2029

\end{enumerate}
%%% Local Variables:
%%% mode: latex
%%% TeX-master: "cvnicola-list-italiano"
%%% End:
