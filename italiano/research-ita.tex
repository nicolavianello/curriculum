\begin{cvblock}{Interessi di ricerca}
\end{cvblock}
Sono stato coinvolto nella ricerca sui plasmi di interesse
termonucleare fin dalla laurea in Fisica nel 1999. Durante questi
\FPtrunc\mydegree{\mydegree}{0}\mydegree\ anni ho cercato di ampliare 
il pi{\`u} possibile le mie competenze focalizzandole in particolare
nella raccolta, analisi, interpretazione e modellizzazione di dati
sperimentali raccolti in differenti configurazioni magnetiche
(Reversed Field Pinches, Tokamaks e Stellarators) con una enfasi
particolare sul confronto fra i dati sperimentali e le interpretazioni
teorico modellistiche. I miei principali interessi di ricerca si
possono riassumere nei seguenti argomenti:
\begin{description}[labelindent=0.5pt, labelsep*=0.4em, leftmargin=!, itemsep=0.05ex]
\item[(a) Trasporto indotto da turbolenza elettromagnetica:]
con una enfasi particolare nel fenomeno noto come \emph{trasporto
  anomalo} indotto da diversi tipi di instabilit{\`a} di plasma quali
i modi elettrostatici di tipo Drift o Interchange, oppure il trasporto
indotto da un campo magnetico stocastico o da fluttuazioni magnetiche
in genere.
\item[(b) Analisi statistica di turbolenza nei plasmi:] questo
  argomento mi ha permesso di acquisire competenze su diverse
  metodologie statistiche (quali trasformate Wavelet, Local Intermittency
  Measurements, Waiting Time distribution) e su sistemi e modelli
  dinamici come Self-Organized Criticality (SOC) e shell-models.
\item[(c) Strutture coerenti, eddies e filamenti:] l'attivit{\`a} di
  ricerca si {\`e} dedicata allo studio di strutture coerenti
  sviluppate come evoluzione non lineari di
  instabilit{\`a} di plasma. Mi sono dedicato in particolare alla
  ricerca dei meccanismi di generazione di queste strutture,
  sul contributo di questi
  eddies al trasporto anomalo di particelle, energia e momento
  includendo uno studio sulla dinamica parallela al campo magnetico
  con la determinazione delle loro caratteristiche elettromagnetiche.
\item[(d) Genearazione di momento e di Sheared flow:] mi sono dedicato
  allo studio dei 
  meccanismi di generazione di momento indotti da turbolenza
  analizzando il ruolo dei tensori di Reynolds e Maxwell, includendo
  uno studio sperimentale dei meccanismi di trasferimento di energia
  fra diverse scale spaziali e temporali.
\item[(e) Interazione fra il flow di plasma e la topologia magnetica:]
  con enfasi sugli effetti indotti da campi non assial-simmetrici
  sulle propriet{\`a} cinetiche del plasma, sul flow e sul campo
  elettrico ambipolare
\item[(f) Interazione fra un neutral Beam ed il plasma:] con enfasi
  sulle instabilit{\`a} di Alfv\'en,
sulle instabilit{\`a} indotte da particelle energetiche, e sul
trasporto di ioni energetici da parte della turbolenza del plasma
\item[(g) Plasma detachment e configurazioni di divertore
  alternativo:] nell'ambito degli studi di fusione termonucleare un
  argomento rilevante {\`e} la comprensione dei meccanismi pi{\`u}
  efficaci per lo smaltimento del carico termico sulle
  pareti materiali. A tal fine mi sono dedicato alle possibile
  configurazioni di divertore alternativo ed allo studio del
  meccanismo di detachment, che assicura carichi termici sostenibili
  per futuri reattori.
\end{description}

Fra i pi{\`u} importanti risultati ottenuti citiamo i seguenti:
\begin{enumerate}[itemsep=0.05ex, label=\textbf{\roman*}]
\item Prima prova sperimentale della non applicabilit{\`a} del
  paradigma della \emph{Self
    Organized Criticality} nello studio della turbolenza nella regione
  esterna dei plasmi \parencite{Spada:2001p3574,Antoni:2001p3221}
\item Prima evidenza sperimentale della generazione non lineare di
  flow in plasmi Reversed Field Pinches attraverso i meccanismi di
  Reynolds stress \parencite{Vianello:2005p1976,Vianello:2005p2671}
\item Prima misura sperimentale di corrente parallela associata ad
  eddies turbolenti in plasmi di interesse termonucleare \parencite{Spolaore:2009p4115} 
\item Prima evidenza sperimentale dell'esistenza di strutture coerenti
  note come \emph{Drift-Kinetic Alfv\'en
    vortices}, risultanti dall'interazione non lineare fra onde di
  tipo 
  Kinetic Alfv\'en waves ed onde di deriva in plasmi di laboratorio
  \parencite{Vianello:2010p4670}.
\item Prima stima sperimentale della corrente parallela associata a
  filamenti di tipo Edge Localized Modes in plasmi di interesse termonucleare \parencite{PhysRevLett.106.125002}
\item Prima misura sperimentale della distribuzione 2D di corrente
  associata ad un blob di plasma \parencite{Furno:2011cs}
\item Evidenza sperimentale della transizione a stati di singola
  elicit{\`a} in Reversed Field pinches
  \parencite{Lorenzini:2009p4248} e conseguenze in termini di campo
  ambipolare al bordo \parencite{Spizzo:2014jn}
\item Evidenza sperimentale del ruolo dei \emph{blobs} nel meccanismo
  che stabilisce il profilo di densit{\`a} nei tokamak \parencite{Carralero:prl2015, Vianello:2017ku}
\end{enumerate}
In tutti i miei studi ho sempre cercato di coniugare l'attivit{\`a}
sperimentale, incluso tutta il lavoro propedeutico necessario per
sviluppare un sistema diagnostico e di acquisizione dati, con un
rigoroso approccio teorico nell'interpretazione dei dati stessi usando
teorie e simulazioni numeriche come guida per la comprensione del
fenomeno. Ho cercato di instaurare una intensa e proficua attivit{\`a}
di collaborazione a livello internazionale.
Il mio attuale ruolo di Deputy Task Force Leader per il Work Package
\emph{Tokamak Exploitation} mi permette di avere un'ampia visione
della ricerca Europea nella fusione e testimonia il riconoscimento
internazionale maturato nell'ambito della mia ricerca. Il mio ambito
principale di ricerca, relativo al problema del \emph{Power Exhaust}
chiaramente mi colloca come importante protagonista in quello che
rappresenta la missione principale del  nuovo
progetto bandiera italiano DTT (Divertor Test Tokamak).\\
