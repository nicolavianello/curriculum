\begin{statementblock}{Partecipazione a progetto scientifico}
\end{statementblock}
\begin{enumerate}[label={[G\arabic*]}]
\item \textbf{2012- in corso:} Partecipazione al progetto RFX e successivamente
  RFX-mod nell'ambito dell'associazione EURATOM-ENEA sul programma
  fusion (2000-2014) e successivamente nell'ambito del controatto
  EUROfusion Consortium-ENEA. L'importo totale dei finanziamenti
  europei e nazionali per l'esperimento RFX e le sue successive
  modifiche \'e di circa 255 M\euro a cui si aggiungono 18 M\euro
  attraverso il progetto PNNR Nefertari. RFX-mod \'e inserito fra i
  progetti ad Alta Priorit\'a del Progamma Nazionale delle
  infrastrutture e di Ricerca 2021-2027. Il candidato ha partecipato
  dal 2002 ad oggi in modo attivo alle attività dell’infrastruttura,
  nella interpretazione dei dati raccolti, come responsabile di
  diagnostiche, responsabile di sessioni sperimentale e come Task
  Force Leader nel periodo 2009-2010
  %
\item \textbf{23/11/2022 - in corso}  Partecipante al progetto Europeo
  \emph{Multi-scale Electrostatic Energisation of Plasmas: Comparison
    of Collective Processes in Laboratory and Space} nell'ambito della
  selezione competitiva per Joint Bilateral Agreement CNR/Royal
  Society. Il progetto ha l'obbiettivo di instaurare una
  collaborazione per lo studio e l'analisi comparativa degli effetti
  di energizzazione delle particelle a scale diverse in plasmi
  astrofisici e plasmi di laboratorio.

\item \textbf{01/01/2014-31/12/2014:} Partecipante al progetto Europeo
  \emph{Understanding, predicting and utilising non-axisymmetry in tokamak plasmas},
  contratto WP14-ER-01/CCFE-03 finanziato con 177 k\euro dal
  consorzio EUROfusion tramite bando competitivo nell'ambito
  dell'iniziativa di \emph{Enabling Research}. Il progetto utilizzava
  i pi{\'u} avanzati metodo numerici e computazionali per comprendere
  gli effetti di perturbazioni magnetiche non assial-simmetrici in
  equilibri tokamak, RFPs e Stellarator, comprendendone la
  stabilit{\'a}, la performance ed anche per guidare le scelte sulle
  capacit{\'a} di controllo non-assialsimmetrico per ITER. 

\item \textbf{01/01/2014-31/12/2014:} Partecipante al progetto Europeo
  \emph{Investigation of edge plasma electromagnetic filaments and
    associated transport: from ELMs to turbulent structures},
  contratto WP14-ER-01/ENEA-RFX-06 finanziato con 187 k\euro dal
  consorzio EUROfusion tramite bando competitivo nell'ambito
  dell'iniziativa di \emph{Enabling Research}. Il progetto era
  finalizzato allo studio delle proprietà elettromagnetiche delle
  strutture turbolente, degli ELMs e dei contributo allo scattering di
  particelle sovratermiche per effetto della turbolenza
  %
\item \textbf{01/01/2013-31/12/2013:} Partecipante al progetto Europeo
  \emph{3D effects on plasma rotation: Comparative
    studies in Tokamak and RFPs }, contratto
  WP13-IPH-A04-P1-01/ENEA-RFX/PS finanziato con 16 k\euro dal
  consorzio EUROfusion. Il candidato ha contribuito al progetto
  tramite l'analisi della velocità di plasma nella regione esterna e
  determinazione della modulazione non assialsimmetrica del campo
  elettrico ambipolare
  %
\item \textbf{08/02/2012-31/12/2012:} Partecipante al progetto Europeo
  \emph{Measurements of SOL transport by probe in H mode during ELM
    and inter-ELM intervals }, contratto
  WP12‐IPH‐A06‐1‐1‐05/PS‐01/ENEA-RFX finanziato con 0.2 FTE (full time
  equivalent) dal
  consorzio EUROfusion. Il candidato ha contribuito al progetto
  tramite l'analisi della corrente parallela associata ai blob ed ELMs
  in plasmi confinati in configurazione tokamak e reversed field pinch
%
 \item \textbf{01/01/2011-31/12/2011:} Partecipante al progetto Europeo
  \emph{Effects of magnetic perturbation and 3D field on blob and filaments }, contratto
  WP11‐TRA‐A05/BS-PS/ENEA-RFX finanziato con 0.5 FTE  dal
  consorzio EUROfusion. Il candidato ha contribuito al progetto
  tramite l'analisi della correlazione esistente fra perturbazioni
  magnetiche indotte o
  prodotte naturalmente nel plasma e caratteristiche della turbolenza

 \item \textbf{2021-In Corso}: Partecipante al progetto
   \emph{Definition of the technical requirements and interfaces for
     embedded divertor probes}  all’interno del Work Package
   WP DIV–IDTT Eurofusion per il Progetto dell’esperimento DTT.
   L’attività svolta dal 2021 prevede la definizione, la progettazione
   e l’integrazione del sistema di sonde elettrostatiche e termiche
   nella regione
   del divertore dell’esperimento DTT.

   
\end{enumerate}
%%% Local Variables:
%%% mode: latex
%%% TeX-master: "cvnicola-list-italiano"
%%% End:
