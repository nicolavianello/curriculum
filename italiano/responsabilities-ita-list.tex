\begin{statementblock}{Ruoli e Responsabilit{\'a}}
\end{statementblock}


% \begin{section*}{Ruoli e Responsabilit{\'a}}
\begin{enumerate}[label={[E\arabic*]}]

\item \textbf{2007--2015:} Responsabile della diagnostica
  \emph{manipolatori remoti e sonde di bordo} per l'esperimento
  RFX-mod
%
\item \textbf{2015-2016:} Responsabile per il sistema tomografico a
  raggi X soffici per l'esperimento TCV operante a Losanna. Vice
  responsabile per il sistema di riscaldamento addizionale basato
  sull'iniezione di fasci di particelle neutre energetiche
%
\item \textbf{2009:} Task force leader per l'esperimento RFX-mod della
  task force \emph{Particle, Momentum and energy transport}. La task
  aveva l'incarico di implementare il programma scientifico e
  coordinare l'attivit{\`a} di analisi atte alla comprensione dei
  meccanismi fisici alla base del trasporto di particelle energia e
  momento.
%
\item \textbf{2010:} Task force leader per l'esperimento RFX-mod della
  task force \emph{Physics integration for high performance RFP}. La
  task aveva l'incarico di implementare il programma scientifico e
  coordinare l'attivit{\`a} di analisi atte alla comprensione dei
  meccanismi fisici alla base della transizione ai regimi di
  confinamento migliorati osservati ad alta corrente.
%
\item \textbf{2011:} Coordinatore del working group europeo EFDA
  \emph{3D field effects in edge and SOL and diagnostic development}
  implementato nell'EFDA Transport Topical Group. Questo gruppo di
  lavoro aveva il compito di coordinare gli sforzi di numerosi
  laboratori europei dedicati ai seguenti argomenti di ricerca:
  \begin{enumerate}[noitemsep,leftmargin=*,topsep=0pt,partopsep=0pt]
  \item Analisi e comprensione degli effetti di campi magnetici
    non-assialsimmetrici sulle strutture turbolente
  \item Analisi e comprensione delle modifiche ai meccanismi di
    transporto nella regione esterna indotti da campi magnetici 3D non
    assialsimmetrici
  \item Caratterizzazione della turbolenza nella regione esterna in
    sistemi 3D, inclusi gli effetti su ioni ed elettroni sovratermici
  \item Sviluppo di codici per lo studio della turbolenza e del
    trasporto che includano effetti di campi magnetici non
    assial-simmetrici.
  \item Confronto fra le diverse configurazioni magnetiche, tokamak,
    stellarators e reversed field pinches
  \end{enumerate}
  I coordinatori promuovo piattaforme di discussione fra le diverse
  associazioni europee e indicano possibili linee di ricerca da
  perseguire.
%
\item \textbf{2012:} Membro del Program committee del workshop
  \emph{17th Joint EU-US Transport Task Force Meeting in combination
    with the 4th EFDA Transport Topical Group meeting}, 3-6 September
  2012, Padova, Italy
%
\item \textbf{2013:} Coordinatore scientifico dell'esperimento
  \emph{B13-19 Investigation of M-Mode} presso l'esperimento europeo
  JET. Il ruolo implicava il coordinamento dell'attivit{\`a}
  preparatoria per la campagna sperimentale, dell'attivit{\`a} di
  analisi e della divulgazione tramite pubblicazione scientifica e
  presentazione a conferenze dei risultati raggiunti. Il lavoro svolto
  ha contribuito ai contributi su rivista  \cite{Solano:2017db,Refy:NuclearFusion2020}
%  
\item \textbf{2014:} Coordinatore scientifico dell'esperimento
  \emph{AUG14-2.2-3, SOL filamentary transport at high density},
  nell'ambito del Work Programme Medium-Sized Tokamaks (MST1)
  implementato dal consorzio europeo EUROfusion. L'attivit{\`a} ha
  comportato la stesura del programma sperimentale, il coordinamento
  dell'attivit{\`a} svolta dal team scientifico che coinvolgeva
  ricercatori da diversi laboratori europei nonch{\`e} l'attivit{\`a}
  di divulgazione e monitoring dei risultati scientifici.
  L'importo finanziato per l'esperimento
  coordinato si pu\'o quantificare in 220 k\euro includendo la stima
  del tempo macchina occupato dal progetto ed il costo
  medio del numero di ppds (persons per day) coordinati. Il lavoro svolto
  ha contribuito ai contributi su rivista \cite{Carralero:2016fs, Carralero:prl2015, Carralero:2017gb}

\item \textbf{2015-2016:} Coordinatore scientifico dell'esperimento
  \emph{TCV15-2.2-3: Filamentary Transport in the SOL} nell'ambito del
  Work Programme Medium-Sized Tokamaks (MST1) implementato dal
  consorzio europeo EUROfusion. L'attivit{\`a} ha comportato la
  stesura del programma sperimentale, il coordinamento
  dell'attivit{\`a} svolta dal team scientifico che coinvolgeva
  ricercatori da diversi laboratori europei nonch{\`e} l'attivit{\`a}
  di divulgazione e monitoring dei risultati scientifici.
  L'importo finanziato per l'esperimento
  coordinato si pu\'o quantificare in 135 k\euro includendo la stima
  del tempo macchina occupato dal progetto ed il costo
  medio del numero di ppds (persons per day) coordinati. Il lavoro
  svolto ha contribuito al contributo su rivista \cite{Vianello:2017ku}
%
\item \textbf{2015-2016:} Coordinatore scientifico dell'esperimento
  \emph{TCV15-1.5-1, Mitigation of high Z impurity accumulation
    through combined central ECRH and tailoring of MHD activity in
    high performance H-modes} nell'ambito del Work Programme
  Medium-Sized Tokamaks (MST1) implementato dal consorzio europeo
  EUROfusion. L'attivit{\`a} ha comportato la stesura del programma
  sperimentale, il coordinamento dell'attivit{\`a} svolta dal team
  scientifico che coinvolgeva ricercatori da diversi laboratori
  europei nonch{\`e} l'attivit{\`a} di divulgazione e monitoring dei
  risultati scientifici. L'importo finanziato per l'esperimento
  coordinato si pu\'o quantificare in 93 k\euro includendo la stima
  del tempo macchina occupato dal progetto ed il costo
  medio del numero di ppds (persons per day) coordinati. 
%
\item \textbf{2017-2018:} Coordinatore scientifico del Topic 21
  \emph{Filamentary transport in high-power H-mode conditions and in
    no/small-ELM regimes to predict heat and particle loads on PFCs
    for future devices } nell'ambito del Work Programme Medium-Sized
  Tokamaks (MST1) implementato dal consorzio europeo
  EUROfusion. L'attivit{\`a} ha comportato la stesura del programma
  sperimentale, il coordinamento dell'attivit{\`a} svolta dal team
  scientifico che coinvolgeva ricercatori da diversi laboratori
  europei nonch{\`e} l'attivit{\`a} di divulgazione e monitoring dei
  risultati scientifici. L'importo finanziato per l'esperimento
  coordinato si pu\'o quantificare in 840 k\euro includendo la stima
  del tempo macchina occupato dal progetto ed il costo
  medio del numero di ppds (persons per day) coordinati. 
  Il lavoro svolto ha contribuito ai seguenti
  prodotti 
  su rivista \cite{agostini:ppcf2019, vianello:nf2019}
%
\item \textbf{2018-2019:} Coordinatore scientifico del task T18-02
  \emph{Scrape-off layer and SOL- pedestal interaction } nell'ambito
  del Work Programme JET1 implementato dal consorzio europeo
  EUROfusion. L'attivit{\`a} ha comportato il coordinamento
  dell'attivit{\`a} svolta dal team scientifico che coinvolgeva
  ricercatori da diversi laboratori europei nonch{\`e} l'attivit{\`a}
  di divulgazione e monitoring dei risultati scientifici. L'ammontare
  del tempo uomo coordinato risulta pari a 374 ppd (person per day)
  nel 2018 e 503 ppdf nel 2019. Il ruolo svolto ha contribuito ai
  seguenti prodotti su rivista \cite{tamain:nme2021, tecxy:nme2021}
%
\item \textbf{2019-2020:} Coordinatore scientifico dell'esperiment
  M18-41 \emph{Divertor geometry effect on detachment and SOL }
  nell'ambito del Work Programme JET1 implementato dal consorzio
  europeo EUROfusion. L'attivit{\`a} ha comportato la stesura del
  programma sperimentale, il coordinamento dell'attivit{\`a} svolta
  dal team scientifico che coinvolgeva ricercatori da diversi
  laboratori europei nonch{\`e} l'attivit{\`a} di divulgazione e
  monitoring dei risultati scientifici. L'ammontare del tempo uomo
  coordinato \'e superiore ai 600 ppds (person per day) nel periodo
  2019-2020 per un totale di 8 sessioni sperimentali dedicate con
  esplorazioni anche durante la campagna in T.
  Il ruolo svolto ha
  contribuito ai seguenti prodotti su rivista \cite{lomanowski:nf2022} ad un invito alla conferenza della Division of
  Plasma Physics dell'American Physical Society
  \cite{lomanowski:aps2022separatrix} ed al contributo per la
  conferenza IAEA Fusion Energy Conference \cite{vianello:iaea2021sol}
 
\item \textbf{2019-2020:} Coordinatore scientifico del Topic 16
  \emph{Effect of filamentary transport on heat and particle loads }
  nell'ambito del Work Programme MST1 Medium Size Tokamak implementato
  dal consorzio europeo EUROfusion. L'attivit{\`a} ha comportato la
  stesura del programma sperimentale, il coordinamento
  dell'attivit{\`a} svolta dal team scientifico che coinvolgeva
  ricercatori da diversi laboratori europei nonch{\`e} l'attivit{\`a}
  di divulgazione e monitoring dei risultati scientifici.  Importo
  finanziato dal Consorzio EUROfusion per l'esperimento coordinato
  risulta pari a circa 800kEuro, stima basata sul tempo macchina
  occupato dal progetto coordinato sulla base del costo medio per
  impulso nel tokamak TCV operante a Losanna e del tokamak
  ASDEX-Upgrade operante in Germania.  Include inoltre il costo del
  tempo uomo (ppy) del team scientifico associato all’esperimento che
  il candidato ha coordinato. Il ruolo svolto \'e stato presentato
  alla conferenza IAEA Fusion Energy Conference 2020 \cite{vianello:iaea2021sol, henderson:iaea2020experimental}
%
\item \textbf{2020-} Rappresentante Europeo nominato
  dell'\emph{International Tokamak Physics Agreement (ITPA)} per il
  gruppo \emph{Pedestal \& Edge Physics Topical
    Group}. Responsabilit\`a del gruppo \'e quello di coordinare
  \emph{joint activity research } sui principali esperimenti di
  fusione termonucleare nel mondo.
%
\item \textbf{2020-} Responsabile del task D34 \emph{Far Scrape Off
    Layer transport} nell'ambito del gruppo ITPA
  Div-SOL. Responsabilit\'a del task coordinator \'e quella di
  coordinare l'attivit\'a sperimentale a livello internazionale su
  specifici topics,  al fine di estrarre le informazioni utili per
  l'estrapolazione a futuri esperimenti e reattori da fusione.
%
\item \textbf{2020-:} Deputy Task Force Leader del Work Package
  Tokamak Exploitation (WPTE) implementato dal consorzio europeo
  EUROfusion.  Il workpackage WPTE \`e stato istituito al fine di
  coordinare l'attivit\`a di ricerca europea sulla fusione in supporto
  al progetto ITER e alla finalizzazione del design dell'esperimento
  DEMO. L'attivit\'a sperimentale sar\`a svolta su 5 diversi
  esperimenti europei, ASDEX-Upgrade (Germania), TCV (Svizzera),
  MAST-U (UK), WEST (Francia) e JET (UK). Compito della task force \`e
  implementare e coordinare il programma sperimentale, monitorare il
  raggiungimento degli obbiettivi scientifici prefissati e stabilire
  il programma plurieannale per un efficiente utilizzo degli
  esperimenti di fusione termonucleare in Europa.

\item \textbf{2020-:} Membro dell'EUROfusion HPC allocation Commitee
  atto. Il ruolo del comitato ha il compito di valutare le proposte per
  l'utilizzo delle risorse per High Performance Computer in dotazione
  al consorzio EUROfusion e di allocare tali risorse sulla base delle
  priorit\'a del programma EUROfusion nonch\'e sulla base della
  valenza scientifica delle proposte stesse. 

\item \textbf{2022-:}  Membro dell'Expert Group 2 \emph{SOL and
    Divertor physics } per la definizione e la stesura di capitoli del
  Research Plan
  per l’esperimento DTT, ENEA, Frascati

\end{enumerate}
%%% Local Variables:
%%% mode: latex
%%% TeX-master: t
%%% End:
