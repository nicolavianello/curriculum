%% this is a second template for the curriculum vitae. I will use the
%% same approach used for the cvnicola.tex referring to the same
%% files in utils but with a different toggle so we do not use the
%% template used in the other for of curriculum


\documentclass[10pt, a4paper]{tabcv}
\addbibresource{utils/biblio.bib}
% this is choosing between files
% \usepackage{ltxcmds}
% \makeatletter
% \newcommand{\IfClassLoaded}[2]{\ltx@ifclassloaded{#1}{#2}{}}
% \makeatother
\usepackage{etoolbox}
\newtoggle{nicolacv}
\togglefalse{nicolacv}

%%%% XeLaTeX settings - Use these to set document fonts %%%
\usepackage{fontspec,xltxtra,xunicode}
\defaultfontfeatures{Mapping=tex-text}
\newfontfamily\bodyfont[]{Helvetica Neue}
\newfontfamily\thinfont[]{Helvetica Neue UltraLight}
\newfontfamily\headingfont[]{Helvetica Neue Condensed Bold}
\setmainfont[Mapping=tex-text, Color=textcolor]{Helvetica Neue Light} % same font used in
                               

% packages to make header
\usepackage{lettrine}
\usepackage{graphicx} % for resizebox

%%%%%%%%%%%%%%%%%%%%%%%%%%%%%%%%%%%%%%%%%%%%%%%%%%%%%%%%%%%%%%%%%%
%%%%%%  To highlight author name in publications sections   %%%%%%
%%%%%%%%%%%%%%%%%%%%%%%%%%%%%%%%%%%%%%%%%%%%%%%%%%%%%%%%%%%%%%%%%%
%\bibliographystyle{myPlain}
% otherwise try, e.g. \bibliographystyle{plain}

% \myname{*} will appear bold in \pubsin{}
\def\myname{Nicola Vianello}%

% you could change \textbf to \emph or any other command
 \newcommand{\FormatName}[1]{%
   \edef\name{#1}%
   \ifx\name\myname
     \textbf{#1}%
   \else
        #1%
   \fi
 }
%%%%%%%%%%%%%%%%%%%%%%%%%%%%%%%%%%%%%%%%%%%%%%%%%%%%%%%%%%%%%%%%%%
%%%%%%%%%%%%%%%%%%%%%%%%%%%%%%%%%%%%%%%%%%%%%%%%%%%%%%%%%%%%%%%%%%


\begin{document}

% if there are publications to be added
% comment out otherwise
%\nobibliography{utils/biblio}

%%%%%%% Title/Heading/Name %%%%%%%%%%%%
% You're on your own here... 

\begin{flushright}
\textcolor{darkBlue}{ \Huge
\lettrine[lines=2,findent=-1pt, loversize = -0.42,
lraise=0.6]{N}{icola Vianello}}\\[5pt]
\textcolor{Gray}{\resizebox{0.398\linewidth}{!}{Swiss Plasma Center EPFL}}\\
\textcolor{Gray}{\resizebox{0.398\linewidth}{!}{Batiment 13, CH-1025, Lausanne }}\\
\textcolor{Gray}{\resizebox{0.398\linewidth}{!}{+41 216934308 $\cdot$  nicola.vianello@epfl.ch }}

\end{flushright}

%%%%%%%%%%%%%%%%%%%%%%%%%%%%%%%%%%%
%%%%%%%%%%%%%%%%%%%%%%%%%%%%%%%%%%%
% I structured here using the same approach and the same 

\section{Education and Qualifications}
\begin{tabular}{lll}
1994 & \textbf{ High School} & Liceo Scientifico U. Morin, Mestre, Venezia,
\textit{56 out 60}\\
March 1999 & \textbf{\textit{Laurea in Fisica} } &
Universit\`a degli Studi di Padova, Padova, Italy \\
 & \textbf{(M.Sci Physics)} & \textit{110 out 110 cum Laude} \\
 &Thesis Title:  & \textbf{\emph{Trasporto di particelle ed energia per effetto
 di turbolenza elettrostatica}} \\
& & \textbf{\emph{in plasmi confinati in configurazione
 Reversed Field Pinch}} \\ 
& & (Particle and energy transport induced by
electrostatic turbulence in \\
& &  Reversed Field Pinch  plasmas) \\ 
& Supervisor: & Prof. S. Lo Russo, Dr. V. Antoni \\
 & Topics: & Electrostatic anomalous transport. Sheared Flows. \\ 
& & Active
 modification of boundary flow through edge biasing \\
 March 2002 & \textbf{Ph.D in Energetics} & Universit\`a degli Studi di Padova, Padova,
Italy \\
& Thesis Title: & \textbf{\emph{Self-organization phenomena and
    coherent structure generation}} \\
 & & \textbf{\emph{in magnetized plasmas}} \\
 & Supervisor: & Prof. A. Buffa, Dr. V. Antoni \\ 
 & Topics: & Electromagnetic turbulence in Reversed Field Pinches and Tokamaks. \\
& & Anomalous transport. Self Organized Criticality.
\end{tabular}


\section{Employment}
\begin{tabular}{lll}
March-October 1999 & \textbf{Consorzio RFX, Padova, Italy} & Borsista
(Research fellow) \\
November 2002 - April 2003 & \textbf{Consorzio RFX, Padova, Italy} &
Borsista (Research fellow) \\
May 2003-December 2005 & \textbf{Consorzio RFX, Padova, Italy} &
Research Scientist \\ 
January 2006 - July 2009 & \textbf{Consorzio RFX, Padova, Italy} &
Research Scientist, Permanent position \\
July 2009 - Date & \textbf{Consiglio Nazionale delle Ricerche} &
Researcher, Permanent position \\
& \textbf{(\emph{Research National Institute})} & \emph{See Competiotion section}\\
& \textbf{Padova, Italy} & 
\end{tabular}


\iftoggle{nicolacv}{
\section{Further experiences}
\begin{entrylist}
\entry
{03-06 2001}
{Visiting Scientist}
{Royal Institute of Technology 
Stockholm,  Sweden}
{under EURATOM-Mobility Staff Movement}


\entry
{05-06 2002}
{Visiting Scientist}
{Royal Institute of Technology 
Stockholm,  Sweden}
{under EURATOM-Mobility Staff Movement}

\entry
{03-04 2003}
{Visiting Scientist}
{Royal Institute of Technology 
Stockholm,  Sweden}
{under EURATOM-Mobility Staff Movement}

\entry
{04-06 2004}
{Visiting Scientist}
{Royal Institute of Technology 
Stockholm,  Sweden}
{under EURATOM-Mobility Staff Movement}

\entry
{10-11 2005}
{Visiting Scientist}
{Ris{\o} National Laboratory, Ris{\o} Denmark}
{under EURATOM-Mobility Staff Movement}

\entry
{02 2008}
{Visiting Scientist}
{Max Plank Instit{\"u}t f\"ur Plasma Physik,  Garching Germany}
{under EURATOM-Mobility Staff Movement}

\entry
{05 2009}
{Visiting Scientist}
{Max Plank Instit{\"u}t f\"ur Plasma Physik,  Garching Germany}
{under EURATOM-Mobility Staff Movement}

\entry
{11 2009}
{Visiting Scientist}
{CRPP,  EPFL,  Lausanne Switzerland}
{under EURATOM-Mobility Staff Movement}

\entry
{03 2011}
{Visiting Scientist}
{Royal Institute of Technology 
Stockholm,  Sweden}
{under EURATOM-Mobility Staff Movement}

\entry
{04 2011}
{Visiting Scientist}
{The National Fusion Laboratory  CIEMAT,  Madrid Spain}
{under EURATOM-Mobility Staff Movement}

\entry
{12 2011}
{Visiting Scientist}
{The National Fusion Laboratory  CIEMAT,  Madrid Spain}
{under EURATOM-Mobility Staff Movement}


\entry
{05 2011}
{Visiting Scientist}
{Max Plank Instit{\"u}t f\"ur Plasma Physik,  Garching Germany}
{under EURATOM-Mobility Staff Movement}

\entry
{02-03 2012}
{Staff Secondment at JET}
{Culham Centre for Fusion Science,  Culham, Oxford UK}
{}

\entry
{07-09 2013}
{Staff Secondment at JET}
{Culham Centre for Fusion Science,  Culham, Oxford UK}
{}

\entry
{05 2014}
{Visiting Scientist}
{Max Plank Instit{\"u}t f\"ur Plasma Physik,  Garching Germany}
{under EURATOM-Mobility Staff Movement}


\entry
{07 2014}
{Staff Secondment at JET}
{Culham Centre for Fusion Science,  Culham, Oxford UK}
{}


\end{entrylist}
}{
\begin{cvblock}{Experience}

\cvitem{March -- June 2001}{Visiting scientist at Royal Institute of
  Technology, Stockholm, Sweden}

\\[-6pt]

\cvitem{May -- June 2002}{Visiting scientist at Royal Institute of Technology, Stockholm, Sweden}

\\[-6pt]

\cvitem{March -- April 2003}{Visiting scientist at Royal Institute of Technology, Stockholm, Sweden}

\\[-6pt]

\cvitem{April -- June 2004}{Visiting scientist at Royal Institute of Technology, Stockholm, Sweden}

\\[-6pt]

\cvitem{October 2005}{Visiting scientist at Ris{\"o} National
  Laboratory,  Denmark}

\\[-6pt]

\cvitem{February 2008}{Visiting Scientist at Max-Planck
Institut f\"ur Plasmaphysik, Garching, Germany}

\\[-6pt]

\cvitem{May 2009}{Visiting Scientist at Max-Planck
Institut f\"ur Plasmaphysik, Garching, Germany}

\\[-6pt]

\cvitem{November 2009}{Visiting Scientist at Centre der Recherches en
  Physique des Plasmas, EPFL,  Lausanne}

\\[-6pt]

\cvitem{March 2011}{Visiting scientist at Royal Institute of
  Technology, Stockholm, Sweden}

\\[-6pt]

\cvitem{April 2011}{Visiting scientist at the National Fusion
  Laboratory, CIEMAT, Madrid}

\\[-6pt]

\cvitem{May 2011}{Visiting Scientist at Max-Planck
Institut f\"ur Plasmaphysik, Garching, Germany}

\\[-6pt]

\cvitem{February-March 2012}{Visiting Scientist at Culham Centre for
  Fusion Energy, Oxford, JET}

\\[-6pt]

\cvitem{July-September 2013}{Visiting Scientist at Culham Centre for
  Fusion Energy, Oxford, JET}

\\[-7pt]

\cvitem{May 2014}{Visiting Scientist at Max-Planck
Institut f\"ur Plasmaphysik, Garching, Germany}

\\[-7pt]

\cvitem{July 2014}{Visiting Scientist at Culham Centre for
  Fusion Energy, Oxford, JET}

\\[-7pt]

\cvitem{July 2015}{Visiting Scientist at Max-Planck
Institut f\"ur Plasmaphysik, Garching, Germany}

\\[-7pt]

\cvitem{July 2015}{Visiting Scientist at Max-Planck
Institut f\"ur Plasmaphysik, Garching, Germany}

\\[-7pt]

\cvitem{October 2015}{Visiting Scientist at Max-Planck
  Institut f\"ur Plasmaphysik, Garching, Germany}

\\[-7pt]

\cvitem{February 2016}{Visiting Scientist at Max-Planck
  Institut f\"ur Plasmaphysik, Garching, Germany}

\\[-7pt]

\cvitem{May 2016}{Visiting Scientist at Swiss Plasma Centre, EPFL,
  Lausanne}

\\[-7pt]

\cvitem{July 2016}{Visiting Scientist at Swiss Plasma Centre, EPFL,
  Lausanne}

\\[-7pt]

\cvitem{April 2017}{Visiting scientist (fellowship), within EUROfusion
  framework, at Max-Planck
  Institut f\"ur Plasmaphysik, Garching, Germanoa}

\\[-7pt]

\cvitem{May 2017}{Visiting scientist (fellowship), within EUROfusion
  framework, at Max-Planck
  Institut f\"ur Plasmaphysik, Garching, Germanoa}

\\[-7pt]

\cvitem{June 2017}{Visiting scientist (fellowship), within EUROfusion
  framework, at Swiss Plasma Centre, EPFL,  Lausanne}

\\[-7pt]

\cvitem{September 2017}{Visiting scientist (fellowship), within EUROfusion
  framework, at the Swiss Plasma Centre, EPFL,  Lausanne}

\\[-7pt]

\cvitem{November 2017}{Visiting scientist (fellowship), within EUROfusion
  framework, at the Swiss Plasma Centre, EPFL,  Lausanne}

\cvitem{July 2018}{Visiting scientist (fellowship), within EUROfusion framework, at Culham Centre for
  Fusion Energy, Oxford, UK}

\cvitem{2019-date}{Several visits to different European Laboratories,
  mainly Swiss Plasma Center at EPFL,  Culham Centre for Fusion Energy
at Culham, UK and Max-Planck
  Institut f\"ur Plasmaphysik, Garching all within EUROfusion
  framework scheme}
\end{cvblock}
% \begin{longtable}{llll}
% 2001 & 5 March-15 June & Visiting scientist under& Royal Institute of
% Technology\\
%  & & EURATOM-Mobility
% staff movement &  Stockholm, Sweden \\

% 2002 & 1 May-30 June & Visiting scientist under& Royal Institute of
% Technology\\
%  & & EURATOM-Mobility
% staff movement &  Stockholm, Sweden \\
% 2003 & 2 March-30 April & Visiting scientist under& Royal Institute of
% Technology\\
%  & & EURATOM-Mobility
% staff movement &  Stockholm, Sweden \\
% 2004 & 19 April -19 June & Visiting scientist under& Royal Institute of
% Technology\\
%  & & EURATOM-Mobility
% staff movement &  Stockholm, Sweden \\
% 2005 & 16 October - & Visiting scientist under& Ris{\o}
% National Laboratory\\
%  & 19 November & EURATOM-Mobility
% staff movement &  Ris{\o}, Denmark \\
% 2008 & 11 - 15 February & Visiting scientist under& Max-Planck
% Institut f\"ur Plasmaphysik \\
%  & & EURATOM-Mobility
% staff movement &  Garching, Germany \\
% 2009 & 12 - 15 May & Visiting scientist under& Max-Planck
% Institut f\"ur Plasmaphysik \\
%  & & EURATOM-Mobility
% staff movement &  Garching, Germany \\
% 2009 & 09 - 13 November & Visiting scientist under& Centre der
% Recherches en Physique des \\
%  & & EURATOM-Mobility
% staff movement & Plasmas, EPFL, Lausanne, Switzerland \\
% 2011 & 07 - 11 March & Visiting scientist under& Royal Institute of
% Technology\\
%  & & EURATOM-Mobility
% staff movement &  Stockholm, Sweden \\
% 2011 & 13 - 15 April & Visiting scientist under& The National Fusion
% Laboratory, CIEMAT \\
%  & & EURATOM-Mobility
% staff movement &  Madrid, Spain \\
% 011 & 23 - 27 May & Visiting scientist under& Max-Planck
% Institut f\"ur Plasmaphysik \\
%  & & EURATOM-Mobility
% staff movement &  Garching, Germany \\
% 2012 & 06 February -  & Secondment Staff & JET, Culham Centre for
% Fusion Science \\
%  & 30 March & &  Culham, Oxford, UK \\
% 2013 & 14 July -  & Secondment Staff & JET, Culham Centre for
% Fusion Science \\
%  & 11 September & &  Culham, Oxford, UK \\
% \end{longtable}
}

\section{Invited lectures and conference talk}
\begin{longtable}{llll}
July 2012 & \textbf{Workshop on Electric Field, Turbulence and} & \textit{The role
  of 3D fields on edge and SOL turbulence} \\
& \textbf{Self-Organisation
in Magnetized Plasmas} & 
\end{longtable}

% \begin{cvblock}{Publications}
% \nocite{*}
% \end{cvblock}
% I have authored a total number of \printbibliography[env=counter,
%  heading=counter, type=article, resetnumbers=true] Articles in peer reviewed journal,  
%  \printbibliography[env=counter, heading=counter, type=inproceedings, resetnumbers=true]
%  conference proceedings and personally presented
%  \printbibliography[env=counter, heading=counter, type=misc, resetnumbers=true] oral
%  contribution. \\
% h-index factor: 21 according to ISI Web of Knowledge (last update \today)
% \printbibliography[type=article, title={article in peer-review journal}, heading=subbibliography, prefixnumbers={A}, resetnumbers=true]
% \printbibliography[type=inproceedings, title={National and
%    international conference}, heading=subbibliography, prefixnumbers={B}, resetnumbers=true]
% \printbibliography[type=misc, title={First author oral contribution}, heading=subbibliography, prefixnumbers={C}, resetnumbers=true]
% \printbibliography[type=report, title={Report}, heading=subbibliography, prefixnumbers={D}, resetnumbers=true]
\end{document}
