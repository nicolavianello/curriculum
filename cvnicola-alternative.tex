%% this is a second template for the curriculum vitae. I will use the
%% same approach used for the cvnicola.tex referring to the same
%% files in utils but with a different toggle so we do not use the
%% template used in the other for of curriculum


\documentclass[10pt, a4paper]{tabcv}
\addbibresource{utils/biblio.bib}
% this is choosing between files
% \usepackage{ltxcmds}
% \makeatletter
% \newcommand{\IfClassLoaded}[2]{\ltx@ifclassloaded{#1}{#2}{}}
% \makeatother
\usepackage{etoolbox}
\newtoggle{nicolacv}
\togglefalse{nicolacv}

%%%% XeLaTeX settings - Use these to set document fonts %%%
\usepackage{fontspec,xltxtra,xunicode}
\defaultfontfeatures{Mapping=tex-text}
\newfontfamily\bodyfont[]{Helvetica Neue}
\newfontfamily\thinfont[]{Helvetica Neue UltraLight}
\newfontfamily\headingfont[]{Helvetica Neue Condensed Bold}
\setmainfont[Mapping=tex-text, Color=textcolor]{Helvetica Neue Light} % same font used in
                               

% packages to make header
\usepackage{lettrine}
\usepackage{graphicx} % for resizebox

%%%%%%%%%%%%%%%%%%%%%%%%%%%%%%%%%%%%%%%%%%%%%%%%%%%%%%%%%%%%%%%%%%
%%%%%%  To highlight author name in publications sections   %%%%%%
%%%%%%%%%%%%%%%%%%%%%%%%%%%%%%%%%%%%%%%%%%%%%%%%%%%%%%%%%%%%%%%%%%
%\bibliographystyle{myPlain}
% otherwise try, e.g. \bibliographystyle{plain}

% \myname{*} will appear bold in \pubsin{}
\def\myname{Nicola Vianello}%

% you could change \textbf to \emph or any other command
 \newcommand{\FormatName}[1]{%
   \edef\name{#1}%
   \ifx\name\myname
     \textbf{#1}%
   \else
        #1%
   \fi
 }
%%%%%%%%%%%%%%%%%%%%%%%%%%%%%%%%%%%%%%%%%%%%%%%%%%%%%%%%%%%%%%%%%%
%%%%%%%%%%%%%%%%%%%%%%%%%%%%%%%%%%%%%%%%%%%%%%%%%%%%%%%%%%%%%%%%%%


\begin{document}

% if there are publications to be added
% comment out otherwise
%\nobibliography{utils/biblio}

%%%%%%% Title/Heading/Name %%%%%%%%%%%%
% You're on your own here... 

\begin{flushright}
\textcolor{darkBlue}{ \Huge
\lettrine[lines=2,findent=-1pt, loversize = -0.42,
lraise=0.6]{N}{icola Vianello}}\\[5pt]
\textcolor{Gray}{\resizebox{0.398\linewidth}{!}{Swiss Plasma Center EPFL}}\\
\textcolor{Gray}{\resizebox{0.398\linewidth}{!}{Batiment 13, CH-1025, Lausanne }}\\
\textcolor{Gray}{\resizebox{0.398\linewidth}{!}{+41 216934308 $\cdot$  nicola.vianello@epfl.ch }}

\end{flushright}

%%%%%%%%%%%%%%%%%%%%%%%%%%%%%%%%%%%
%%%%%%%%%%%%%%%%%%%%%%%%%%%%%%%%%%%
% I structured here using the same approach and the same 

\section{Education and Qualifications}
\begin{tabular}{lll}
1994 & \textbf{ High School} & Liceo Scientifico U. Morin, Mestre, Venezia,
\textit{56 out 60}\\
1999 & \textbf{\textit{Laurea in Fisica} } &
Universit\'a degli Studi di Padova, Padova, Italy \\
 & \textbf{(M.Sci Physics)} & \textit{110 out 110 cum Laude} \\
 &Thesis Title:  & \textbf{\emph{Trasporto di particelle ed energia per effetto
 di turbolenza elettrostatica}} \\
& & \textbf{\emph{in plasmi confinati in configurazione
 Reversed Field Pinch}} \\ 
& & (Particle and energy transport induced by
electrostatic turbulence in \\
& &  Reversed Field Pinch  plasmas) \\ 
2002 & \textbf{Ph.D} & Universit\'a degli Studi di Padova, Padova,
Italy \\
& Thesis Title: & \textbf{\emph{Fenomeni di auto organizzazione e
    generazione di strutture coerenti}} \\
& & \textbf{in plasmi magnetizzati} \\
& & (Self-organization phenomena and coherent structure generation \\
& & in magnetized plasmas)
\end{tabular}


\section{Employment}
\begin{tabular}{lll}
March-October 1999 & \textbf{Consorzio RFX, Padova, Italy} & Research fellow \\
November 2002 - April 2003 & \textbf{Consorzio RFX, Padova, Italy} &
Research fellow \\
May 2003-December 2005 & \textbf{Consorzio RFX, Padova, Italy} &
Research Scientist \\ 
January 2006 - July 2009 & \textbf{Consorzio RFX, Padova, Italy} &
Research Scientist, Permanent position \\
July 2009 - Date & \textbf{Consiglio Nazionale delle Ricerche} &
Researcher, Permanent position \\
& \textbf{(\emph{Research National Institute})} & \emph{See Competion section}\\
& \textbf{Padova, Italy} & 
\end{tabular}

\begin{cvblock}{International Experience}

\cvitem{March -- June 2001}{Visiting scientist at Royal Institute of
  Technology, Stockholm, Sweden}

\\[-4pt]

\cvitem{May -- June 2002}{Visiting scientist, under EURATOM-Mobility Staff Movement,  at Royal Institute of Technology, Stockholm, Sweden}

\\[-4pt]

\cvitem{March -- April 2003}{Visiting scientist, under EURATOM-Mobility Staff Movement, at Royal Institute of Technology, Stockholm, Sweden}

\\[-4pt]

\cvitem{April -- June 2004}{Visiting scientist, under EURATOM-Mobility Staff Movement, at Royal Institute of Technology, Stockholm, Sweden}

\\[-4pt]

\cvitem{October 2005}{Visiting scientist, under EURATOM-Mobility Staff Movement, at Ris{\"o} National
  Laboratory,  Denmark}

\\[-4pt]

\cvitem{February 2008}{Visiting Scientist, under EURATOM-Mobility Staff Movement, at Max-Planck
Institut f\"ur Plasmaphysik, Garching, Germany}

\\[-4pt]

\cvitem{May 2009}{Visiting Scientist, under EURATOM-Mobility Staff Movement, at Max-Planck
Institut f\"ur Plasmaphysik, Garching, Germany}

\\[-4pt]

\cvitem{November 2009}{Visiting Scientist, under EURATOM-Mobility Staff Movement, at Centre der Recherches en
  Physique des Plasmas, EPFL,  Lausanne}

\\[-4pt]

\cvitem{March 2011}{Visiting scientist, under EURATOM-Mobility Staff Movement, at Royal Institute of
  Technology, Stockholm, Sweden}

\\[-4pt]

\cvitem{April 2011}{Visiting scientist, under EURATOM-Mobility Staff Movement, at the National Fusion
  Laboratory, CIEMAT, Madrid}

\\[-4pt]

\cvitem{May 2011}{Visiting Scientist, under EURATOM-Mobility Staff Movement, at Max-Planck
Institut f\"ur Plasmaphysik, Garching, Germany}

\\[-4pt]

\cvitem{February-March 2012}{Visiting Scientist, under EURATOM-Mobility Staff Movement, at Culham Centre for
  Fusion Energy, Oxford, JET}

\\[-4pt]

\cvitem{July-September 2013}{Visiting Scientist, under EURATOM-Mobility Staff Movement, at Culham Centre for
  Fusion Energy, Oxford, JET}

\\[-4pt]

\cvitem{May 2014}{Visiting Scientist at Max-Planck
Institut f\"ur Plasmaphysik, Garching, Germany}

\\[-4pt]

\cvitem{July 2014}{Visiting Scientist at Culham Centre for
  Fusion Energy, Oxford, JET}

\\[-4pt]

\cvitem{July 2015}{Visiting Scientist at Max-Planck
Institut f\"ur Plasmaphysik, Garching, Germany}

\\[-4pt]

\cvitem{July 2015}{Visiting Scientist at Max-Planck
Institut f\"ur Plasmaphysik, Garching, Germany}

\\[-4pt]

\cvitem{October 2015}{Visiting Scientist at Max-Planck
  Institut f\"ur Plasmaphysik, Garching, Germany}

\\[-4pt]

\cvitem{February 2016}{Visiting Scientist at Max-Planck
  Institut f\"ur Plasmaphysik, Garching, Germany}

\\[-4pt]

\cvitem{May 2016}{Visiting Scientist at Swiss Plasma Centre, EPFL,
  Lausanne}

\\[-4pt]

\cvitem{July 2016}{Visiting Scientist at Swiss Plasma Centre, EPFL,
  Lausanne}

\\[-4pt]

\cvitem{April 2017}{Visiting scientist (fellowship), within EUROfusion
  framework, at Max-Planck
  Institut f\"ur Plasmaphysik, Garching, Germanoa}

\\[-4pt]

\cvitem{May 2017}{Visiting scientist (fellowship), within EUROfusion
  framework, at Max-Planck
  Institut f\"ur Plasmaphysik, Garching, Germanoa}

\\[-4pt]

\cvitem{June 2017}{Visiting scientist (fellowship), within EUROfusion
  framework, at Swiss Plasma Centre, EPFL,  Lausanne}

\\[-4pt]

\cvitem{September 2017}{Visiting scientist (fellowship), within EUROfusion
  framework, at the Swiss Plasma Centre, EPFL,  Lausanne}

\\[-4pt]

\cvitem{November 2017}{Visiting scientist (fellowship), within EUROfusion
  framework, at the Swiss Plasma Centre, EPFL,  Lausanne}

\\[-4pt]

\cvitem{July 2018}{Visiting scientist (fellowship), within EUROfusion framework, at Culham Centre for
  Fusion Energy, Oxford, UK}

\\[-4pt]

\cvitem{2019-date}{Several visits to different European Laboratories,
  mainly Swiss Plasma Center at EPFL,  Culham Centre for Fusion Energy
at Culham, UK and Max-Planck
  Institut f\"ur Plasmaphysik, Garching all within EUROfusion
  framework scheme}
\end{cvblock}

\begin{cvblock}{Invited Lectures}
\cvitem{July 2012}{Invited lecture at the \emph{Workshop on Electric
    Field, Turbulence Self Organization in Magnetized Plasmas}, Stockholm, Sweden}
\cvitem{Title}{\emph{The role of 3D fields on edge and SOL
    turbulence}}
\\[-6pt]
\cvitem{July 2014}{Invited lecture at the \emph{41$^{st}$ EPS
    Conference in Plasma Physics}, Berlin, Germany}
\cvitem{Title}{\emph{Magnetic perturbation as a viable tool for edge
    turbulence modification}}
\\[-6pt]

\cvitem{December 2014}{Invited lecture at the \emph{1st
International and Interdisciplinary Workshop on Fusion and Technological Plasmas (FUSTECH)},
Collaborative Research Center SFB‐TR87, Ruhr‐University Bochum}

\\[-10pt]

\cvitem{Title}{\emph{Fluctuations in tokamaks and RFPs: Relation with
    topology}}
\end{cvblock}


% \begin{cvblock}{Publications}
% \nocite{*}
% \end{cvblock}
% I have authored a total number of \printbibliography[env=counter,
%  heading=counter, type=article, resetnumbers=true] Articles in peer reviewed journal,  
%  \printbibliography[env=counter, heading=counter, type=inproceedings, resetnumbers=true]
%  conference proceedings and personally presented
%  \printbibliography[env=counter, heading=counter, type=misc, resetnumbers=true] oral
%  contribution. \\
% h-index factor: 21 according to ISI Web of Knowledge (last update \today)
% \printbibliography[type=article, title={article in peer-review journal}, heading=subbibliography, prefixnumbers={A}, resetnumbers=true]
% \printbibliography[type=inproceedings, title={National and
%    international conference}, heading=subbibliography, prefixnumbers={B}, resetnumbers=true]
% \printbibliography[type=misc, title={First author oral contribution}, heading=subbibliography, prefixnumbers={C}, resetnumbers=true]
% \printbibliography[type=report, title={Report}, heading=subbibliography, prefixnumbers={D}, resetnumbers=true]
\end{document}
