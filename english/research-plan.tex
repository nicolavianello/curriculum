\documentclass[12pt,a4paper]{article}
\usepackage[english]{babel}
\usepackage{geometry}
\geometry{a4paper,margin=2.5cm}
\usepackage{amsmath}									% Math
% ----------
% font used
\usepackage{fontspec,lipsum, xltxtra, xunicode}
\defaultfontfeatures{Mapping=tex-text}
\newfontfamily\bodyfont[]{Helvetica Neue}
\newfontfamily\thinfont[]{Helvetica Neue UltraLight}
\newfontfamily\headingfont[]{Helvetica Neue Condensed Bold}
\setmainfont[Mapping=tex-text, Color=textcolor]{Helvetica Neue Light} % same font used in
% ----------
% color code
\usepackage[usenames,dvipsnames,svgnames,table]{xcolor}
\definecolor{lg}{gray}{0.85} % used for hrules
\definecolor{darkBlue}{RGB}{39, 64, 139} % used in headings
\definecolor{bl}{rgb}{0.0,0.2,0.6} % alternative blue
% ----------
%   Biblio
\usepackage[style=phys,biblabel=brackets,backend=biber]{biblatex}
\addbibresource{../utils/biblio.bib}
\addbibresource{../utils/others.bib}
\renewcommand{\bibfont}{\normalfont\footnotesize}
% ----------
% Sections
\usepackage{sectsty}
\usepackage[compact]{titlesec} 
\titleformat{\section}[block]{\normalfont\Large\color{darkBlue}}{\thesection}{1em}{}[\vspace{-6pt}\textcolor{lg}{\titlerule}]
\titleformat{\subsection}[block]{\normalfont\color{darkBlue}}{\thesection}{1em}{}


% ----------
%  header
\usepackage{fancyhdr}
	\pagestyle{fancy}					% Enabling the custom headers/footers
\usepackage{lastpage}	
	% Header (empty)
	\lhead{}
	\chead{}
	\rhead{}
	% Footer (you may change this to your own needs)
	\lfoot{\footnotesize \textcolor{Gray}{\textit{Dr. Vianello Research Statement}}}
	\cfoot{}
	\rfoot{\footnotesize \textcolor{Gray}{\textit{page \thepage\ of \pageref{LastPage}}}}	% "Page 1 of 2"
	\renewcommand{\headrulewidth}{0.0pt}
	\renewcommand{\footrulewidth}{0.4pt}

% ----------
% Change the abstract environment
\usepackage[runin]{abstract}			% runin option for a run-in title
\setlength\absleftindent{30pt}		% left margin
\setlength\absrightindent{30pt}		% right margin
\abslabeldelim{\quad}						% 
\setlength{\abstitleskip}{-10pt}
\renewcommand{\abstractname}{}
\renewcommand{\abstracttextfont}{\color{bl} \small \slshape}	% slanted text
% ----------
%   Lettrine
\usepackage{lettrine}
%%% Start of the document
\begin{document}
\begin{flushright}
\textcolor{darkBlue}{ \Huge
\lettrine[lines=3,findent=-1pt, loversize = -0.42,
lraise=0.6]{N}{icola Vianello}}\\[2pt]
\textcolor{Gray}{\resizebox{0.288\linewidth}{!}{Swiss Plasma Center EPFL}}\\
\textcolor{Gray}{\resizebox{0.288\linewidth}{!}{Batiment 13, CH-1025, Lausanne }}\\
\textcolor{Gray}{\resizebox{0.288\linewidth}{!}{+41 216934308 $\cdot$  nicola.vianello@epfl.ch }}
\end{flushright}
% now the title
% ----------
%   Title 
\begin{center} 
\textcolor{darkBlue}{ \Huge
\lettrine[lines=2,findent=-1pt, loversize = -0.42,
lraise=0.6]{F}{luctuations in the era of} \lettrine[lines=2,findent=-1pt, loversize = -0.42,
lraise=0.6]{B}urning plasmas}
\end{center}


\section*{Introduction}
Plasmas, and in particular Fusion plasmas represent a complex system
where many interacting degrees of freedom coexist determining a
variety of non-linear behavior spreading over a broad range of
spatio-temporal scales
\cite{Kadomtsev:1992us,Sornette:2006dt}. It is known since a lot of
time that simplified diffusion paradigm can't correctly describe the
transport of energy, particles and momentum and a different
description of this dynamical system, generally found close to
marginal stability needs to be used. Proper description of plasma
dynamics require consequently to disentangle the role played by
fluctuations, which are found to emerge at all spatial and temporal
scales. My personal research activity has been indeed devoted to the analysis,
interpretation and modeling of experimental and numerical results
obtained in magnetized plasmas, with emphasis
on magnetically confined ones. More in details I've focused my effort
on electromagnetic fluctuations induced transport of energy, particle
and momentum, with interpretation in the wider framework of turbulence
theory. During my research activity I have collected a large
experience on electromagnetic transport analysis, working on different
magnetic configurations, from Reversed Field Pinches (working on
RFX-mod operating in Padova, Extrap-T2R operating in Stockholm and
TPE-1RM20 which was in operation in Japan), stellarators (with
experimental activity on TJ-II heliac type operating in Spain) and Tokamaks
(ASDEX-Upgrade and JET), and low temperature plasmas as the Simple
Magnetized Torus experiment TORPEX at EPFL. In the following I will
try to provide you with a brief overview of my principal research
interests and result as well as to guide to possible future field of
research I think deserve attention and effort. 
\subsection*{prova}
\end{document}
