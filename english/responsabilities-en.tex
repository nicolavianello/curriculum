\section{Duties and Responsibilities}
\begin{longtable}{>{\bfseries}l p{15cm}}
2010 - Date & Responsible Scientist for edge manipulators in RFX-mod
device. Responsibilities implies the maintenance and improvement of
the two manipulators used in RFX-mod for the insertion of edge probes,
including maintenance and improvement of the probe heads. Development of new
complex probe head, project which has required the coordination
between design, mechanical and diagnostic technicians.  \\
2009 & Task force leader in RFX-mod experiment for task force
\emph{Particle, Momentum and energy transport}. The task force was in
charge to implement experimental proposals aimed to the comprehension
of physical mechanisms which regulate particle momentum and energy
transport in RFX-mod. The task force leaders together with the
Scientific Coordinators take part to the decision processes concerning
the experimental program of the machine, deciding priorities and
objectives. \\
2010 & Task force leader in RFX-mod experiment for task force
\emph{Physics integration for high performance RFP}. The task force
aimed to coordinate all the efforts devoted to the comprehension of
the physical mechanism behind the  appearance of improved confinement
regimes in RFX-mod, to establish the physical requirement for a controlled achievement  of h-mode confinement regime
and to explore all the still open basic physics issues whose knowledge
could help to improve plasma performances. As in the previous year the
task force leaders take part to the scientific program schedule,
coordinating in particular the activities for the high current
performance operations. \\
2011 & Coordinator of the EFDA working group \emph{3D field effects in
  edge and SOL and diagnostic development} under EFDA Transport
Topical Group. This working group has been established to
coordinate the effort promoted by different EFDA associations on the
following subject:
\begin{enumerate}
\item Investigation on  the effect of non-axisymmetric fields on the filamentary structures (L and H-mode regimes)
\item Investigation into changes in edge transport due to the application of 3D fields
\item Characterization of the edge turbulence in these 3D situations (including effect of ion temperature and 3D fast particle losses)
\item Edge turbulence and transport modeling by incorporating 3D field effects into the codes.
\item Comparison studies between tokamaks, stellarators and RFPs on the above topics.
\end{enumerate}
The coordinators promote exchange of results between different
association and the definition of common objectives which facilitate
the comparison between different devices. \\
2012 & Member of the Program committee of the 17th Joint EU-US Transport Task Force
Meeting in combination with the 4th EFDA Transport Topical Group
meeting, 3-6 September 2012, Padova, Italy
\end{longtable}
