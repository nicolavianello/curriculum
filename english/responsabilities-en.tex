\iftoggle{nicolacv}{
\section{Duties and Responsibilities}
\begin{longtable}{>{\bfseries}l p{11cm}}
2007 - 2015  & Responsible Scientist for edge manipulators in RFX-mod
device. Responsibilities implies the maintenance and improvement of
the two manipulators used in RFX-mod for the insertion of edge probes,
including maintenance and improvement of the probe heads. Development of new
complex probe head, project which has required the coordination
between design, mechanical and diagnostic technicians.  \\
2009 & Task force leader in RFX-mod experiment for task force
\emph{Particle, Momentum and energy transport}. The task force was in
charge to implement experimental proposals aimed to the comprehension
of physical mechanisms which regulate particle momentum and energy
transport in RFX-mod. The task force leaders together with the
Scientific Coordinators take part to the decision processes concerning
the experimental program of the machine, deciding priorities and
objectives. \\
2010 & Task force leader in RFX-mod experiment for task force
\emph{Physics integration for high performance RFP}. The task force
aimed to coordinate all the efforts devoted to the comprehension of
the physical mechanism behind the  appearance of improved confinement
regimes in RFX-mod, to establish the physical requirement for a controlled achievement  of h-mode confinement regime
and to explore all the still open basic physics issues whose knowledge
could help to improve plasma performances. As in the previous year the
task force leaders take part to the scientific program schedule,
coordinating in particular the activities for the high current
performance operations. \\
2011 & Coordinator of the EFDA working group \emph{3D field effects in
  edge and SOL and diagnostic development} under EFDA Transport
Topical Group. This working group has been established to
coordinate the effort promoted by different EFDA associations on the
following subject:
\begin{enumerate}
\item Investigation on  the effect of non-axisymmetric fields on the filamentary structures (L and H-mode regimes)
\item Investigation into changes in edge transport due to the application of 3D fields
\item Characterization of the edge turbulence in these 3D situations (including effect of ion temperature and 3D fast particle losses)
\item Edge turbulence and transport modelling by incorporating 3D field effects into the codes.
\item Comparison studies between tokamaks, stellarators and RFPs on the above topics.
\end{enumerate}
The coordinators promote exchange of results between different
association and the definition of common objectives which facilitate
the comparison between different devices. \\
2012 & Member of the Program committee of the 17th Joint EU-US Transport Task Force
Meeting in combination with the 4th EFDA Transport Topical Group
meeting, 3-6 September 2012, Padova, Italy \\
2013 & Scientific Coordinator of back-up experiment \textbf{B13-19
  Investigation of M-Mode} on JET
campaigns C31-C34. Coordination implies assigning activities to the
experimental team, plan the possible experimental campaign to be
designed in collaboration with Session leaders,  establish scientific
objectives and monitoring scientific activities. \\
2014 & Scientific Coordinator of experiment \textbf{AUG14-2.2.3, SOL
filamentary transport at high density}, under the MST1 Eurofusion
Work-Packages \\
2015 & Scientific Coordinator of experiment
  \textbf{TCV15-2.2-3: Filamentary Transport in the SOL} under MST1
  Eurofusion Work-Package \\
2015 & Scientific Coordinator of experiment
  \textbf{TCV15-1.5-1, Mitigation of high Z impurity accumulation
    through combined central ECRH and tailoring of MHD activity in
    high performance H-modes} under MST1 Eurofusion Work-Package \\

\end{longtable}}
{\begin{cvblock}{Duties and Responsibilities}
\cvitem{2007--2015}{ Responsible Scientist for edge manipulators in RFX-mod
device. Responsibilities implies the maintenance and improvement of
the two manipulators used in RFX-mod for the insertion of edge probes
and the development of new
probe heads with the coordination
between design, mechanical and diagnostic technicians.}
\\[-4pt]
\cvitem{2015-2016}{ Responsible Scientist Soft X ray diagnostic in the
  TCV tokamak. Deputy Responsible Scientist for the Neutral Beam
  Heating system in the TCV tokamak.}

\\[-4pt]

\cvitem{2009}{Task force leader in RFX-mod experiment for task force
\emph{Particle, Momentum and energy transport}. The task force was in
charge to implement experimental proposals aimed to the comprehension
of physical mechanisms which regulate particle momentum and energy
transport in RFX-mod. The task force leaders together with the
Scientific Coordinators take part to the decision processes concerning
the experimental program of the machine, deciding priorities and
objectives}

\\[-4pt]

\cvitem{2010}{Task force leader in RFX-mod experiment for task force
\emph{Physics integration for high performance RFP}. The task force
aimed to coordinate all the efforts devoted to the comprehension of
the physical mechanism behind the  appearance of improved confinement
regimes in RFX-mod, to establish the physical requirement for a controlled achievement  of h-mode confinement regime
and to explore all the still open basic physics issues whose knowledge
could help to improve plasma performances. As in the previous year the
task force leaders take part to the scientific program schedule,
coordinating in particular the activities for the high current
performance operations.}

\\[-4pt]

\cvitem{2011}{Coordinator of the EFDA working group \emph{3D field effects in
  edge and SOL and diagnostic development} under EFDA Transport
Topical Group. This working group has been established to
coordinate the effort promoted by different EFDA associations on the
following subject:
\begin{enumerate}[noitemsep,leftmargin=*,topsep=0pt,partopsep=0pt]
\item Investigation on  the effect of non-axisymmetric fields on the filamentary structures (L and H-mode regimes)
\item Investigation into changes in edge transport due to the application of 3D fields
\item Characterization of the edge turbulence in these 3D situations (including effect of ion temperature and 3D fast particle losses)
\item Edge turbulence and transport modelling by incorporating 3D field effects into the codes.
\item Comparison studies between tokamaks, stellarators and RFPs on the above topics.
\end{enumerate}
The coordinators promote exchange of results between different
association and the definition of common objectives which facilitate
the comparison between different devices.}

\\[-6pt]

\cvitem{2012}{Member of the Program committee of the 17th Joint EU-US Transport Task Force
Meeting in combination with the 4th EFDA Transport Topical Group
meeting, 3-6 September 2012, Padova, Italy}

\\[-4pt]

\cvitem{2013}{Scientific Coordinator of experiment \emph{B13-19
  Investigation of M-Mode} on JET Tokamak
campaigns C31-C34. Coordination implies assigning activities to the
experimental team, plan the possible experimental campaign to be
designed in collaboration with Session leaders,  establish scientific
objectives and monitoring scientific activities.}

\\[-4pt]

\cvitem{2014}{Scientific Coordinator of experiment \emph{AUG14-2.2-3, SOL
filamentary transport at high density}, under the MST1 Eurofusion
Work-Packages. Coordination implies assigning activities to the
international 
experimental team, plan the possible experimental campaign to be
designed in collaboration with Session leaders,  establish scientific
objectives and monitoring scientific activities.}

\\[-4pt]

\cvitem{2015-2016}{Scientific Coordinator of experiment
  \emph{TCV15-2.2-3: Filamentary Transport in the SOL} under MST1
  Eurofusion Work-Package. Coordination implies assigning activities to the
international 
experimental team, plan the possible experimental campaign to be
designed in collaboration with Session leaders,  establish scientific
objectives and monitoring scientific activities.}

\\[-4pt]

\cvitem{2015-2016}{Scientific Coordinator of experiment
  \emph{TCV15-1.5-1, Mitigation of high Z impurity accumulation
    through combined central ECRH and tailoring of MHD activity in
    high performance H-modes} under MST1 Eurofusion Work-Package. Coordination implies assigning activities to the
international 
experimental team, plan the possible experimental campaign to be
designed in collaboration with Session leaders,  establish scientific
objectives and monitoring scientific activities.}

\\[-4pt]

\cvitem{2017-2018}{Scientific Coordinator of Topic 21
  \emph{Filamentary transport in high-power H-mode conditions and in
    no/small-ELM regimes to predict heat and particle loads on PFCs
    for future devices }
  under MST1 Eurofusion Work-Package. Coordination implies assigning activities to the
international 
experimental team, plan the experimental campaigns in three European
Tokamak Devices (Asdex-Upgrade (Germany),  TCV (Switzerland), MAST-U (UK)) in collaboration with Session leaders,  establish scientific
objectives and monitoring scientific activities.}

\\[-4pt]

\cvitem{2018}{Scientific Coordinator of JET Task T18-02 
  \emph{Scrape-off layer and SOL- pedestal interaction }
  under JET1 Eurofusion Work-Package. Coordination implies assigning activities to the
international 
experimental team, plan data analysis campaign and interface with
Scientific Coordinators of different experiments.}

\end{cvblock}

}
