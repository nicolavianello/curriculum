%%% LaTeX Template: Article/Thesis/etc. with colored headings and special fonts
%%%
%%% Source: http://www.howtotex.com/
%%% Feel free to distribute this template, but please keep to referal to http://www.howtotex.com/ here.
%%% February 2011

%%%%% Preamble
\documentclass[12pt,a4paper]{article}
\usepackage[english]{babel}
\usepackage{geometry}
\geometry{a4paper,margin=2.5cm}
\usepackage[latin1]{inputenc}							% Input encoding
\usepackage{amsmath}									% Math
\usepackage{fontspec,lipsum}
\defaultfontfeatures{Ligatures=TeX}
\setromanfont{Minion Pro}
\setsansfont{Myriad Pro}
\usepackage{xcolor}
\definecolor{bl}{rgb}{0.0,0.2,0.6} 

\usepackage[style=phys,biblabel=brackets,backend=biber]{biblatex}
\addbibresource{../utils/biblio.bib}
\addbibresource{../utils/others.bib}
\renewcommand{\bibfont}{\normalfont\footnotesize}

\usepackage{sectsty}
\usepackage[compact]{titlesec} 
\allsectionsfont{\color{bl}\scshape\selectfont}

%%%%% Definitions
% Define a new command that prints the title only
\makeatletter							% Begin definition
\def\printtitle{%						% Define command: \printtitle
    {\color{bl} \centering \huge \sc \textbf{\@title}\par}}		% Typesetting
\makeatother							% End definition

\title{Research Pogram of Dr. Vianello}

% Define a new command that prints the author(s) only
\makeatletter							% Begin definition
\def\printauthor{%					% Define command: \printauthor
    {\centering \small \@author}}				% Typesetting
\makeatother							% End definition

\author{%
   \vspace{10pt}
	N. Vianello\\
	\vspace{20pt}
	}

% Custom headers and footers
\usepackage{fancyhdr}
	\pagestyle{fancy}					% Enabling the custom headers/footers
\usepackage{lastpage}	
	% Header (empty)
	\lhead{}
	\chead{}
	\rhead{}
	% Footer (you may change this to your own needs)
	\lfoot{\footnotesize Dr. Vianello Research Plan}
	\cfoot{}
	\rfoot{\footnotesize page \thepage\ of \pageref{LastPage}}	% "Page 1 of 2"
	\renewcommand{\headrulewidth}{0.0pt}
	\renewcommand{\footrulewidth}{0.4pt}

% Change the abstract environment
\usepackage[runin]{abstract}			% runin option for a run-in title
\setlength\absleftindent{30pt}		% left margin
\setlength\absrightindent{30pt}		% right margin
\abslabeldelim{\quad}						% 
\setlength{\abstitleskip}{-10pt}
\renewcommand{\abstractname}{}
\renewcommand{\abstracttextfont}{\color{bl} \small \slshape}	% slanted text


%%% Start of the document
\begin{document}
%%% Top of the page: Author, Title and Abstact
\printtitle 

\printauthor
\subsubsection*{3D Physics in present and future Fusion devices}
3D effects are becoming extremely important in the fusion research. 
Also intrinsic two-dimensional configurations as Tokamaks are now
fostering research 
in the application of the magnetic perturbation, which naturally breaks
the symmetry of the configuration, 
in particular with the aim of controlling plasma instabilities such as
the Edge Localized Mode. 
The possibility of controlling these instabilities is mandatory for
the future devices. 
A bunch of work remains to be done to understand the consequences of
3D fields 
and magnetic perturbation in confined plasma. 
Planned research foresees the study of effects of 3D field in:
\begin{description}
\item[(a) Profiles: ] Detailed experimental characterization of
  edge and SOL modification caused by magnetic perturbations in fusion
  devices including tokamaks, reversed field pinches and
  stellarators (international collaborations are foreseen). Emphasis will
  be devoted to the determination of the reasons causing the
  density pump-out observed in varios experiments highlighting the
  enhanced transport channel and the role of high-$k$ turbulence in
  determining this process. This require the analysis of the interaction
  between perturbed magnetic surfaces and plasma turbulence.
\item[(b) Edge flow:] Edge flow and edge radial electric field are 
  found to be modified by magnetic perturbation. Helical radial
  electric field arises as a consequence of ambipolar response. Still
  lack of knowledge exists on the spatial relationship between applied
  perturbation and ambipolar response and the role of
  collisionality/viscosity or plasma density in determining the spatial phase
  relation between vector and scalar potential remains unknown. 
\item[(c) Core region: ]Proposed activity regards the role of 3D fields
in controlling MHD modes 
as RWM or tearing instabilities or tailoring of sawtooth activity, and in the effect of 3D fields in the
modification of 
core rotation of both plasma and MHD modes. Indeed these effects are
mandatory in order to understand 
the possibility to modify flow profile by opportunely governing
magnetic perturbation by exploring 
the role of Neoclassical Toroidal Viscosity and electromagnetic torque
in the plasma. 
This research line may be exploited using data obtained both in Reversed Field Pinches and in Tokamaks
\item[(d) Energetic particles: ]ITER and future devices will operate with a large population of fast particles, 
both as fusion reaction products, or because of high energy Neutral
Beam Injection 
(ITER Neutral Beam Injector will operate with a maximum energy of 1
MeV). 
Thus research on Energetic Particle driven modes are mandatory in
order to anticipate the challenges future magnetic devices.
Research has to be pursued by means of theoretical and numerical
models (turbulence spreading of energetic particles, energetic
particle modes stability as examples) 
but also experimentally. 
For this porpoise experiments may be considered in small 
devices with fast ion sources (NBI or ICRH) or in general  with detailed studies of the interaction between MHD modes and fast
particles. Tools for controlling MHD modes or modification of profiles
through non-axysimmetric magnetic field can be considered for
perturbative studies of energetic particle redistribution.
  
\end{description}

As an ancillary subject a research activity on 
\emph{Spontaneous and induced rotation} is proposed. 
A strong effort has been devoted by the fusion   community to the
comprehension of 
the spontaneous rotation, both toroidal and poloidal rotation,
observed in fusion relevant plasmas. 
Understanding the physics underlying this topic is fundamental, as spontaneous
rotation would help in the control of 
dangerous MHD instabilities in future devices. Various theories have
been proposed including theories 
regarding the role of turbulence and residual stress or Up-down
asymmetry. Furthermore little knowledge 
exists on the mechanism governing momentum transport from the core to
the edge of toroidal plasmas. 
This task may be studied using perturbative approach such as biased
limiter or divertor plates, 
pulsed NBI momentum transfer or varying edge magnetic ripple (which
has been shown to produce 
effects on spontaneous rotation) or applying Resonant and
non-resonanant magnetic perturbation. On this topic experimental
effort should be devoted on studying flow generation around magnetic
islands and difference arising from the application of resonant and
non-resonant magnetic perturbation. Momentum transfer may be studied
also using spontaneous 
effects such as Relaxation events in Reversed Field Pinches or
sawtooth crashes exploring
in this way the effect 
of electromagnetic torque on plasma rotation.
\vspace{0.5cm}

All the proposed activity
share commonalities among all the major magnetic configuration but also with different scientific communities in
particular with astrophysical plasmas: indeed interaction between
magnetic islands and magnetic field with energetic particles, or
modification of high-$k$ turbulence by means of magnetic field are all
topics with great resonances also in astrophysical
science. Beneficial results can be obtained by strengthening
collaborations within all these communities and additional benefits could
result from the already established collaborations of the candidate
also with solar physics
groups.


% \printbibliography[title=Personal publications,notkeyword=others]
% \printbibliography[title=Other Sources, keyword=others]

\end{document}
