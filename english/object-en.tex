\begin{cvblock}{Diagnostic realisation}
\cvitem{Extrap-T2R Alfv{\'e}n Probe:} {Project responsible for the
  probe-head dubbed \emph{ Alfv{\'e}n probe} mounted on edge
  manipulator on the Extrap-T2R Reversed Field Pinch Experiment. The
  system represent a fist of a kind prototype for the investigation of
  electromagnetic turbulence. The role implied the definition of the
  scientific objectives,  the supervision of mechanical and
  electronical assembly, the implementation of acquisition system. The
  probe head contributed to different scientific peer-reviewed paper
  among which we can cite \cite{Vianello:2005p1976, Vianello:2006p1149}}
   
\cvitem{RFX-mod U-probe:}{ Project responsible for the probe-head
  dubbed \emph{U-probe} installed on the edge manipulator on RFX-mod.
  The probe head is capable of withstanding high heat flux (up to 40
  MW for 0.2s) and can provide detailed spatial and temporally
  resolved information on electromagnetic turbulence including direct
  measurement of local current density and plasma vorticity. As a
  prokect leader I've defined the conceptual design, supervised the
  mechanical realisation as well as 'installata sui manipolatori inseribili
  dell'esperimento RFX-mod. Il sistema rappresenta una struttura
  complessa e all'avanguardia operante in condizioni impegnative dal
  punto di vista dei carichi termici
  (fino a 40 MW per 0.2 s) per quali campi elettrici, corrente,
  densità e temperatura con elevata risoluzione spaziale
  (misure in un piano 2 dimensionale perpendicolari al campo magnetico
  guida in una regione di 5x13 cm con passo 6 mm)
  e temporale (fino a 0.2 microsecondi di risoluzione temporale). Il
  ruolo di responsabilit{\'a'} ha implicato: la definizione degli
  obbiettivi scientifici, il progetto meccanico in collaborazione con
  ingegneri e disegnatori meccanici, il progetto di cablaggio ed
  alimentazione elettrica, il sistema di trasporto e digitalizzazione
  del segnale, l'installazione e la messa in funzionamento,  la
  definizione del programma di storage e analisi dati. Il sistema di
  misura ha contribuito a diversi studi scientifici i pi{\'u}
  rilevanti dei quali sono stati \cite{Spolaore:2009p4115,
    Vianello:2010p4670, Zuin:2009p3794, Vianello:2016bm}. Ruolo di
  responsabile della diagnostica.}

\cvitem{TJ-II U-probe}{ Co-Responsabile della progettazzione, 
  supervisione e realizzazione della testa di misura
  denominata U-probe installata sui manipolatori inseribili
  dell'esperimento TJ-II. Il sistema permette la misura di campi elettrici, corrente,
  densità e temperatura con elevata risoluzione spaziale e temporale.  Il
  ruolo di responsabilit{\'a'} ha implicato: la definizione degli
  obbiettivi scientifici, il progetto meccanico in collaborazione con
  ingegneri e disegnatori meccanici, il progetto di cablaggio ed
  alimentazione elettrica, il sistema di trasporto e digitalizzazione
  del segnale, l'installazione e la messa in funzionamento,  la
  definizione del programma di storage e analisi dati. Il sistema di
  misura ha contribuito a diversi studi scientifici il pi{\'u}
  rilevante dei quali {\'e} stato \cite{Spolaore:2015ij}}

\cvitem{Torpex Current Probe}{ Co-Responsabile della progettazzione, 
  supervisione e realizzazione della testa di misura
  denominata Current Probe installata sull'esperimento Torpex.
  Il sistema permette la misura della densit{\'a} di corrente locale
  indotta da strutture turbolente.  Il
  ruolo di responsabilit{\'a'} ha implicato: la definizione degli
  obbiettivi scientifici, il progetto meccanico, il progetto di
  cablaggio,
  il sistema di trasporto e digitalizzazione
  del segnale, l'installazione e la messa in funzionamento,  la
  definizione del programma di storage e analisi dati.   Il sistema di
  misura ha contribuito a diversi studi scientifici i pi{\'u}
  rilevanti dei quali sono stati \cite{Furno:2011cs,
    Fasoli:2013gj,0741-3335-53-12-124016}}

  \cvitem{ASDEX-Upgrade IPP Probe}{Co-Responsabile della progettazzione, 
  supervisione e realizzazione della testa di misura
  denominata IPP probe installata sull'esperimento ASDEX-Upgrade.
  Il sistema permette la misura del campo elettrico,  della
  densit{\'a} di plasma e delle fluttuazioni magnetiche locali.  Il
  ruolo di responsabilit{\'a'} ha implicato: la definizione degli
  obbiettivi scientifici, il progetto meccanico, il progetto di
  cablaggio,
  il sistema di trasporto e digitalizzazione
  del segnale, l'installazione e la messa in funzionamento,  la
  definizione del programma di storage e analisi dati.   Il sistema di
  misura ha contribuito a diversi studi scientifici i pi{\'u}
  rilevanti dei quali sono stati \cite{PhysRevLett.106.125002,
    Naulin:2011im, Muller:2011kj}}

  \cvitem{Compass U-Probe}{ Co-Responsabile della progettazzione, 
  supervisione e realizzazione della testa di misura
  denominata U-probe installata sull'esperimento Compass.
  Il sistema permette la misura del campo elettrico,  della
  densit{\'a} di plasma e della densit{\'a} di corrente locale e delle
  relative fluttuazioni.  Il
  ruolo di responsabilit{\'a'} ha implicato: la definizione degli
  obbiettivi scientifici, il progetto meccanico, il progetto di
  cablaggio,
  il sistema di trasporto e digitalizzazione
  del segnale, l'installazione e la messa in funzionamento,  la
  definizione del programma di storage e analisi dati.   Il sistema di
  misura ha contribuito a diversi studi scientifici i pi{\'u}
  rilevanti dei quali sono stati \cite{Kovarik:2017bp, Kovarik:2014tq, Spolaore:2016bo}}

  \cvitem{W7-X High Resolution Probe}{Collaborazione alla progettazzione, 
  del testa di misura denominata High Resolution Probe presso
  l'esperimento W7-X. 
  Il sistema permette la misura del campo elettrico,  della
  densit{\'a} di plasma, della temperatura e della densit{\'a} di corrente locale e delle
  relative fluttuazioni in due posizioni radiali differenti simultaneamente.    Il sistema di
  misura ha contribuito a diversi studi scientifici i pi{\'u}
  rilevanti dei quali sono stati \cite{spolaore:jinstr2019,
    Agostinetti:2018bm}}

  \cvitem{Sistema ISIS per RFX-mod}{ Collaborazione alla
    realizzazione,  sviluppo e mantenimento del sistema ISIS,
    costituito da un vasto e complesso insieme di sonde
    elettrostatiche,
    magnetiche e termiche interne alla camera da vuoto
    dell'esperimento RFX
    (il numero totale di sensori è superiore a 250).
    il valore della diagnostica (stimato in difetto) è di circa
    700Keuro. Ruolo di co-responsabile della diagnostica. Il sistema di
  misura ha contribuito a diversi studi scientifici i pi{\'u}
  rilevanti dei quali sono stati \cite{Rea:2015he,Vianello:2015ek, Agostini:2014fk,Spizzo:2014jn, Vianello:2013jt}}

\end{cvblock}
%%% Local Variables:
%%% mode: latex
%%% TeX-master: "../cvnicola-alternative"
%%% End:
