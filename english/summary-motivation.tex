% \begin{cvblock}{Motivation}
% \end{cvblock}
% I'm a physicists with \FPtrunc\mydegree{\mydegree}{0}\mydegree\ years
% experience on plasma physics, with primary focus on fusion science, but with a
% broad range of interests from solar to atomic physics and low
% temperature plasmas. As a Deputy Task Force leader of the largest Work
% Package of EUROfusion Fusion Science department, I've proven capabilities of managing large
% international scientific program, which span from high-level objectives and
% priority definition, to public and private stake-holders interactions 
% up to day-to-day operational decision.
% The present historical period, with revamped interest on plasma and 
% fusion science in particular, represents a unique opportunity to take
% advantage of the large spectrum of expertise existing in within the Institute of Plasma Science
% and Technology (ISTP). An inclusive approach, with a synergical and coordinated interaction of
% the different spirits
% constituting this research institute, could foster and
% strengthen the role that the ISTP will play in future national and
% international endeavors. Large research infrastructures which will come
% into operation in the near future as RFX-mod2 or DTT,  as well as
% participation to international fusion and non fusion programs
% represent the natural framework where ISTP can play a leading role,
% attracting both public and private investments,  and
% where each of the peculiarities constituting the institute can find the
% proper allocation and scientific relevance. I believe that the strength
% of my scientific background as well as my experience in designing and
% managing large international scientific program can provide me with
% the capabilities for a challenging position as being the director of the ISTP.

% Bla bla bla
\begin{cvblock}{Summary}
\end{cvblock}
I've been involved in fusion plasma science since my M.Sci. in
Physics in 1999.
The ultimate goal of my research path is focused on the use of thermonuclear fusion as an energy alternative with high energy density, low environmental impact, and potentially unlimited supply. To accelerate the understanding of confinement in magnetic toroidal systems, I have focused on the mechanisms that regulate particle and energy transport, combining rigorous analysis of experimental data with the most advanced techniques and theoretical interpretation supported by numerical modeling. 
The physics of plasma of thermonuclear interest is inherently a
multidisciplinary study. Consequently 
over the past 
\FPtrunc\mydegree{\mydegree}{0}\mydegree\ years
I have developed scientific and technical skills in various areas such
as plasma transport mechanism and plasma turbulence, Scrape-Off-Layer
(SOL) and divertor physics,  dynamical and chaotic systems, advanced statistical analysis,  
atomic and molecular physics,
high-performance computing, plasma-material interactions,
the mechanical design of diagnostics, power and control electronics,
data acquisition and processing methods. All these competences have
been applied not only to fusion relevant plasmas on different magnetic configurations 
 (Reversed Field Pinches, Tokamaks, and Stellarators) but as well
 extending my interest towards solar physics due to
 the strong similarities between turbulence in laboratory and
 astrophysical plasmas. 

Particular emphasis has been given to integrated fusion plasma
scenarios, aiming  to mimic conditions similar to those of a reactor
(in terms of plasma performance, electron density, and use of
radiating impurities) to support the design and definition of the
operating point of next step devices. 

During my M.sci thesis, I conducted experiments to improve the
performance of a plasma confined in a Reversed Field Pinch
configuration by applying a polarized electrode and modifying the
electric field in the outer region of the experiment. These initial
activities, published in  \cite{Antoni:2000p3587} allowed me to
acquire skills in advanced
data analysis methods for small-scale plasma fluctuations and edge diagnostics, including Langmuir probes.

During my PhD research, I further expanded these skills through
periods spent in other European laboratories, particularly at the
Alfvén Laboratory, KTH, Stockholm.  I was responsible for the design,
installation and scientific exploitation of various diagnostics,
including a probe combining electrostatic and magnetic measurements to
extend the analysis of turbulence to the electromagnetic
component. This experience expanded my scientific knowledge of
anomalous transport phenomena occurring at the edge of RFPs, as well technical expertise in areas such as ultra-high vacuum compatible materials, signal transmission, vacuum technologies, data acquisition and management.

During the same period, I also started a fruitful collaboration with experts in  solar wind
plasma: this collaboration allowed me to broaden my experience in nonlinear
dynamics (including self-organized criticality systems) and analysis
of multifractal systems and advanced techniques (wavelets, POD,
etc.). I played a significant role in the development of these
techniques for thermonuclear plasma, establishing the intermittent
nature of thermonuclear plasma fluctuations \cite{Vianello:2002p3579, Carbone:2002p2809, Sattin:2005p1561, Sattin:2004p1318, Spolaore:2004p245} , their relationship with macroscopic MHD phenomena \cite{Antoni:2001p662}, and the inapplicability of SOC dynamic models \cite{Antoni:2001p3221, Spada:2001p3574}.

Afterwards,  my interest then shifted towards the investigation of multi-scale interaction
mechanisms, particularly towards the spontaneous generation of momentum in
plasma induced by electromagnetic turbulence. This topic is crucial in
fusion relevant plasmas, 
due to the role played by plasma flow in stabilizing instabilities,
especially considering that future reactors will not have active
methods of momentum injection. I unequivocally demonstrated the role
of the Reynolds stress mechanism in momentum generation in an RFP
plasma as published in Physical Review Letters and other journals
\cite{Vianello:2005p1976, Vianello:2005p2671, Antoni:2005p3340} and
presented at various conferences and workshops
\cite{vianello:rfp05b, vianello:ipels05, vianello:aps04,
  vianello:ttf04, vianello:rpf04, vianello:ttf07}
with a significant impact on the scientific community.

With the development of a new diagnostic, I further focused my
scientific interest on the electromagnetic nature of plasma
instabilities. For the first time I directly measured the current
associated with turbulent plasma filaments \cite{Spolaore:2009p4115}
and identified the coupling of drift-Alfvén mechanisms as the origin
of these fluctuations \cite{Vianello:2010p4670}. These discoveries
have been the subject of invited talks and a series of publications,
highlighting the universality of this non-linear coupling with strong
similarities with similar observations in magnetospheric studies \cite{Martines:2009p4483}.

Concerning the relevance of laboratory plasma for the understanding of
astrophysical relevant plasma phenomena, I recently expanded to
include collaborations in the study of astrophysical plasma through my
participation in a bilateral collaboration agreement with the Mullard
Space Science Laboratory, UCL London, responsible for some of
diagnostic diagnostics in the Solar Orbiter mission.

The resonance of the results and methodology regarding the
determination of current filaments has led to collaborations at the
European level, with the installation of conceptually similar diagnostics in other
experiments such as the TJ-II stellarator (ES) \cite{Spolaore:2015ij},
the Torpex machine (CH) \cite{Furno:2011cs, Fasoli:2013gj,
  0741-3335-53-12-124016}, the ASDEX-Upgrade tokamak (DE)
\cite{PhysRevLett.106.125002, Naulin:2011im}, COMPASS-U
\cite{Spolaore:2016bo, Kovarik:2017bp}, and the W7X stellarator
\cite{Agostinetti:2018bm, spolaore:jinstr2019}. In these
collaborations, I have contributed to the design, installation, and
scientific exploitation of mentioned diagnostics in all
experiments. In particular, I have published the first measurement of
the current perturbation associated with Edge Localized Modes (modes
that occur in improved confinement regimes and potentially dangerous
for first wall materials) \cite{PhysRevLett.106.125002} as well as
contributed to the publication providing clear experimental evidence
validating an analytical model of current perturbation associated with
blobs in conditions similar to those in the Scrape Off Layer region of
a tokamak \cite{Furno:2011cs}.

In addition to the work performed in the framework of these
international collaborations, I have actively contributed to the study
of Reversed Field Pinch configurations through my participation in the
RFX-mod project.  I am a co-author of the paper on the discovery of
low-dissipation helical regimes with improved confinement
\cite{Lorenzini:2009p4248}, published in Nature Physics. My primary
research interest on this respect has been the analysis of the effects of three-dimensional magnetic perturbations on edge plasmas \cite{Vianello:2013jt, Scarin:2013ci, Spizzo:2012hw, Agostini:2014fk}, including transport \cite{Rea:2015he, Vianello:2015ek}, plasma-wall interaction \cite{Scarin:2014dm, DeMasi:go}, plasma velocity modification, and observed turbulence \cite{Ciaccio:2014bo, Spizzo:2014jn}. The importance of these contributions is evident from invited talks at specialized workshops and invited participation in the European Physical Society Conference on Plasma Physics, as well as the number of publications for which I am an author or co-author. I have also co-supervised a PhD thesis dedicated to this topic .

Since 2014, my activity concentrated on the study of transport
properties in the outer region of tokamak plasmas. Such an activity
implied the participation to the experimental activity in several
European devices,  through my involvement in various Work Packages of
the EUROfusion Consortium. The EUROfusion Consortium coordinates
European fusion activities and participates in experimental
exploitation of the major national experiments. I have been the
Scientific Coordinator continuously since 2014 for several experiments
conducted at ASDEX-Upgrade, TCV, and JET, the world's largest fusion
experiment. I focused my scientific effort on the analysis transport
phenomena in the Scrape Off Layer region under conditions similar to
those expected for future reactors as well as on the investigation of
alternative regimes without ELMs, such as the M-mode observed for the
first time at JET \cite{Solano:2017db, Refy:NuclearFusion2020}. This
research strongly contributed to the understanding of the relationship
between the Scrape Off Layer and the Divertor region, where most of
the exhausted power is deposited once it leaves the confined
region. It is worth remembering that integrated scenarios,  combining
confinement properties with effective power management strategies are
currently among the most urgent challenges in fusion research. During
this period, thanks to the scientific recognition, I was invited to spend 1 year at the Swiss Plasma Center, Ecole Polytechnique Federale de Lausanne, as a Scientific Collaborator.

The experiments I've coordinated  between 2014 and 2020 contributed to 
significant improvements on the understanding of
the transport occurring in the peripheral region of the plasma, in particular concerning the modification
during high-density plasma operations \cite{Stagni:nf2024,
  stagni:nf2022, Carralero:prl2015, Vianello:2017ku,
  vianello:nf2019, Carralero:2017gb, tsui:pop2022},
the modification caused by different divertor recycling conditions \cite{Lomanowski2023, lomanowski:nf2022,
  kevin:nf2019, Reimerdes:2017cp}, or by different divertor topological geometries \cite{Vianello:2017ku, theiler:nf2017}. These studies have advanced our integrated understanding, highlighting the key role played by collisionality at the magnetic separatrix \cite{Stagni:nf2024, vianello2023h, stagni:nf2022, vianello:iaea2021sol} and neutral density in the divertor or midplane region \cite{agostini:ppcf2019, vianello:nf2019}. The relevance of this work is evidenced by the number of publications for which I am the first author or co-author as the experiment coordinator.

The role of Scientific Coordinator implied the establishment of  experimental/analysis/modeling strategy to achieve
specific scientific deliverables,  the careful planning of the
experimental sessions in tight collaboration with the device
operational and diagnostic teams,  the organization of large
international teams for the best usage of peculiar expertise in an
synergistic approach. The management of a budget of approximately 2.5
M\euro for
dedicated machine time and approximately 5 FTE (full-time equivalent)
of coordinated scientific personnel can be quantified as a minimum. My
research work has also involved managing a research group and
supervising doctoral students whose research topics were designed to
broaden my scientific horizons in numerical modeling
\cite{Mancini:nf2023, mancini:nf2021} and integrated analysis of
kinetic, fluctuation, and calorimetric diagnostics
\cite{Redl:ppcf2023, Redl2023}. The recognition of this work is
evident from my election (through peer reviewed selection) as a
European member of the International Tokamak Physics Activity (ITPA)
Pedestal and Edge (PEP) group, as well as my active participation as a
task coordinator in the ITPA Divertor and SOL group. 

From November 2020,  following a highly competitive 
selection, I've been 
chosen as the Deputy Task Force Leader of the EUROfusion Tokamak
Exploitation Work Package (WPTE). WPTE is the largest work package of
the EUROfusion Department
of Fusion Science and is in charge of developing the experimental
program on different tokamak devices, namely JET (UK), MAST-U (UK),
ASDEX-Upgrade (AUG), TCV (CH), and WEST (FR).
WPTE aims to develop an experimental and modeling/interpretation
program to determine plasma operational scenarios for a future
reactor, including the
management of rapid and disruptive transients, active control
mechanisms to
determine confinement conditions in real-time, and the dynamics of
superthermal particles.
This program also integrates efficient power exhaust management solutions,
such as scenarios with
radiating impurities or magnetic geometries favorable for heat load
dissipation, as well as studying the impact of such scenarios on
plasma-facing materials.
As a Task Force Leader (TFL) I have defined the large scale experimental program,
organized it into several Research Topics
determining specific scientific objectives, allocated
experimental time based on
scientific priorities in different devices, selected Scientific
Coordinators proposed by
national associations, allocated human resources to form the best
scientific team
to achieve the scientific objectives, and monitored their
activities. I'm in charge as well of establishing and prioritizing
international collaborations with non-European members,  ensuring the possibility to increase
European scientific expertise and promoting European science to the
worldwide fusion community. 
On top of high level scientific programming, the Deputy
TFL role include as well day-to-day management, including rapid
programmatic responses to changes in machine operating conditions. 
The overall total allocation of WPTE is approximately 100 M\euro in 5
years, to be spent for device exploitation,
human resources and travel costs to facilitate integration among
personnel from different European laboratories.
The program I contributed to develop is fully embedded into the 
EUROfusion Consortium program, and it is the responsibility of the
TFLs to
identify and monitor the Deliverables and Milestones that are integral parts of
the grant agreement between
EUROfusion and the European Commission.
Consequently throughout my career I've demonstrated a strong and
robust scientific multidisciplinary background, a vital tendency to develop scientific 
network, a proved capability to construct, coordinate and monitor
large pan-European scientific program while interacting with various European and national
stakeholders. The undeniable role of responsibility clearly places me
at the center of international fusion research, and this would allow
me to promote the Italian contribution to the field.  
All these expertise qualifies my proposal for the role of Director
of the Institute of Plasma Science and Technology of the CNR. 

%%% Local Variables:
%%% mode: latex
%%% TeX-master: "../cvnicola-alternative"
%%% End:
