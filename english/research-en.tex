\iftoggle{nicolacv}{
\section{Summary of research interest}
I've been involved in fusion plasma science since my M.Sci. thesis in
Physics in 1999. During these 13 years I've tried to expand as much as
possible my personal research skills focusing in particular on collection, analysis, interpretation and modeling of experimental data
collected in fusion oriented experiments (Reversed Field Pinches,
Tokamaks and Stellarators), with particular emphasis on the comparison
with theoretical and numerical results. Main research subjects may be summarized as follow: 

\begin{longtable}{@{}>{\bfseries}p{.1\textwidth} p{.9\textwidth}@{}}
(a) & \textbf{Electromagnetic turbulence induced transport:}, with
emphasis on anomalous transport studies induced by different source of
turbulence: electrostatic as Drift-induced or interchange induced
transport, or electromagnetic including the role of magnetic flutter fluxes
in the mechanism of particle and energy losses\\
(b) & \textbf{Statistical analysis of plasma turbulence:} the topic
allowed me to get confident with advanced statistical tool (as Wavelet Transforms, Local Intermittency
  Measurements, Waiting Time distribution) and with dynamical system
  model as Self-Organized Criticality (SOC) systems, shell-models  \\
(c) & \textbf{Blobs and ELM filaments:} non linear coherent structures
arising as a non-linear evolution of plasma
instabilities have been experimentally investigated. The research
includes studies on the generation and evolution of these
structures including
their parallel dynamics with emphasis on turbulent \emph{blobs} and
ELM \emph{filaments}  \\
(d) & \textbf{Sheared flow generation} and non linear interaction
between turbulence and sheared flows including experimental
investigation of the role of Maxwell and Reynolds stress in the
momentum generation of edge flow in Reversed Field Pinches  \\
(e) & \textbf{Numerical modeling of electromagnetic plasma turbulence}
  using fluid approach \\
(f) & \textbf{Magnetic topology and its relation with plasma flow}, with
emphasis on the effect of non-axysimmetric magnetic field perturbation on kinetic
properties of the plasma, as plasma flow, ambipolar electric field and
Plasma Wall Interaction\\
(g) & \textbf{Beam plasma interaction} with emphasis on Alfv\'en instabilities,
Energetic Particle Driven instabilities, and turbulent transport of
energetic ions
\end{longtable}

Among the results the following should be highlighted:
\begin{longtable}{@{}>{\bfseries} p{.1\textwidth} p{.9\textwidth}@{}}
(i) &  First experimental proof of non applicability of \emph{Self
    Organized Criticality} paradigm to edge plasma
  turbulence \parencite{Spada:2001p3574,Antoni:2001p3221} \\
(ii) &  First experimental evidence of non-linear generation of edge
  flow in Reversed Field Pinches through Reynolds stress
  mechanism \parencite{Vianello:2005p1976,Vianello:2005p2671} \\
(iii) & First experimental measurements of parallel current associated
  to coherent structures in a fusion relevant plasma \parencite{Spolaore:2009p4115} \\
(iv) & First experimental evidence of the existence of a particular 
  class of coherent structure, named \emph{Drift-Kinetic Alfv\'en
    vortices}, arising because of the non linear coupling of Drift and
  Kinetic Alfv\'en waves in a laboratory plasma \parencite{Vianello:2010p4670}. This type of structure has been
  previously detected in the magnetosphere \\
(v) &  First experimental estimate of parallel current associated to
  Edge Localized Modes filament \parencite{PhysRevLett.106.125002} \\
(vi) & First experimental measurements of 2D current distribution
associated to plasma blobs \parencite{Furno:2011cs} \\
(vii) & Experimental evidence of transition towards helical states in
high current Reversed Field Pinch
operation \parencite{Lorenzini:2009p4248} \\
\end{longtable}

In all my carrier I've always tried to conjugate a strong experimental
insight on the data collection, participating in all the experimental
activities mandatory in order to obtain useful experimental results, and a rigorous theoretical approach in
the data analysis and interpretation, using theories and numerical
tools as a framework to understand real plasma signals. This
approach helped me to build a bridge between theories and experiments,
a necessary effort in order to understand complex plasma dynamics.
}
{\begin{cvblock}{Research Interest}


  \end{cvblock}
I've been involved in fusion plasma science since my M.Sci. thesis in
Physics in 1999. During these 13 years I've tried to expand as much as
possible my personal research skills focusing in particular on collection, analysis, interpretation and modeling of experimental data
collected in fusion oriented experiments (Reversed Field Pinches,
Tokamaks and Stellarators), with particular emphasis on the comparison
with theoretical and numerical results. Main research subjects may be summarized as follow: 
\setlength\LTleft{0.75in}
\setlength\LTright{1in}
\begin{longtable}{@{}>{\bfseries} p{.1\textwidth} p{.75\textwidth}@{}}
(a) & \textbf{Electromagnetic turbulence induced transport:}, with
emphasis on anomalous transport studies induced by different source of
turbulence: electrostatic as Drift-induced or interchange induced
transport, or electromagnetic including the role of magnetic flutter fluxes
in the mechanism of particle and energy losses\\
(b) & \textbf{Statistical analysis of plasma turbulence:} the topic
allowed me to get confident with advanced statistical tool (as Wavelet Transforms, Local Intermittency
  Measurements, Waiting Time distribution) and with dynamical system
  model as Self-Organized Criticality (SOC) systems, shell-models  \\
(c) & \textbf{Blobs and ELM filaments:} non linear coherent structures
arising as a non-linear evolution of plasma
instabilities have been experimentally investigated. The research
includes studies on the generation and evolution of these
structures including
their parallel dynamics with emphasis on turbulent \emph{blobs} and
ELM \emph{filaments}  \\
(d) & \textbf{Sheared flow generation} and non linear interaction
between turbulence and sheared flows including experimental
investigation of the role of Maxwell and Reynolds stress in the
momentum generation of edge flow in Reversed Field Pinches  \\
(e) & \textbf{Numerical modeling of electromagnetic plasma turbulence}
  using fluid approach \\
(f) & \textbf{Magnetic topology and its relation with plasma flow}, with
emphasis on the effect of non-axysimmetric magnetic field perturbation on kinetic
properties of the plasma, as plasma flow, ambipolar electric field and
Plasma Wall Interaction\\
(g) & \textbf{Beam plasma interaction} with emphasis on Alfv\'en instabilities,
Energetic Particle Driven instabilities, and turbulent transport of
energetic ions
\end{longtable}

Among the results the following should be highlighted:
\begin{longtable}{@{}>{\bfseries} p{.1\textwidth} p{.75\textwidth}@{}}
(i) &  First experimental proof of non applicability of \emph{Self
    Organized Criticality} paradigm to edge plasma
  turbulence \parencite{Spada:2001p3574,Antoni:2001p3221} \\
(ii) &  First experimental evidence of non-linear generation of edge
  flow in Reversed Field Pinches through Reynolds stress
  mechanism \parencite{Vianello:2005p1976,Vianello:2005p2671} \\
(iii) & First experimental measurements of parallel current associated
  to coherent structures in a fusion relevant plasma \parencite{Spolaore:2009p4115} \\
(iv) & First experimental evidence of the existence of a particular 
  class of coherent structure, named \emph{Drift-Kinetic Alfv\'en
    vortices}, arising because of the non linear coupling of Drift and
  Kinetic Alfv\'en waves in a laboratory plasma \parencite{Vianello:2010p4670}. This type of structure has been
  previously detected in the magnetosphere \\
(v) &  First experimental estimate of parallel current associated to
  Edge Localized Modes filament \parencite{PhysRevLett.106.125002} \\
(vi) & First experimental measurements of 2D current distribution
associated to plasma blobs \parencite{Furno:2011cs} \\
(vii) & Experimental evidence of transition towards helical states in
high current Reversed Field Pinch
operation \parencite{Lorenzini:2009p4248} \\
\end{longtable}

In all my carrier I've always tried to conjugate a strong experimental
insight on the data collection, participating in all the experimental
activities mandatory in order to obtain useful experimental results, and a rigorous theoretical approach in
the data analysis and interpretation, using theories and numerical
tools as a framework to understand real plasma signals. This
approach helped me to build a bridge between theories and experiments,
a necessary effort in order to understand complex plasma dynamics.

\printbibliography[type=article,  title = {Cited publications}, heading=subbibliography, prefixnumbers={A}, resetnumbers=true]
}
