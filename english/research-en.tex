\begin{cvblock}{Motivation}
\end{cvblock}
\begin{cvblock}{Summary}
 \end{cvblock}.
I've been involved in fusion plasma science since my M.Sci. in
Physics in 1999. During these
\FPtrunc\mydegree{\mydegree}{0}\mydegree\ years
I have focused on developing my skills on analysis,
interpretation and modeling of experimental data from various
magnetic configurations (Reversed Field Pinches, Tokamaks, and
Stellarators) in order to actively contribute to the development of
Fusion as a credible alternative in the energy sources landscape.

In all my carrier I've always tried to conjugate a strong experimental
insight on the data collection, participating in all the experimental
activities mandatory in order to obtain useful experimental results, and a rigorous theoretical approach in
the data analysis and interpretation, using innovative theories and numerical
tools as a framework to understand real plasma signals. This
approach helped me to build a bridge between theories and experiments,
a necessary effort in order to understand complex plasma dynamics.

During this period I've addressed several aspects of plasma physics,
primarily but not exclusively related to 
fusion science. A non exhaustive summary of my main research interest
and main achievements is listed below: 

\begin{description}[labelindent=0.5pt, labelsep*=0.4em, leftmargin=!, itemsep=0.05ex]
\item[(a) Multiscale Electromagnetic turbulence and induced transport:] with
emphasis on anomalous transport studies induced by different source of
turbulence: electrostatic as Drift-induced or interchange induced
transport, or electromagnetic including the role of magnetic flutter fluxes
in the mechanism of particle and energy losses. Such a topic has been
investigated in all the major magnetic confinement configurations \cite{Vianello:2002p3579, Vianello:2017ku}
\item[(b) Statistical analysis of plasma turbulence:] the topic
allowed me to get confident with advanced statistical tool (as Wavelet Transforms, Local Intermittency
  Measurements, Waiting Time distribution) and with dynamical system
  model as Self-Organized Criticality (SOC) systems,
  shell-models. Among the main results on this topic we might cite the
  first experimental proof of non applicability of \emph{Self
    Organized Criticality} paradigm to edge plasma
  turbulence \cite{Spada:2001p3574,Antoni:2001p3221}. Furthermore this represented an important point of
  contact with astrophysical plasmas, in particular of solar physics,
  with national and international collaboration, the latter further
  fostered with a recent bilateral agreement with the Royal Society
  and the Mullard Space Science Laboratory
\item[(c) Blobs and ELM filaments:] non linear coherent structures
arising as a non-linear evolution of plasma
instabilities have been experimentally investigated. The research
includes studies on the generation and evolution of these
structures including
their parallel dynamics with emphasis on turbulent \emph{blobs} and
ELM \emph{filaments}. Outstanding results on this topic are: the first experimental measurements of parallel current associated
   to coherent structures in a fusion relevant
   plasma \cite{Spolaore:2009p4115},   the first experimental evidence of the existence of  \emph{Drift-Kinetic Alfv\'en
    vortices}, arising because of the non linear coupling of Drift and
  Kinetic Alfv\'en waves in a laboratory plasma
  \cite{Vianello:2010p4670}, with strong similarities with
  astrophysical plasmas \cite{Martines:2009p4483}, the first experimental estimate of parallel current associated to
   Edge Localized Modes filament \cite{PhysRevLett.106.125002} and the
   first experimental measurements of 2D current distribution
associated to plasma blobs \cite{Furno:2011cs}
\item[(d) Sheared flow generation:] Non linear interaction
between turbulence and sheared flows including experimental
investigation of the role of Maxwell and Reynolds stress in the
momentum generation of edge flow. On this respect the main result is
the first experimental evidence of non-linear generation of edge
  flow in Reversed Field Pinches through Reynolds stress
  mechanism \cite{Vianello:2005p1976,Vianello:2005p2671}
\item[(e) Magnetic topology and its relation with plasma flow:] with
emphasis on the effect of non-axysimmetric magnetic field perturbation on kinetic
properties of the plasma, as plasma flow, ambipolar electric field and
Plasma Wall Interaction. I've actively contributed on this topic on
the experimental evidence of transition towards helical states in
high current Reversed Field Pinch
operation \cite{Lorenzini:2009p4248} and its consequence on edge
ambipolar electric field  \cite{Spizzo:2014jn}
\item[(f) Divertor and SOL physics:] with emphasis on the
  modification of upstream SOL and pedestal structures depending on
  the recycling condition of the divertor. This represents my main
  research topic and among all the results we can cite the experimental investigation on the role of \emph{blobs} in the
  formation of the so-called \emph{shoulder} in density gradient in
  high density regime \cite{Carralero:prl2015, vianello:nf2019, stagni:nf2022, Stagni:nf2024}
\item[(g) 3D self consistent simulation of edge plasma:] This topic
  has been addressed primarily through my role of PhD supervisor and
  among the various results we can cite the first simulations of plasma detachment in fully developed 3D
turbulence code including neutral-plasma interaction \cite{Mancini:nf2023}
\end{description}

Throughout the year I have developed a rich  international collaboration network,
primarily grown  thanks to the works performed in
different experiments in particular in Europe. From 2014 to 2020 I've
acted as Scientific or Task Coordinators for different experiments
within various EUROfusion Consortium Work-packages. These type of
roles implied the establishment of  experimental/analysis/modeling strategy to achieve
specific scientific deliverables,  the careful planning of the
experimental session in tight collaboration with the device
operational and diagnostic teams,  the organization of large
international teams for the best usage of peculiar expertise in an
unified approach. The value of the scientific achievements obtained as
Scientific Coordinator, allow me to be selected as european memeber of
the International Tokamak Physics Activity (ITPA) Pedestal and Edge
(PEP) group.

From November 2020,  following a highly competitive peer review
selection, I've been 
chosen as the Deputy Task Force Leader of the EUROfusion Tokamak
Exploitation Work Package (WPTE). WPTE is the largest work package of
the EUROfusion Department
of Fusion Science and is in charge of developing the experimental
program on different tokamak devices, namely JET (UK), MAST-U (UK),
ASDEX-Upgrade (AUG), TCV (CH), and WEST (FR).
WPTE aims to develop an experimental and modeling/interpretation
program to determine plasma operational scenarios for a future
reactor, including the
management of rapid and disruptive transients, active control
mechanisms to
determine confinement conditions in real-time, and the dynamics of
suprathermal particles.
This program also integrates efficient energy management solutions,
such as scenarios with
radiating impurities or magnetic geometries favorable for heat load
dissipation, as well as studying the impact of such scenarios on
plasma-facing materials.
As a Task Force Leader (TFL) I have defined the experimental program,
organized it into several Research Topics
determining specific scientific objectives, allocated
experimental time based on
scientific priorities in different devices, selected Scientific
Coordinators proposed by
national associations, allocated human resources to form the best
scientific team
to achieve the scientific objectives, and monitored their
activities. This program
has an allocation of 65M\euro for experiments, 24 M\euro for human
resource determination,
and 3.5M\euro for travel costs to facilitate integration among
personnel from different European laboratories.
It is important to emphasize that the program I contributed to develop is part
of the overall
EUROfusion Consortium program, and it is the responsibility of the
TFLs to
identify the Deliverables and Milestones that are integral parts of
the grant agreement between
EUROfusion and the European Commission.
Consequently throughout my career I've demonstrated a strong and
robust scientific multidisciplinary background, a vital tendency to develop scientific 
network, a proved capability to construct, coordinate and monitor
large pan-European scientific program while interacting with various European and national
stakeholders. All these qualifies my proposal for the role of Director
of the Institute of Plasma Science and Technology of the CNR. 

%\printbibliography[type=article,  title = {Cited publications}, heading=subbibliography, prefixnumbers={A}, resetnumbers=true]

