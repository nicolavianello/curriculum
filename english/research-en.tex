\section{Brief summary of research interest}
Since my M.Sci. thesis my primary research interest has been the
collection, analysis, interpretation and modeling of experimental data
collected in fusion oriented experiment (Reversed Field Pinches,
Tokamaks and Stellarators), with particular emphasis on the comparison
with theoretical and numerical results. Main research subject may be summarized as follow: 

\begin{longtable}{>{\bfseries}l p{17cm}}
(a) & Electromagnetic turbulence induced transport in the
  external region of magnetically confined plasmas \\
(b) & Statistical properties of plasma turbulence, investigated
  trough advanced tools as Wavelet Transforms, Local Intermittency
  Measurements, Waiting Time distribution \\
(c) & Coherent structure generation as results of non-linear
  evolution of plasma instabilities \\
(d) & Non linear interaction between turbulence and sheared flows \\
(e) & Sheared flow generation mechanism through non-diagonal part
  of stress tensor \\
(f) & Numerical modeling of electromagnetic plasma turbulence
  using fluid approach \\
(g) & Filaments and coherent structure with emphasis on
  \emph{blobs} observed in laboratory plasmas and \emph{Edge Localized
  Modes} \\
(e) & Magnetic topology and its relation with plasma flow, with
emphasis on the effect of 3D magnetic field perturbation on kinetic
properties of the plasma \\
(f) & Beam plasma interaction with emphasis on Alfv\'en instabilities,
Energetic Particle Driven instabilities, and turbulent transport of
energetic ions
\end{longtable}

The principal results obtained may be summarized as
follow:
\begin{longtable}{>{\bfseries}l p{17cm}}
(i) &  First experimental proof of non applicability of \emph{Self
    Organized Criticality} paradigm to edge plasma
  turbulence \parencite{Spada:2001p3574,Antoni:2001p3221} \\
(ii) &  First experimental evidence of non-linear generation of edge
  flow in Reversed Field Pinches through Reynolds stress
  mechanism \parencite{Vianello:2005p1976,Vianello:2005p2671} \\
(iii) & First experimental measurements of parallel current associated
  to coherent structures in a fusion relevant plasma \parencite{Spolaore:2009p4115} \\
(iv) & First experimental evidence of the existence of a particular 
  class of coherent structure, named \emph{Drift-Kinetic Alfv\'en
    vortices}, arising because of the non linear coupling of Drift and
  Kinetic Alfv\'en waves in a laboratory plasma \parencite{Vianello:2010p4670}. This type of structure has been
  previously detected in the magnetosphere \\
(v) &  First experimental estimate of parallel current associated to
  Edge Localized Modes filament \parencite{PhysRevLett.106.125002} \\
(vi) & First experimental measurements of 2D current distribution
associated to plasma blobs \parencite{Furno:2011cs} \\
(vii) & Experimental evidence of transition towards helical states in
high current Reversed Field Pinch
operation \parencite{Lorenzini:2009p4248} \\
\end{longtable}

In all my carrier I've always tried to coniugate a strong experimental
insight on the data collection, participating in all the experimental
activities mandatory in order to obtain useful experimental results, and a rigorous theoretical approach in
the data analysis and interpretation, using theories and numerical
tools as a framework to understand real plasma signals. This
approach helped me to build a bridge between theories and experiments,
a necessary effort in order to understand complex plasma dynamics.
