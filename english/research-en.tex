\section{Research interest}
Plasma Physics and Magnetic Confinement Fusion (Reversed Field
Pinches and Tokamaks) both experimentally and via numerical modelling,
with emphasis on:
\begin{description}
\item[(a)] Electromagnetic turbulence induced transport in the
  external region of magnetically confined plasmas
\item[(b)] Statistical properties of plasma turbulence, investigated
  trough advanced tools as Wavelet Transforms, Local Intermittency
  Measurements, Waiting Time distribution
\item[(c)] Coherent structure generation as results of non-linear
  evolution of plasma instabilities
\item[(d)] Non linear interaction between turbulence and sheared flows
\item[(e)] Sheared flow generation mechanism through non-diagonal part
  of stress tensor
\item[(f)] Numerical modeling of electromagnetic plasma turbulence
  using fluid approach
\item[(g)] Filamentation of coherent structure with emphasis on
  \emph{blobs} observed in laboratory plasmas and \emph{Edge Localized
  Modes}
\end{description}
The principal results obtained by Dr. Vianello may be summarized as
follow:
\begin{itemize}
\item First experimental proof of non applicability of \emph{Self
    Organized Criticality} paradigm to edge plasma turbulence \cite{Spada:2001p3574,Antoni:2001p3221}
\item First experimental evidence of non-linear generation of edge
  flow in Reversed Field Pinches through Reynolds stress mechanism \cite{Vianello:2005p1976,Vianello:2005p2671}
\item First experimental measurements of parallel current associated
  to coherent structures in a laboratory plasma \cite{Spolaore:2009p4115}
\item First experimental evidence of the existence of a particular
  class of coherent structure, named \emph{Drift-Kinetic Alfv\'en
    vortices}, arising because of the non linear coupling of Drift and
  Kinetic Alfv\'en waves in a laboratory plasma \cite{Vianello:2010p4670}. This type of structure has been
  previously detected in the magnetosphere
\item First experimental estimate of parallel current associated to
  Edge Localized Modes filament \cite{PhysRevLett.106.125002}
\end{itemize}
