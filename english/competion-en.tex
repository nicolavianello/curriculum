\section{Competition}
\iftoggle{nicolacv}{
\begin{entrylist}
\entrytwo
{05 2009}
{Public selection (Ref. 364/13) held by Consiglio
  Nazionale delle Ricerche. Advisor Committee:
  \begin{itemize}
  \item Prof. A. Fasoli, Full Professor, Ecole Polytechnique Federal Lausanne, Switzerland
  \item Dr. V. Antoni, Director Istituto Gas Ionizzati, Consiglio Nazionale delle Ricerche
  \item Dr. D. Farina, Research Scientist, Istituto di Fisica del
    Plasma, Consiglio Nazionale delle ricerche, Milano
  \end{itemize}
  The competition included two written exams
  and one colloquium. The candidate results the winner of the
  competition with a final mark of 104.5/120}
\end{entrylist}
\begin{entrylist}
\entrytwo
{2012}
{National Scientific Qualification (Abilitazione Scientifica
  Nazionale). Public evaluation of the competences and scientific
  achievements to obtain the qualification of \emph{Professore
    Associato} (Associate Professor) in Experimental Physics on
  Material Science (settore concorsuale 02/B1 Fisica Sperimentale
  della Materia). Grade: Excellent  
}
\end{entrylist}
}{

\begin{tabular}{>{\bfseries}l p{13cm}}
May 2009 & Public selection (Ref. 364/13) held by Consiglio
  Nazionale delle Ricerche. Advisor Committee:
  \begin{itemize}
  \item Prof. A. Fasoli, Full Professor, Ecole Polytechnique Federal Lausanne, Switzerland
  \item Dr. V. Antoni, Director Istituto Gas Ionizzati, Consiglio Nazionale delle Ricerche
  \item Dr. D. Farina, Research Scientist, Istituto di Fisica del
    Plasma, Consiglio Nazionale delle ricerche, Milano
  \end{itemize}
  The competition included two written exams
  and one colloquium. The candidate results the winner of the
  competition with a final mark of 104.5/120 
% June 2011 & Public selection (Ref. VL-2010-0130) held by Royal
% Institute of Technology, Stockholm for the position of \emph{Associate
% Professor in Fusion Plasma Physics with emphasis on analysis of
% experiment data}. Expert committee:
% \begin{itemize}
% \item Prof. Tunde F\"ulop, Chalmers University of Technology,
%   Gotheborg, Sweden 
% \item Prof. Steven A. Sabbagh, Adjunct Professor, Columbia University, USA
% \end{itemize} 
\end{tabular}}
