\iftoggle{nicolacv}{
\section{Competition and Habilitation}
\begin{entrylist}
\entrytwo
{05 2009}
{Public selection (Ref. 364/13) held by Consiglio
  Nazionale delle Ricerche. Advisor Committee:
  \begin{itemize}
  \item Prof. A. Fasoli, Full Professor, Ecole Polytechnique Federal Lausanne, Switzerland
  \item Dr. V. Antoni, Director Istituto Gas Ionizzati, Consiglio Nazionale delle Ricerche
  \item Dr. D. Farina, Research Scientist, Istituto di Fisica del
    Plasma, Consiglio Nazionale delle ricerche, Milano
  \end{itemize}
  The competition included two written exams
  and one colloquium. The candidate results the winner of the
  competition with a final mark of 104.5/120}
\end{entrylist}
\begin{entrylist}
\entrytwo
{2012}
{National Scientific Qualification (Abilitazione Scientifica
  Nazionale). Public evaluation of the competences and scientific
  achievements to obtain the qualification of \emph{Professore
    Associato} (Associate Professor) in Experimental Physics on
  Material Science (settore concorsuale 02/B1 Fisica Sperimentale
  della Materia). Grade: Excellent  
}
\end{entrylist}
}{
\begin{cvblock}{Competition and Habilitation}
% ------------
 \cvitem{May 2009}{Publich selection (Ref.364/12) held by Consiglio
  Nazionale delle Ricerche, for research position}
\cvitem{Evaluation panel}{Prof. A. Fasoli, Ecole Polytechnique Federale de Lausanne,
  Switzerland}
\cvitem{}{Dr. V. Antoni, Consiglio Nazionale delle Ricerche,
Istituto Gas Ionizzati,  Padova}
\cvitem{}{Dr. D. Farina,  Consiglio Nazionale delle Ricerche, Istituto di
  Fisica del Plasma, Milano}
\cvitem{Result}{The competition included written exams and oral
  colloquium. The candidate resulted the winner of the competition
  with a final mark of 104.5/120}
% ------------

\\[-4pt]

\cvitem{2012}{Abilitazione Scientifica Nazionale, Bando D.D. 222/2012,  (ASN National
  Scientific Habilitation). Public evaluation of the competences and
  scientific achievements to obtain the qualification of
  \emph{Professore Associato} (Associate Professor) in Experimental
  Physics and Material Science }
\cvitem{Evaluation Panel}{Prof. Mattera Lorenzo, Universit{\'a} degli
  Studi di Genova, Italy}
\cvitem{}{Prof. Rinaldo Cubeddu, Politecnico di Milano, Italy}
\cvitem{}{Prof. Stefano Nannarone, Universit{\'a} degli Studi di
  Modena e Reggio Emilia, Italy}
\cvitem{}{Prof. Mobilio Settimio, Universit{\'a} degli Studi di Roma
  Tre, Italy}
\cvitem{}{Prof. Andrea Cavalleri, Max Planck Institute for the
Structure and Dynamics of Matter, Hamburg}
\cvitem{Grade}{Excellent}
\cvitem{}{Valid from 11/12/2013 to 11/12/2019}

\\[-4pt]

% ------------

\cvitem{2018}{Abilitazione Scientifica Nazionale, Bando D.D. 1532/2016,  (ASN National
  Scientific Habilitation). Public evaluation of the competences and
  scientific achievements to obtain the qualification of
  \emph{Professore Ordinario} (Full Professor) in Experimental
  Physics and Material Science } 
\cvitem{Evaluation Panel}{Prof. Federico Boscherini, Universit{\'a} degli
  Studi di Bologna}
\cvitem{}{Prof. Giulio Nicola Cerullo, Politecnico di Milano}
\cvitem{}{Prof.ssa Pasqualino Maria Maddalena, Universit{\'a} degli Studi di
  Napoli}
\cvitem{}{Prof. Francesco Saverio Pavone, Universit{\'a} degli Studi di Firenze}
\cvitem{}{Prof. Sandro Santucci, Universit{\'a} degli Studi
  dell'Aquila}
\cvitem{}{Valid from 26/07/2018 to 26/07/2029}
%------------------
\\[-4pt]
\cvitem{2018}{Abilitazione Scientifica Nazionale, Bando D.D. 1532/2016,  (ASN National
  Scientific Habilitation). Public evaluation of the competences and
  scientific achievements to obtain the qualification of
  \emph{Professore Associato} (Associate Professor) in Experimental
  Physics and Material Science}
\cvitem{Evalaution Panel}{Prof. Federico Boscherini, Universit{\'a} degli
  Studi di Bologna}
\cvitem{}{Prof. Giulio Nicola Cerullo, Politecnico di Milano}
\cvitem{}{Prof. Pasqualino Maria Maddalena, Universit{\'a} degli Studi di
  Napoli}
\cvitem{}{Prof. Francesco Saverio Pavone, Universit{\'a} degli Studi di Firenze}
\cvitem{}{Prof. Sandro Santucci, Universit{\'a} degli Studi
  dell'Aquila}
\cvitem{}{Valid from 26/07/2018 to 26/07/2029}
%------------------
\\[-4pt]
\cvitem{2018}{Abilitazione Scientifica Nazionale, Bando D.D. 1532/2016 (ASN National
  Scientific Habilitation). Public evaluation of the competences and
  scientific achievements to obtain the qualification of
  \emph{Professore Associato} (Associate Professor) in Theoretical
  Physics of Matter}
\cvitem{Evaluation Panel}{Prof. Federico Boscherini, Universit{\'a} degli
  Studi di Bologna}
\cvitem{}{Prof.ssa Vincenza Cupri, Universit{\'a} degli Studi di
  Messina}
\cvitem{}{Prof. Amos Maritan, Universit{\'a} degli Studi di Padova}
\cvitem{}{Prof. Alessandro Tredicucci, Universit{\'a} degli Studi di Pisa}
\cvitem{}{Prof. Pierluigi Veltri, Universit{\'a} della Calabria}
\cvitem{}{Abilitazione valida dal 08/08/2018 al 08/08/2029}

%------------------
\\[-4pt]
\cvitem{2020}{Consiglio Nazionale delle Ricerche (National Research
  Council) Procedure N. 315.15 PR for the promotion to the level of
  \emph{Senior Researcher} (Ricercatore I Livello)}
\cvitem{Evaluation Panel}{Prof. Stefano Zapperi, Universit{\'a} degli
  Studi di Milano}
\cvitem{}{Dott. Michael Pusch, Istituto di biofisica (IBF)-CNR,  Genova}
\cvitem{}{Dott.ssa Paola Mantica, Istituto per la scienza e tecnologia
  dei plasmi (ISTP)-CNR, Milano}
\cvitem{}{The candidate has been evaluated as eligible and promoted to
  the level of Senior Researcher from 01/01/2023}
\end{cvblock}
% \begin{tabular}{>{\bfseries}l p{13cm}}
% May 2009 & Public selection (Ref. 364/13) held by Consiglio
%   Nazionale delle Ricerche. Advisor Committee:
%   \begin{itemize}
%   \item Prof. A. Fasoli, Full Professor, Ecole Polytechnique Federal Lausanne, Switzerland
%   \item Dr. V. Antoni, Director Istituto Gas Ionizzati, Consiglio Nazionale delle Ricerche
%   \item Dr. D. Farina, Research Scientist, Istituto di Fisica del
%     Plasma, Consiglio Nazionale delle ricerche, Milano
%   \end{itemize}
%   The competition included two written exams
%   and one colloquium. The candidate results the winner of the
%   competition with a final mark of 104.5/120 
% June 2011 & Public selection (Ref. VL-2010-0130) held by Royal
% Institute of Technology, Stockholm for the position of \emph{Associate
% Professor in Fusion Plasma Physics with emphasis on analysis of
% experiment data}. Expert committee:
% \begin{itemize}
% \item Prof. Tunde F\"ulop, Chalmers University of Technology,
%   Gotheborg, Sweden 
% \item Prof. Steven A. Sabbagh, Adjunct Professor, Columbia University, USA
% \end{itemize} 
% \end{tabular}
}
