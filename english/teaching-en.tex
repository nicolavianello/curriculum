\iftoggle{nicolacv}{
\section{Pedagogical activity}
\subsection*{Teaching}
\begin{entrylist}
\entry
{2008-2009}
{Assistant for the course \emph{Fluid and Plasma Physics}}
{University of Padova} {tenured Prof. T. Bolzonella. Tot. 4 h.  Seminar on MHD and Fluid turbulence. 
A summary is presented on the theory and experimental results on turbulence, both in
ordinary fluid and in plasmas. A description of the most
recent results regarding turbulence and eddy’s characterization in
thermonuclear relevant plasmas is given. Exercises on fluid
turbulence}

\entry
{2010}{Assistant for the course \emph{Fluid and Plasma Physics}}
{University of Padova} {tenured Prof. T. Bolzonella. Tot. 6 h.
  Tangential stresses in ordinary fluid. Seminar on MHD and fluid
  turbulence (see previous years)}

\entry
{2011-2012}{Assistant to the course \emph{Fundamental of Plasma
    Physics}}{University of Padova}{Tenured Prof. G. Serianni. Tot
  8h. Lectures on plasma waves including Plasma Oscillations,
  Langmuir waves,  Ion acoustic waves,  Upper and Lower Hybrid waves,
Whistler waves and MHD waves (magneto-acoustic and Alfv\'en waves)}

\entry
{2012-2013}{Assistant to the course \emph{Fundamental of Plasma
    Physics}}{University of Padova}{Tenured Prof. G. Serianni. Tot
  10h. Lectures on plasma waves including Plasma Oscillations,
  Langmuir waves,  Ion acoustic waves,  Upper and Lower Hybrid waves,
Whistler waves and MHD waves (magneto-acoustic and Alfv\'en waves),
plasma stability}

\entry
{2013-2014}{Assistant to the course \emph{Fundamental of Plasma
    Physics}}{University of Padova}{Tenured Prof. G. Serianni. Tot
  10h. Lectures on plasma waves including Plasma Oscillations,
  Langmuir waves,  Ion acoustic waves,  Upper and Lower Hybrid waves,
Whistler waves and MHD waves (magneto-acoustic and Alfv\'en waves),
plasma stability}

\entry
{2013-2014}{Joint Research Doctorate
and European Interuniversity Doctoral Network on Fusion Science and Engineering}{University of Padova}
{Lecturer for basic Physics course. Tot 4h. Lecture on Transport and Turbulence}

\entry
{2014-2015}{Assistant to the course \emph{Fundamental of Plasma
    Physics}}{University of Padova}{Tenured Prof. G. Serianni. Tot
  10h. Lectures on plasma waves including Plasma Oscillations,
  Langmuir waves,  Ion acoustic waves,  Upper and Lower Hybrid waves,
Whistler waves and MHD waves (magneto-acoustic and Alfv\'en waves),
plasma stability}

\end{entrylist}

\subsection*{Supervising}
\begin{entrylist}
\entry
{2007}{Supervisor for Bachelor Thesis in Physics}{University of
  Padova}
{ \textbf{Candidate:} Alessandro Scaggion \\
\textbf{Thesis title:} \emph{Electrostatic fluctuations characterization in
   RFX-mod experiment in different experimental condition} \\
\textbf{Thesis subject:} Characterization of floating potential
 measurements as obtained from an internal array of sensors in
 different discharge conditions highlighting dependence on
 equilibrium and density. 
}

\entry
{2009}{Supervisor for M.Sci Thesis in Physics}{University of
  Padova}
{ \textbf{Candidate:} Alessandro Scaggion \\
\textbf{Thesis title:} \emph{Filamentary structures in the edge
  turbulence of fusion device} \\
\textbf{Thesis subject:} Characterization of turbulence
electromagnetic structures in two different devices: RFX-mod Reversed
Field Pinch experiment, characterized by the presence of
Drift-Alfv\'en filaments, and ASDEX-Upgrade tokamak, with emphasis on
type I ELM's filaments.
}

\entry
{2011}{Supervisor for Bachelor Thesis in Physics}{University of
  Padova}
{ \textbf{Candidate:} Alberto Mazzi \\
\textbf{Thesis title:} \emph{Experimental evaluation of toroidal
   velocity distribution in the edge region of RFX-mod and its impact
   on high density regimes} \\
\textbf{Thesis subject:} Experimental determination of the
 spatio-temporal distribution of the toroidal
 velocity  in RFX-mod and its relationship with edge magnetic
 topology. The strong link between magnetic islands and
 plasma flow distribution is addressed.
}

\entry
{2013}{Supervisor for M.Sci. Thesis in Physics}{University of
  Padova}
{ \textbf{Candidate:} Alberto Mazzi \\
\textbf{Thesis title:} \emph{Analysis of vorticity coherent structures
in magnetically confined plasmas} \\
\textbf{Thesis subject:} Experimental characterization of vorticity
and parallel current associated to the existence of plasma structure
at large scales (associated to MHD activity) and small scales
(associated to fully developed MHD turbulence)
}

\entry
{2015}{Supervisor for PhD Thesis in Physics}{University of
  Padova}
{ \textbf{Candidate:} Cristina Rea \\
\textbf{Thesis title:} \emph{} \\
\textbf{Thesis subject:} Experimental characterization of the effect
of a 3D magnetic field on electrostatic fluctuations responsible for
turbulence induced transport. Comparative analysis between different
magnetic configuration)
}

\end{entrylist}
}
{
\begin{cvblock}{Teaching}
\cvitem{2008--2009}{Assistant for the course \emph{Fluid and Plasma
    Physics}, Department of Physics, University of Padova}

\\[-6pt]

\cvitem{2010}{Assistant for the course \emph{Fluid and Plasma
    Physics}, Department of Physics, University of Padova}

\\[-6pt]

\cvitem{2011--2012}{Assistant to the course \emph{Fundamental of
    Plasma Physics}, Department of Physics, University of Padova }

\\[-6pt]

\cvitem{2012-2013}{Assistant to the course \emph{Fundamental of
    Plasma Physics}, Department of Physics, University of Padova }

\\[-6pt]

\cvitem{2013-2014}{Assistant to the course \emph{Fundamental of
    Plasma Physics}, Department of Physics, University of Padova }

\\[-6pt]

\cvitem{2013-2014}{Lecturer for basic Physics course of the Joint Research Doctorate
and European Interuniversity Doctoral Network on Fusion Science and Engineering}

\\[-6pt]

\cvitem{2014-2015}{Assistant to the course \emph{Fundamental of
    Plasma Physics}, Department of Physics, University of Padova }
\end{cvblock}
\begin{cvblock}{Supervising}
\cvitem{2007}{Supervisor for Bachelor Thesis, Department of Physics,
  University of Padova, candidate: A. Scaggion}

\\[-6pt]

\cvitem{2009}{Supervisor for M.Sci. Thesis, Department of Physics,
  University of Padova, candidate: A. Scaggion}

\\[-6pt]

\cvitem{2011}{Supervisor for Bachelor Thesis, Department of Physics,
  University of Padova, candidate: A. Mazzi}

\\[-6pt]

\cvitem{2013}{Supervisor for M.Sci. Thesis, Department of Physics,
  University of Padova, candidate: A. Mazzi}

\\[-6pt]

\cvitem{2015}{Supervisor for PhD. Thesis, Department of Physics,
  University of Padova,  candidate: C. Rea}
\end{cvblock}
% \section{Pedagogical activities}
% \subsection{Teaching}
% \begin{longtable}{>{\bfseries}l p{15cm}}
% 2008-2009 & Assistant for the course \emph{Fluid and Plasma
%   Physics} \\
% & tenured
% Prof. Tommaso Bolzonella  \\
%  & Total h 4  \\ 
% & \textbf{Subject}: Seminar on MHD and Fluid turbulence. 
% A summary is presented on the theory and experimental results on turbulence, both in
% ordinary fluid and in plasmas. A description of the most
% recent results regarding turbulence and eddy’s characterization in
% thermonuclear relevant plasmas is given. Exercises on fluid turbulence \\
% 2010 &  Assistant for the course \emph{Fluid and Plasma
%   Physics} \\ 
% & tenured
% Prof. Tommaso Bolzonella  \\
%  & Total h 6  \\
% & \textbf{Subject}: Tangential stress in ordinary
% fluids. Seminar on MHD and Fluid turbulence (see previous years)  \\
% 2011-2012 & Assistant for the course \emph{Fundamentals of Plasma
%   Physics} \\ 
% & tenured Prof. Gianluigi Serianni \\
% & Total h. 10  \\
% & \textbf{Subject}: Plasma oscillations, Langmuir Waves, Ion Acoustic
% waves, Upper and Lower hybrid waves, Whistler waves, MHD waves
% (magneto-acoustic, Alfv\'en waves)
% \end{longtable}

% \subsection{Supervising}
% \begin{longtable}{>{\bfseries}l p{15cm}}
% 2007 & Supervisor for Bachelor Thesis in Physics, University of Padova \\
%  & \textbf{Candidate:} Alessandro Scaggion \\
%  & \textbf{Thesis title:} \emph{Electrostatic fluctuations characterization in
%    RFX-mod experiment in different experimental condition} \\
%  & \textbf{Thesis subject:} Characterization of floating potential
%  measurements as obtained from an internal array of sensors in
%  different discharge conditions highlighting dependence on
%  equilibrium and density. \\
% 2009 & Supervisor for M.Sci. Thesis in Physics, University of Padova \\
%  & \textbf{Candidate:} Alessandro Scaggion \\
% & \textbf{Thesis title:} \emph{Filamentary structures in the edge
%   turbulence of fusion devices} \\
% & \textbf{Thesis subject:} Characterization of turbulence
% electromagnetic structures in two different devices: RFX-mod Reversed
% Field Pinch experiment, characterized by the presence of
% Drift-Alfv\'en filaments, and ASDEX-Upgrade tokamak, with emphasis on
% type I ELM's filaments \\
% 2011 & Supervisor for Bachelor Thesis in Physics, University of Padova \\
%  & \textbf{Candidate:} Alberto Mazzi \\
%  & \textbf{Thesis title:} \emph{Experimental evaluation of toroidal
%    velocity distribution in the edge region of RFX-mod and its impact
%    on high density regimes } \\
%  & \textbf{Thesis subject:} Experimental determination of the
%  spatio-temporal distribution of the toroidal
%  velocity  in RFX-mod and its relationship with edge magnetic
%  topology. The strong link between magnetic islands and
%  plasma flow distribution is addressed.\\

% \end{longtable}
}
