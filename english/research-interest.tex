%%% LaTeX Template: Article/Thesis/etc. with colored headings and special fonts
%%%
%%% Source: http://www.howtotex.com/
%%% Feel free to distribute this template, but please keep to referal to http://www.howtotex.com/ here.
%%% February 2011

%%%%% Preamble
\documentclass[12pt,a4paper]{article}
\usepackage[english]{babel}
\usepackage{geometry}
\geometry{a4paper,margin=2.5cm}
\usepackage[latin1]{inputenc}							% Input encoding
\usepackage{amsmath}									% Math
\usepackage{fontspec,lipsum}
\defaultfontfeatures{Ligatures=TeX}
\setromanfont{Minion Pro}
\setsansfont{Myriad Pro}
\usepackage{xcolor}
\definecolor{bl}{rgb}{0.0,0.2,0.6} 

\usepackage[style=phys,biblabel=brackets,backend=biber]{biblatex}
\addbibresource{../utils/biblio.bib}
\addbibresource{../utils/others.bib}
\renewcommand{\bibfont}{\normalfont\footnotesize}

\usepackage{sectsty}
\usepackage[compact]{titlesec} 
\allsectionsfont{\color{bl}\scshape\selectfont}

%%%%% Definitions
% Define a new command that prints the title only
\makeatletter							% Begin definition
\def\printtitle{%						% Define command: \printtitle
    {\color{bl} \centering \huge \sc \textbf{\@title}\par}}		% Typesetting
\makeatother							% End definition

\title{Personal Research Interest of Dr. Vianello}

% Define a new command that prints the author(s) only
\makeatletter							% Begin definition
\def\printauthor{%					% Define command: \printauthor
    {\centering \small \@author}}				% Typesetting
\makeatother							% End definition

\author{%
   \vspace{10pt}
	N. Vianello\\
	\vspace{20pt}
	}

% Custom headers and footers
\usepackage{fancyhdr}
	\pagestyle{fancy}					% Enabling the custom headers/footers
\usepackage{lastpage}	
	% Header (empty)
	\lhead{}
	\chead{}
	\rhead{}
	% Footer (you may change this to your own needs)
	\lfoot{\footnotesize Dr. Vianello Research Interest}
	\cfoot{}
	\rfoot{\footnotesize page \thepage\ of \pageref{LastPage}}	% "Page 1 of 2"
	\renewcommand{\headrulewidth}{0.0pt}
	\renewcommand{\footrulewidth}{0.4pt}

% Change the abstract environment
\usepackage[runin]{abstract}			% runin option for a run-in title
\setlength\absleftindent{30pt}		% left margin
\setlength\absrightindent{30pt}		% right margin
\abslabeldelim{\quad}						% 
\setlength{\abstitleskip}{-10pt}
\renewcommand{\abstractname}{}
\renewcommand{\abstracttextfont}{\color{bl} \small \slshape}	% slanted text


%%% Start of the document
\begin{document}
%%% Top of the page: Author, Title and Abstact
\printtitle 

\printauthor

% \begin{abstract}
% Viene qui di seguito descritta una proposta per attivit\`a di
% dottorato da svolgersi presso il gruppo FB. L'attivit\`a risulta
% articolata e prevede lo sviluppo di nozioni sia 
% sperimentali, includendo attivit\`a di allestimento di apparati
% diagnostici, che interpretativi tramite l'analisi dati che
% modellistici. L'attivit\`a \`e volta a creare una migliore comprensione della
% regione esterna di RFX-mod permettendo inoltre di stabilire analogie e
% corrispondenti collaborazioni con attivit\`a analoghe presenti nei
% tokamak nel pi\`u ampio ambito degli studi delle Resonant Magnetic Perturbation.
% \end{abstract}

%%% Start of the 'real' content of the article, using a two column layout
% \section*{Motivazioni}
Dr. Vianello's research activity has been devoted to the analysis,
interpretation and modeling of experimental and numerical results
obtained in the framework of high temperature plasmas, with emphasis
on magnetically confined plasmas. More in details Dr. Vianello focused
his attention on the electromagnetic fluctuations induced transport
phenomena occurring in the edge region of confined plasmas, 
with interpretation in the wider framework of turbulence
theory. 

It is well known that transport phenomena in thermonuclear plasmas are
governed by electromagnetic turbulence, which represents the main
cause of the so-called \emph{anomalous transport}. Scientific interest
is further enhanced by the consideration that magnetically confined
plasmas are a complex and self-organized system, which thus represents
an ideal environment for non-linear dynamics studies.

During his research activity Dr. Vianello has collected a large experience on
electromagnetic transport analysis, working on different
magnetic configurations, from Reversed Field Pinches (working on
RFX-mod operating in Padova, Extrap-T2R operating in Stockholm and
TPE-1RM20 which was in operation in Japan), stellarators (with
experimental activity on TJ-II heliac type operating in Spain) and Tokamaks
(ASDEX-Upgrade and JET). 
In the following a brief overview of the principal results obtained
during his research activity is reported.
\subsubsection*{Electrostatic turbulence induced transport and sheared
  flow interaction}
At the beginning of his career Dr. Vianello studied the effect of
active modification of flow shear, obtained
through edge biasing experiment performed on RFX Reversed Field Pinch,
on electrostatic turbulence induced particle transport
\parencite{Antoni:2000p3587}. Indeed Dr. Vianello has experimentally proved that
flow shear enhancement causes a reduction of particle losses due to the
electrostatic turbulence obtained through a modification of the phase
difference between density and potential fluctuations. This reduction
is not homogeneous in $k$-space but concentrated around wavenumber
where restitive g-modes are expected to be unstable
\parencite{Zuin:2010p4689}. 
\subsubsection*{Intermittency and self-similarity studies}
Strong effort has been devoted to the studies of the
statistical properties of electromagnetic turbulence as measured at
the edge of fusion relevant plasmas. It has been proved that
magnetically confined turbulence exhibits an high degree of
intermittency: this phenomenon is actually responsible for the lack of
self-similarity which causes the breaking of Kolmogorow paradigm of
turbulence energy cascade with the existence of localized (in
time and space) stronger fluctuations which can be better described as
\emph{coherent structure} observed in various devices
\cite{Spolaore:2004p245, Vianello:2002p3579,eps31tpe}. The experimental verification of this lack
of self-similarity has been made possible through a multi-scale
analysis, realized through the application of advanced analysis
technique 
borrowed from fluid dynamics such as
\emph{Continuos Wavelet Transform} and \emph{Local Intermittency
  Measurements}. The observations done by Dr. Vianello and co-authors
allowed also to discriminate the applicability of the so-called
\emph{Self Organized Criticality}  paradigm to fusion plasma
turbulence: indeed it has been proved that this dynamical model does
not apply to fusion plasmas \cite{Spada:2001p3574,Antoni:2001p3221} as
SOC models are inherently self-similar (apart models where
characteristics scales, which are in open contradiction with the
original SOC paradigm, are \emph{ad hoc} introduced). Interestingly
similar observations have been compared with solar wind and
hydrodynamical turbulence\cite{Carbone:2002p2809} showing remarkable similarities.

\subsubsection*{Coherent structures and filaments}
As aforementioned intermittent character of electromagnetic turbulence
is caused by the presence of strong localized fluctuations often
referred as \emph{blobs} or \emph{coherent structure}. Extensive
experimental and interpretative work has been done by Dr. Vianello in
order to characterize completely these fluctuations. These structure
exhibit an higher pressure than the surrounding plasma, with a
vortex-like velocity pattern resembling monopolar or dipolar vortices \cite{Antoni:2006p3585,Serianni:2007p3575}.
They have been
found to strongly contribute to particle and energy transport through
their radial convective motion and also through an enhancement of
particle and heat diffusivity through their merging and coalescence
\cite{Spolaore:2004p245,Spolaore:2005p2243}.
These structures have also been electromagnetically characterized,
providing the first direct measurements of the parallel current
associated to a plasma blob in a thermonuclear relevant plasma
\cite{Spolaore:2009p4115,Spolaore2010}. An accurate interpretation of the
measurements proved also that these blobs, as detected in the edge
region of an RFP plasma are actually \emph{drift-kinetic alfv\'en
  vortices} \cite{Vianello:2010p4670} resulting from the non-linear
coupling of drift and kinetic alfv\'en waves. Indeed pioneering
studies of current filaments associated to plasma blobs have been
extended during the research activity of Dr.Vianello with
measurements performed in stellarators \cite{Spolaore:eps2012}, in
simple magnetized torus \cite{Furno:2011cs,0741-3335-53-12-124016} and
in low current ohmic tokamak \cite{Spolaore:eps2012}. Expertise in
current filaments studies has been extended to the studies of ELM
filaments providing for the first time an experimental direct estimation of the
current density associated to a type-I filament
\cite{PhysRevLett.106.125002,Naulin:2011im,Muller:2011kj}.
\subsubsection*{Momentum transport studies}
Deepening the subject of interaction between turbulence and flow,
Dr. Vianello focused on the turbulence flow generation processes
\cite{Vianello:2005p1976,Vianello:2005p2671}
through Reynolds and Maxwell stresses. Providing pioneering
measurements on Reversed Field Pinches of both the quantities, and
actually first measurements of Maxwell stress in a fusion relevant
plasma, Dr. Vianello proved that, despite the high level of magnetic
fluctuation characterizing the RFP configuration, perpendicular flow
is still driven by electrostatic turbulence. The analysis has been
further extended considering the whole energy transfer process between
fluctuations and mean fields, including Kinetic and Magnetic energies
\cite{Vianello:2006p1149}, trying to establish the whole scheme of
relationship between energetic basins and sinks. Interestingly those
analysis have drained attention very recently as they are suspected to
have a role in driving H mode in tokamak plasmas as shown for example
in \cite{Manz:2012jq,Manz:2012hh}, through the so-called Limit Cycle Oscillations.


\subsubsection*{3D Physics}
In present fusion research a strong effort is devoted to the
comprehension of the effects of a 3D magnetic field on the plasma.
Indeed, apart from the research line devoted to strongly 3D shaped
plasmas as stellarators \cite{Boozer:2009p4091}, non-axysimmetric
magnetic fields are now considered as fundamental also for
configurations long considered as toroidally symmetric, or
inherently 2D. In fact, 3D fields are considered for example as a
viable technique for momentum variation in torqueless tokamak
plasmas \cite{Garofalo:2009p4775,Nave:2010p4812} or for mitigation
of power loads on the divertor~\cite{Evans:2006p2554}. Among  all
of these tools, it is worth mentioning Resonant and Non Resonant
Magnetic perturbations for ELM control, which are presently
installed in all of the major
tokamaks~\cite{Evans:2006p2554,Suttrop:2011bp,Liang:2011ww,Kirk:2010p4626},
and foreseen also for next generation
devices~\cite{LoarteIAEAv:2012ui}. These experiments share the
rationale of the edge ergodization obtained for example in the DED
experiment at TEXTOR~\cite{Lehnen:2005ir}, and more generally to
the island divertor concept developed for present and future
stellarators \cite{0741-3335-44-11-306}.

The edge physics of the improved confinement regime observed in
high current operations in Reversed Field Pinch (RFP)
configuration~\cite{Lorenzini:2009p4248}, whose discovery and
characterization Dr. Vianello has actively contributed, exhibits strong
analogies with physical phenomena observed in 3D magnetic
configurations.

In particular in recent years Dr. Vianello focused his attention on
the interaction between non-axysimmetric magnetic field and flow. It
has contributed to the observation of the role of magnetic islands in
determining the flow pattern associated to the MARFE phenomenon
\cite{Spizzo:2012hw,Spizzo:2010p4796} occurring in high density regime
in an RFP and more recently to the existence of an helical flow
pattern velocity, as the result of an ambipolar electric field
response to a magnetic perturbation in helical Reversed Field Pinches
\cite{Vianello:2012uh}. The  provided experimental evidences exhibit
analogies with the RMP experiments,
suggesting the necessity to correctly consider the spatial phase
relation between experimental observations and magnetic perturbations
also in the case of non-axysimmetric tokamak plasmas. Magnetic
perturbation causes also the modification of pressure profiles, with
different behavior of density and temperature: as a consequence high-k
turbulence has been found to be profoundly influenced
\cite{0741-3335-54-6-065003} and this phenomenon represents the
present subject of investigation of Dr. Vianello.

\printbibliography[title=Personal publications,notkeyword=others]
\printbibliography[title=Other Sources, keyword=others]

\end{document}
