\documentclass[12pt,stdletter,a4paper,dateno,sigleft]{newlfm}
% \usepackage{kpfonts}
\usepackage{url}
%\usepackage{charter}
\usepackage{fontspec,lipsum}
\setromanfont{Minion Pro}
\setsansfont{Myriad Pro}

%\widowpenalty=1000
%\clubpenalty=1000

%for the logo
% \newsavebox{\RFXlogo}
% \sbox{\RFXlogo}{%
% 	\parbox[b]{1.75in}{%
% 		\vspace{0.5in}%
% 		\includegraphics[scale=.2,ext=.pdf]
% 		{RFX-logo}%
% 	}%
% }%
% \makeletterhead{Uiuc}{\Lheader{\usebox{\RFXlogo}}}

% for the signature
\newsavebox{\Sigx} 
\sbox{\Sigx}{%
          \includegraphics[height=7\baselineskip]{signature.pdf}}
\signature{\usebox{\Sigx}}
\makesignature{NV}{\newsavebox{\Signature}} 

\newlfmP{headermarginskip=10pt}
\newlfmP{sigsize=50pt}
\newlfmP{dateskipafter=20pt}
%\newlfmP{paperheight=700pt}
\newlfmP{addrfromphone}
\newlfmP{addrfromemail}
\newlfmP{MinFoot=13pt}
%\newlfmP{MinHead=50pt}
\PhrPhone{Phone}
\PhrEmail{Email}
\encllist{Curriculum Vitae, Reference Letter from Prof. Francesco Gnesotto}
% \lthUiuc
\sigNV

\namefrom{Nicola Vianello}
\addrfrom{%
    \today\\[10pt]
    Consorzio RFX \\
    Associazione EURATOM-Enea sulla Fusione\\
    C.so Stati Uniti 4\\
    35127 Padova, Italy
}
\phonefrom{+39 0498295991}
\emailfrom{nicola.vianello@igi.cnr.it}

\addrto{%
\textbf{To:} Mr. Hans Jahreiss, Fusion for Energy, European Domestic Agency \\
\textbf{Job Title:} Plasma Physics,  Scientific Coordinator \\
\textbf{Ref:}  Senior Scientific Officer POP-009 }

\greetto{To Whom It May Concern,}
\closeline{Sincerely,}
\begin{document}
\begin{newlfm}
my name is Nicola Vianello, I am a 36 years old Phd Physic Scientist,
currently working at the RFX-mod experiment. I would like to apply for
the aforementioned job position

I've been involved in Fusion plasma Science since my M.Sci. Thesis in
Physics in 1999. My primary research interests is transport phenomena in fusion
oriented plasmas with strong emphasis on non-linear dynamics. I have
addressed the problem both experimentally, through the collection, 
analysis, interpretation and modeling of experimental data, and
numerically, through the use of massive parallel fluid codes. 

I've a strong attitude in
data analysis and evaluation with a particular emphasis on the
comparison with theories and codes which provide the suitable framework for the
correct interpretation of real data. I've been involved in
interpretation of a variety of phenomena, ranging from electrostatic
turbulence particle transport, sheared flow turbulence generation,
turbulent non-linearly generated structures, interplay between 3D
magnetic perturbation and plasma transport phenomena. Each of these
topics has required the establishment of a solid theoretical
background, and the adaptation of theories, generally proposed for
different configuration or different plasmas (as for example in the
studie of Drift-Kinetic Alfv\'en turbulence in RFPs, previously
considered in tokamaks or in the magnetosphere). I strongly believe that this
attitude, which combines experimental expertise  with clear and solid
theoretical background provide an additional important element for the
establishment of a comprehensive integrated modeling capability of
tokamak scenarios. 

I've been actively involved in the coordinations of experimental
campaigns in RFX-mod experiment, as task force leader during 2009 and
2010, taking part to the decision processes concerning the
experimental program of the machine, indicating priorities and
objectives. This effort was tightly linked with the theory department
of my institute, in order to design and suggesting adeguate experiments
which could provide answers to open theoretical issues.
In 2011 I've been also coordinator of the 
EFDA working group \emph{3D field effects in
  edge and SOL and diagnostic development} under EFDA Transport
Topical Group umbrella. As a duties I have to coordinate the
activities of different  european laboratories, stimulating
discussion and defining common objectives. I believe these experiences
provide me a good management and coordination capabilities also in an
international environment. 

My international experience is very good, with vital and active
collaborations with different european and international laboratories. 

I think that during my research carrier I have proved good autonomy
accompanied by good capability to work in small and large groups. 

All these qualities and competences fits well with the
requirements for the position. I would be eager to have the possibility to work in such an exciting environment as ITER.\\
Thank you for your consideration. 

\end{newlfm}

\end{document}
